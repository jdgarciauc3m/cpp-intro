\section{Práctica}

\subsection{Objetivo}

El objetivo de esta práctica es ejercitar los siguientes conceptos:

\begin{itemize}

\item Diseño de tipos abstractos de datos.
\item Constructores.
\item Mecanismos de copia.
\item Sobrecarga de operadores.
\item Gestión de errores.
\item Entrada/salida definida por el usuario.
\item Pruebas unitarias.

\end{itemize}

\subsection{Descripción general}

El objetivo de esta práctica es el diseño de un tipo abstracto de datos que
represente una matriz (tipo \cppid{matriz}) bidimensional de números reales
dentro de una biblioteca de álgebra lineal (espacio d enombres \cppid{linalg}).

El tipo de datos diseñado debera soportar la siguiente funcionalidad:

\begin{itemize}
  \item Construcción de una matriz a partir de un número de filas y columnas.
  \item Construcción por defecto.
  \item Consulta de filas y columnas.
  \item Operador de acceso.
  \item Suma de matrices.
  \item Producto de matrices.
  \item Lectura de una matriz de un archivo de texto.
  \item Lectura de una matriz de un archivo binario.
  \item Escritura de una matriz en un archivo de texto.
  \item Escritura de una matriz en un archivo binario.
\end{itemize}

\subsubsection{Representación interna}

Para representar internamente la matriz se utilizarán los siguientes datos
miembro:

\begin{itemize}
  \item \cppid{filas\_}: Número de filas de la matriz. Se representará como un
número entero.
  \item \cppid{columnas\_}: Número de columnas de la matriz. Se representará
como un número entero.
  \item \cppid{vec\_}: Vector unidimensional de números reales en doble
precisión con los valores de la matriz. Se almancenarán primero todos los
elementos de la primera fila, a continuación los de la segunda fila, y así
sucesivamente.  
\end{itemize}

\subsubsection{Organización del código}

La clase \cppid{matriz} debe definirse dentro del espacio de nombres
\cppid{linalg}. Para ello se usará la siguiente organización:

\begin{itemize}
  \item \textmark{matriz.hpp}: Archivo de cabecera con las declaraciones
necesarias.
  \item \textmark{matriz.cpp}: Archivo de implementación con las definiciones
necesarias.
  \item \textmark{matriz\_ut.cpp}: Archivo con las pruebas unitarias necesarias
utilizando la biblioteca \textemph{Google Test}.
enunciado.
\end{itemize}

\subsubsection{Gestión de errores}

Deberá documentar en todas las operaciones las precondiciones y postcondiciones
correspondientes. Siempre que sea posible utilice la biblioteca
\textemph{mincontracts}.

\subsection{Operaciones}

\subsubsection{Construcción}

Se debe poder construir una \cppid{matriz} a partir de un número de filas y un
número de columnas, siendo ambos enteros positivos. El tamaño del vector interno
\cppid{vec\_} se fijará a \cppid{filas\_ * columnas\_} iniciando todos los
elementos del vector a \cppid{0.0}.

\begin{lstlisting}
linalg::matriz a{3,4}; // Matriz de 3 filas y 4 columnas
linalg::matriz b{-1,2}; // Violación de precondición
\end{lstlisting}

\subsubsection{Construcción por defecto}

Se debe poder construir una \cppid{matriz} sin parámetros. En este caso se
fijarán tanto el número de filas como el número de columnas a \cppid{0} y e
tamaño del vector interno \cppid{vec\_} se fijará también a \cppid{0}.

\subsubsection{Consulta de filas y columnas}

\begin{lstlisting}
linalg::matriz a; // Matriz de 0 filas y 0 columnas
\end{lstlisting}

\subsubsection{Operador de acceso}

Redefina el operador \cppid{()} para que tome una posición de la matriz y
devuelva o permita acceder al valor de una celda.

Tenga en cuenta que para esto necesitará definir dos versiones del operador:

\begin{lstlisting}
double operator()(int i, int j) const;
double & operator()(int i, int j);
\end{lstlisting}

Si los índices \cppid{i} o \cppid{j} no son válidos debera tratarse como una
violación de precondición.

Con esta definición, el siguiente código debe ser válido:

\begin{lstlisting}
linalg::matriz a{3,4};
a(0,2) = 5.0;
double x = a(2,3);
a(3,4) = 1.5; // Cuidado violación de precondición.
\end{lstlisting}

\subsubsection{Suma de matrices}

Sobrecargue el operador \cppkey{+} para que se pueda calcular la suma de dos
matrices. Recuerde que para que dos matrices puedan sumarse deben tener el mismo
número de filas y de columnas. Así mismo sobrecargue el operador \cppkey{+=}.

Con estas definiciones, el siguiente código debe ser válido:

\begin{lstlisting}
linalg::matriz a{3,4}, b{3,4};
//...
linalg::matriz c = a + b;
c += a;
\end{lstlisting}

\subsubsection{Producto de matrices}

Sobrecargue el operador \cppkey{*} para que se pueda calcular el producto de dos
matrices. Recuerde que para que dos matrices puedan multiplica deben 
cumplirse que el número de columnas de la primera matriz debe ser igual al
número de filas de la segunda matriz.Así mismo sobrecargue el operador \cppkey{*=}.

Con estas definiciones, el siguiente código debe ser válido:

\begin{lstlisting}
linalg::matriz a{3,3}, b{3,3};
//...
linalg::matriz c = a * b;
c *= a;
\end{lstlisting}

\subsubsection{Escritura de un archivo de texto}

Sobrecargue el operador \cppkey{<<} para que realice la escritura de texto en un
flujo de salida.

Escriba en una línea el número de filas y el número de columnas separados por un
espacio. Posteriormente escriba en líneas separadas todos los valores
correspondientes a cada fila separados entre sí por espacios.

Gestione los errores modificando el estado del flujo.

\subsubsection{Lectura de archivo de texto}

Sobrecargue el operador \cppkey{>>} para que realice la lectura de texto de un
flujo de entrada.

Este operador debe leer correctamente los datos generados por un operador de
salida de texto.

Gestione los errores modificando el estado del flujo.

\subsubsection{Escritura de un archivo binario}

Defina una operación \cppid{guarda()} que tome como argumento un nombre de un
archivo. La operación realizará las siguientes tareas:

\begin{itemize}
  \item Abrir para escritura en modo binario un archivo con ese nombre.
  \item Si el archivo ya existía se eliminará su contenido anterior.
  \item Escribir de forma binaria el número de filas y el número de columnas.
  \item Escribir de forma binaria los datos del vector interno recorriendo por
filas.
  \item Cerrar el archivo binario.
\end{itemize}

\subsubsection{Lectura de un archivo binario}

Defina una opración \cppid{carga()} que tome como argumento un nombre de un
archivo. La operación realizará las siguientes tareas:

\begin{itemize}
  \item Abrir para lectura en modo binario un archivo con ese nombre.
  \item Leer los datos de una matriz almacenada previamente con la función
\cppid{guarda()}.
  \item Cerrar el archivo.
\end{itemize}

\subsection{Pruebas unitarias}

Como complemento al ejercicio deberá definir todas las pruebas unitarias
necesarias para comprobar el correcto funcionamiento de cada una de las
operaciones.

\subsection{Material a entregar}

Deberá entregar el siguiente material:

\begin{itemize}
  \item \textmark{matriz.hpp}: Archivo de cabecera para la clase \cppid{matriz}.
  \item \textmark{matriz.cpp}: Archivo de implementación para la clase
\cppid{matriz}.
  \item \textmark{matriz\_ut.cpp}: Archivo con pruebas unitarias de la clase
\cppid{matriz}.
  \item \textmark{CMakeLists.txt}: Archivo de configuración \textemph{cmake}.
  \item \textmark{memoria.pdf}: Breve archivo en formato PDF resumiendo las
principales decisiones tomadas, las pruebas realizadas y una vaoloración global
del trabajo y los resultados.
\end{itemize}
