\section{Ejercicios}

\begin{enumerate}

\item Trata de comparar el lenguaje con el que tengas más experiencia con C++ con
respecto a los siguientes aspectos:

\begin{enumerate}

  \item Nivel de abstracción: ¿permite acceder a características de bajo nivel?
              ¿permite construir software complejo con alto nivel de abstracción?

  \item Paradigmas de programación soportados:
    \begin{itemize}
      \item Orientación a objetos: tipos abstractos de datos, herencia simple,
            herencia múltiple, polimorfismo dinámico.
      \item Programación genérica: tipos genéricos, algoritmos genéricos.
      \item Programación funcional: expresiones lambda.
      \item Otros paradigmas.
    \end{itemize}

  \item Gestión de la memoria:
    \begin{itemize}
      \item Problemas de goteo de memoria.
      \item \emph{Buffer overflows}.
      \item Recolección de basura.
    \end{itemize}

  \item Modelo de ejecución: interpretado, basado en máquina virtual,
        ejecución nativa.

  \item Rendimiento.

\end{enumerate}

\item Elabora una lista de herramientas para el lenguaje C++ disponibles en tu
      plataforma preferida.
\begin{enumerate}
  \item Compiladores.
  \item Depuradores.
  \item Analizadores estáticos de código.
  \item Entornos de desarrollo.
  \item Alguna otra categoría que consideres interesante.
\end{enumerate}

\item Elabora una lista de bibliotecas disponibles para C++ en las siguientes
      categorías:
\begin{enumerate}
  \item Gráficos y multimedia.
  \item Redes y comunicaciones.
  \item Concurrencia y paralelismo.
  \item Procesamiento de formatos: XML, JSON, YAML, PDF, \ldots
  \item Interfaz gráfica de usuario.
  \item Criptografía.
  \item Bases de datos.
  \item Cálculo matemático.
  \item Procesamiento de grafos.
  \item Cálculo financiero.
  \item Motores de videjuegos.
  \item Aprendizaje automático (\emph{machine learning}).
  \item Alguna otra categoría de tu interés.
\end{enumerate}

\item Elabora una lista de las 10 aplicaciones de escritorio que utilizas con más 
      frecuencia e investiga en qué lenguaje está escrita cada una de ellas.

\end{enumerate}
