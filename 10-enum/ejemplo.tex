\section{Ejemplo: Meses como enumerados}

\begin{frame}[t]
\begin{block}{fecha.h}
\lstinputlisting[firstline=1,lastline=2]{ejemplos/10-enum/fecha-enum/fecha.hpp}
\lstinputlisting[firstline=5,lastline=16]{ejemplos/10-enum/fecha-enum/fecha.hpp}
\end{block}
\end{frame}

\begin{frame}[t]
\begin{block}{fecha.hpp}
\lstinputlisting[firstline=19]{ejemplos/10-enum/fecha-enum/fecha.hpp}
\end{block}
\end{frame}

\begin{frame}[t]
\begin{block}{main.cpp}
\lstinputlisting{ejemplos/10-enum/fecha-enum/main.cpp}
\end{block}
\end{frame}

\mode<presentation>{

\begin{frame}[t]
\begin{block}{fecha.cpp}
\lstinputlisting[lastline=19]{ejemplos/10-enum/fecha-enum/fecha.cpp}
\end{block}
\end{frame}

\begin{frame}[t]
\begin{block}{fecha.cpp}
\lstinputlisting[firstline=21, lastline=37]{ejemplos/10-enum/fecha-enum/fecha.cpp}
\end{block}
\end{frame}

\begin{frame}[t]
\begin{block}{fecha.cpp}
\lstinputlisting[firstline=39, lastline=45]{ejemplos/10-enum/fecha-enum/fecha.cpp}
\end{block}
\end{frame}

\begin{frame}[t]
\begin{block}{fecha.cpp}
\lstinputlisting[firstline=47, lastline=67]{ejemplos/10-enum/fecha-enum/fecha.cpp}
\end{block}
\end{frame}
}

\mode<article>
{
\begin{frame}[t]
\begin{block}{fecha.cpp}
\lstinputlisting{ejemplos/10-enum/fecha-enum/fecha.cpp}
\end{block}
\end{frame}
}



