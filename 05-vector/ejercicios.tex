\section{Ejercicios}

\begin{enumerate}

\item Escribe un programa que lea una lista de números reales. El programa debe:
\begin{itemize}
  \item Imprimir el valor medio.
  \item Imprimir el valor absoluto de la diferencia entre cada valor y el promedio.
  \item Imprimir la desviación típica.
\end{itemize}

\textmark{Nota}: Para calcular el valor absoluto puedes utilizar la función
\cppid{std::abs()} (consulta
\url{https://en.cppreference.com/w/cpp/numeric/math/abs}).

\item Escribe un programa que lea una lista de posiciones (dadas por un nombre y 
      dos números reales cada una de ellas para las coordenadas $x$ e $y$). 
      El programa debe:
\begin{itemize}
  \item Calcular la distancia desde cada posición a cualquier otra posición.
  \item Determinar las dos posiciones más cercanas.
  \item Determinar las dos posiciones más alejadas.
  \item Determinar la distancia media entre posiciones.
\end{itemize}

\end{enumerate}
