\section{Sobrecarga de operadores}

\begin{frame}[t,fragile]{Operadores}
\begin{itemize}
  \item Los operadores no funcionan con tipos definidos por el usuario.
       
\begin{lstlisting}
enum class dia_semana { 
  lunes, martes, miercoles, jueves, viernes, 
  sabado, domingo
};
dia_semana x { dia_semana::martes };
++x; // Error de compilación
\end{lstlisting}

  \mode<presentation>{\vfill\pause}
  \item Pero se puede definir una función operador:
\begin{lstlisting}
dia_semana operator++(dia_semana & d) {
  d = static_cast<dia_semana>(static_cast<int>(d)+1);
  return d;
}
\end{lstlisting}

  \mode<presentation>{\vfill\pause}
  \item El uso del operador invoca a la función operador
\begin{lstlisting}
++x; // operator++(x);
\end{lstlisting}
\end{itemize}

\end{frame}



\begin{frame}[t,fragile]{Incremento y enumerados}
\begin{itemize}
  \item Puede llegarse a tener valores fuera del enumerado.
\begin{lstlisting}
x = dia_semana::sabado;
++x; // x == dia_semana(6) == domingo
++x; // x == dia_semana(7) == ??
\end{lstlisting}

  \mode<presentation>{\vfill\pause}
  \item Se puede controlar en la función operador
\begin{lstlisting}
dia_semana operator++(dia_semana & d) {
  d = (d==dia_semana::domingo)?
    dia_semana::lunes :
    static_cast<dia_semana>(static_cast<int>(d)+1);
  return d;
}
\end{lstlisting}
\end{itemize}
\end{frame}

\begin{frame}[t,fragile]{Postincremento}
\begin{itemize}
  \item La función \cppkey{operator}\cppid{++} define operador de 
        preincremento.
\begin{lstlisting}
z = ++x; // z = operator++(x)
z = x++; // ??
\end{lstlisting}
  
  \mode<presentation>{\vfill\pause}
  \item La función de post-incremento toma un segundo parámetro
        no usado de tipo \cppid{int}.
\begin{lstlisting}
dia_semana operator++(dia_semana & d) {
  auto anterior = d; // Copiar el valor anterior
  ++d; // Incrementa d
  return anterior;
}
\end{lstlisting}

  \mode<presentation>{\vfill\pause}
  \item El uso del postincremento invoca a la función operador.
\begin{lstlisting}
z = ++x; // z = operator++(x)
z = x++; // z = operator++(x,int)
\end{lstlisting}
\end{itemize}
\end{frame}

\begin{frame}[fragile]{Operadores de salida}
\begin{itemize}
  \item Permite definir la salida a un flujo de un dato de un tipo definido por el usuario.
  \item Hacen uso de tipo de biblioteca \cppid{ostream} (flujo de salida).
    \begin{itemize}
      \item \cppid{cout} es un \cppid{ostream}.
    \end{itemize}
\begin{lstlisting}
ostream & operator<<(ostream & os, color c) {
  switch (c) {
    case color::rojo:
      return cout << "rojo";
    case color::verde:
      return cout << "verde";
    case color::azul:
      return cout << "azul";
  }
  CONTRACT_ASSERT(false);
}
\end{lstlisting}
  \item Ejemplo de uso:
\begin{lstlisting}
  std::cout << color::verde; // operator<<(std::cout,color::verde)
\end{lstlisting}
\end{itemize}
\end{frame}

\section{Ejemplo: Operadores en \texttt{fecha}}

\begin{frame}
\begin{block}{fecha.h}
\lstinputlisting[lastline=4]{ejemplos/11-operadores/fecha-oper/fecha.hpp}
\lstinputlisting[firstline=6,lastline=18]{ejemplos/11-operadores/fecha-oper/fecha.hpp}
\end{block}
\end{frame}

\begin{frame}[t,fragile]{Operador de incremento}
\begin{block}{fecha.cpp}
\lstinputlisting[firstline=9,lastline=18]{ejemplos/11-operadores/fecha-oper/fecha.cpp}
\end{block}
\end{frame}

\begin{frame}[t,fragile]{Operador de salida}
\begin{block}{fecha.cpp (mes)}
\lstinputlisting[firstline=20,lastline=26]{ejemplos/11-operadores/fecha-oper/fecha.cpp}
\end{block}
\mode<presentation>{\vfill\pause}
\begin{block}{fecha.cpp (fecha)}
\lstinputlisting[firstline=55,lastline=58]{ejemplos/11-operadores/fecha-oper/fecha.cpp}
\end{block}
\end{frame}
