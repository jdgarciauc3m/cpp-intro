\section{Sobrecarga de operadores}


\begin{frame}[fragile]{Operadores}
\begin{itemize}
  \item Permite definir el comportamiento de un operador cuando
        se aplica a un tipo definido por el usuario.
\begin{lstlisting}
enum class dia_semana { 
  lunes, martes, miercoles, jueves, viernes, 
  sabado, domingo
};
dia_semana d { dia_semana::martes };
++d; // miércoles
++d; // jueves
d = dia_semana::domingo;
++d; // lunes
\end{lstlisting}
  \item Se puede definir una función operador
\begin{lstlisting}
dia_semana operator++(dia_semana & d) {
  d = (d==dia_semana::domingo)?
    lunes :
    static_cast<dia_semana>(static_cast<int>(d)+1);
  return d;
}
\end{lstlisting}
\end{itemize}
\end{frame}

\begin{frame}[fragile]{Operadores de salida}
\begin{itemize}
  \item Permite definir la salida a un flujo de un dato de un tipo definido por el usuario.
  \item Hacen uso de tipo de biblioteca \cppid{ostream} (flujo de salida).
    \begin{itemize}
      \item \cppid{cout} es un \cppid{ostream}.
    \end{itemize}
\begin{lstlisting}
ostream & operator<<(ostream & os, color c) {
  switch (c) {
    case color::rojo:
      return cout << "rojo";
    case color::verde:
      return cout << "verde";
    case color::azul:
      return cout << "azul";
  }
  throw logic_error("Dia no esperado");
}
\end{lstlisting}
  \item Ejemplo de uso:
\begin{lstlisting}
  cout << color::verde << endl;
\end{lstlisting}
\end{itemize}
\end{frame}

\section{Ejemplo: Operadores en \texttt{fecha}}

\begin{frame}
\begin{block}{fecha.h}
\lstinputlisting[lastline=17]{11-operadores/fecha6/fecha.h}
\end{block}
\end{frame}

\begin{frame}
\begin{block}{fecha.cpp}
\lstinputlisting[firstline=7,lastline=21]{11-operadores/fecha6/fecha.cpp}
\end{block}
\end{frame}
