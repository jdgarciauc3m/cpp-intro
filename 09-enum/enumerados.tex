\section{Enumerados}

\begin{frame}[fragile]{Enumerados}
\begin{itemize}
  \item Un tipo enumerado es un tipo definido por el usuario que puede
        representar un conjunto finito de valores enteros.
  \item Dos familias de tipos enumerados:
    \begin{itemize}
      \item \cppkey{enum class}: Enumerados fuertemente tipados y con alcance.
\begin{lstlisting}
enum class color { rojo, verde, azul};
\end{lstlisting}
      \item \cppkey{enum}: Enumerados heredados de C y C++98.
\begin{lstlisting}
enum nivel { bajo, medio, alto, extremo };
\end{lstlisting}
    \end{itemize}
  \item Cada valor enumerado se representa internamete mediante un valor entero.
\end{itemize}
\end{frame}

\begin{frame}[fragile]{Alcance y tipo}
\begin{itemize}
  \item Un enumerado define un tipo con un conjunto de valores que puede tomar.
\begin{lstlisting}
enum class color { rojo, verde, azul };
\end{lstlisting}
  \item El tipo define un alcance con el nombre del tipo.
\begin{lstlisting}
color c = rojo; // Error: rojo no está en alcance
color c = color::rojo; // OK
\end{lstlisting}
  \item No hay conversiones implícitas a entero
\begin{lstlisting}
color c = 2; // Error: int->color
color d = static_cast<color>(2); // OK
int x = color::rojo; // Error: color->int
int y = static_cast<int>(color::rojo); // OK
\end{lstlisting}
\end{itemize}
\end{frame}

\begin{frame}[fragile]{Representación interna}
\begin{itemize}
  \item Un enumerado se representa internamente con algún tipo entero.
    \begin{itemize}
      \item Cada valor enumerado se corresponde con un valor entero.
      \item El tipo subyacente debe ser un tipo entero con o sin signo.
      \item Por defecto, el tipo subyacente es \cppkey{int}.
    \end{itemize}
  \item Se puede especificar explícitamente el tipo subyacente.
\begin{lstlisting}
enum class color : int { rojo, verde, azul };
enum class alerta : char { amarilla, naranja, roja };
int n1 = sizeof(color); // sizeof(int)
int n2 = sizeof(alerta); // 1
\end{lstlisting}
\end{itemize}
\end{frame}

\begin{frame}[fragile]{Valores de enumeradores}
\begin{itemize}
  \item Los enumeradores toman valores consecutivos empezando desde 0;
\begin{lstlisting}
enum class color { 
  rojo, // 0
  verde, // 1
  azul // 2
};
\end{lstlisting}
  \item Se puede especificar el valor concreto de un enumerador
\begin{lstlisting}
enum class estado { correcto = 0, ocupado = 1, esperando = 2, 
    error = 4, apagado = 8 };

enum class color { 
  rojo=1, // 1
  verde, // 2
  azul // 3
};
\end{lstlisting}
\end{itemize}
\end{frame}


\begin{frame}[fragile]{Enumerados simples}
\begin{itemize}
  \item Los enumerados simples permancen por compatibilidad hacia atrás.
\begin{lstlisting}
enum color { rojo, verde, azul };
enum semaforo { naranja, verde, rojo };
\end{lstlisting}
  \item Los enumeradores se inyectan en el alcance externo.
\begin{lstlisting}
color x = azul; // OK
color y = color::rojo; // OK
color z = verde; // ¿Qué verde?
\end{lstlisting}
  \item Hay conversiones implícitas a enteros
\begin{lstlisting}
color x = 2; // Error: int -> color
int n = naranja; // semaforo -> int
if (x == 4) { /*...*/ } // OK. ¿Qué color es ese?
if (x == semaforo::rojo) { /* ... */ } // OK. Cuidado
\end{lstlisting}
\end{itemize}
\end{frame}

\begin{frame}[fragile]{Representación interna y conversiones}
\begin{itemize}
  \item El tipo subyacente es un tipo entero.
    \begin{itemize}
      \item Se puede especificar el tipo subyacente.
\begin{lstlisting}
enum color : int { rojo, verde, azul };
\end{lstlisting}
      \item Si no se especifica hay un algoritmo que lo determina.
        \begin{itemize}
          \item El número de bits más pequeño capaz de contener los valores usando
                complemento a dos.
        \end{itemize}
    \end{itemize}
  \item Las conversiones no están definidas si el valor está fuera del rango de representación
\begin{lstlisting}
enum estado { ok = 0, ocupado = 1, esperando = 2, error = 4, apagado = 8};
estado e1 = static_cast<estado>(5); // OK
estado e2 = static_cast<estado>(22); // No definido. Rango 0:15
\end{lstlisting}
\end{itemize}
\end{frame}

\section{Ejemplo: Meses como enumerados}

\begin{frame}
\begin{block}{fecha.h}
\lstinputlisting[lastline=12]{09-enum/fecha5/fecha.h}
\end{block}
\end{frame}

\begin{frame}
\begin{block}{fecha.h}
\lstinputlisting[firstline=14]{09-enum/fecha5/fecha.h}
\end{block}
\end{frame}

\begin{frame}
\begin{block}{main.cpp}
\lstinputlisting{09-enum/fecha5/main.cpp}
\end{block}
\end{frame}

\begin{frame}
\begin{block}{fecha.cpp}
\lstinputlisting[lastline=20]{09-enum/fecha5/fecha.cpp}
\end{block}
\end{frame}

\begin{frame}
\begin{block}{fecha.cpp}
\lstinputlisting[firstline=22, lastline=39]{09-enum/fecha5/fecha.cpp}
\end{block}
\end{frame}

\begin{frame}
\begin{block}{fecha.cpp}
\lstinputlisting[firstline=41, lastline=48]{09-enum/fecha5/fecha.cpp}
\end{block}
\end{frame}

\begin{frame}
\begin{block}{fecha.cpp}
\lstinputlisting[firstline=50]{09-enum/fecha5/fecha.cpp}
\end{block}
\end{frame}




