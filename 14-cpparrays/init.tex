\section{Iniciación de arrays}

\begin{frame}[t,fragile]{Iniciadores}
\mode<presentation>{\vspace{-0.5em}}
\begin{itemize}
  \item Un array puede \textgood{iniciarse} con una \textmark{lista de valores}.
\begin{lstlisting}
std::array<double,4> v { 1.0, 2.0, 3.5, 4.7 };
std::array<char,4> c { 'c', '+', '+', 0 };
\end{lstlisting}

  \mode<presentation>{\vfill\pause}
  \item No se puede \textbad{omitir el tamaño} aunque haya un valor inicial.
\begin{lstlisting}
std::array<double> v {1.2, 2.4, 3.6 }; // Error
\end{lstlisting}

  \mode<presentation>{\vfill\pause}
  \item Pero se puede \textgood{omitir tipo y tamaño} cuando hay valor inicial
\begin{lstlisting}
std::array v {1.2, 2.4, 3.6 }; // std::array<double,3>
\end{lstlisting}

  \mode<presentation>{\vfill\pause}
  \item Si \textbad{faltan iniciadores}, \textgood{se completa} con \cppid{0}.
\begin{lstlisting}
std::array<double,4> v {1.5, 2.5}; // v[2]==0.0, v[3]==0.0
\end{lstlisting}

  \mode<presentation>{\vfill\pause}
  \item No se pueden especificar \textbad{más iniciadores de los necesarios}.
\begin{lstlisting}
std::array<char,3> c { 'H', 'o', 'l', 'a' };
\end{lstlisting}

  \mode<presentation>{\vfill\pause}
  \item Iniciación a \cppid{0}.
\begin{lstlisting}
std::array<double,10> v {};
\end{lstlisting}
\end{itemize}
\end{frame}

\begin{frame}[t,fragile]{Consulta del tamaño de un array}
\begin{itemize}
  \item Se puede consultar el tamaño de un \cppid{std::array} con la 
        operación \cppid{size()}.
    \begin{itemize}
      \item El resultado es del tipo entero sin signo \cppid{std::size\_t}.
    \end{itemize}
\begin{lstlisting}
std::array v {1, 2, 3, 4, 5};
std::size_t s = v.size();
auto n = v.size(); // n es de tipo std::size_t
\end{lstlisting}

  \mode<presentation>{\vfill\pause}
  \item También se puede usar en versión de función libre.
\begin{lstlisting}
std::array v {1, 2, 3, 4, 5};
std::size_t s = std::size(v);
\end{lstlisting}

  \mode<presentation>{\vfill\pause}
  \item Se puede obtener el tamaño como un valor con signo de tipo \cppid{ssize\_t}.
\begin{lstlisting}
std::array v {1, 2, 3, 4, 5};
auto n = std::ssize(v); // n es de tipo std::ssize_t
\end{lstlisting}
\end{itemize}
\end{frame}

\begin{frame}[t,fragile]{Iniciación de array como dato miembro}
\begin{itemize}
  \item Un dato miembro que sea de tipo \cppid{std::array} \textgood{se puede iniciar} 
        con una \textmark{lista de valores}.
\begin{lstlisting}
class conversor {
private:
  std::array<double,4> coef;
public:
  conversor(double x, double y, double z, double t) : coef{x,y,z,t} {}
  double convierte(double a) {
    return coef[0] * a + coef[1] * a + coef[2] * a + coef[3] * a;
  }
};

void f() {
  conversor c{1.0, 0.0, -1, 2};
  cout << c.convierte(3) << "\n";
}
\end{lstlisting}
\end{itemize}
\end{frame}
