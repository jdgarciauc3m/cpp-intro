\section{Copia de arrays}

\begin{frame}[t,fragile]{Copia de arrays}
\begin{itemize}
  \item Se pueden copiar variables \cppid{std::array} con el operador \cppkey{=}.
\begin{lstlisting}
std::array v {1, 2, 3, 4};
std::array w {5, 6, 7, 8}
v = w;
\end{lstlisting}

  \mode<presentation>{\vfill\pause}
  \item Solamente se puede realizar la copia si tienen el mismo tipo de elemento
        y el mismo tamaño.
\begin{lstlisting}
std::array z {1.0, 2.0, 3.0};
std::array t {1, 2, 3};
z = t; // Error distinto tipo de elemento
v = t; // Error distinto tamaño
\end{lstlisting}
\end{itemize}
\end{frame}

\begin{frame}[t,fragile]{Alternativas de copia}
\begin{itemize}
  \item \textmark{Copia explícita} elemento a elemento.
\begin{lstlisting}
std::array v {1, 2, 3, 4}; // std::array<int,4>
std::array w {2, 4, 6 , 8, 10, 12, 14}; // std::array<int,7>
for (int i=0; i<4; ++i) {
  w[i] = v[i];
}
\end{lstlisting}

  \mode<presentation>{\vfill\pause}
  \item \textmark{Algoritmo de copia} con límites.
\begin{lstlisting}
std::array v {1, 2, 3, 4}; // std::array<int,4>
std::array w {2, 4, 6 , 8, 10, 12, 14}; // std::array<int,7>
std::copy(v.begin(), v.end(), w.begin());
\end{lstlisting}

  \mode<presentation>{\vfill\pause}
  \item \textmark{Algoritmo de copia} con tamaño.
\begin{lstlisting}
std::array v {1, 2, 3, 4}; // std::array<int,4>
std::array w {2, 4, 6 , 8, 10, 12, 14}; // std::array<int,7>
std::copy(v.begin(), 4, w.begin());
\end{lstlisting}
\end{itemize}
\end{frame}
