\section{Ejercicios}

\begin{enumerate}

\item
Define un tipo de datos \cppid{punto} que representa un punto del plano dado por
sus coordenadas en doble precisión \cppid{x}, \cppid{y} y \cppid{z}. La
representación interna debe venir dada por una única variable miembro
\cppid{vec\_} usando un \cppid{std::array}.

Debe ofrecer las siguientes operaciones:

\begin{itemize}

\item Constructor por defecto:
\begin{lstlisting}
punto p; // Construye en coordenadas 0.0, 0.0, 0.0
\end{lstlisting}

\item Constructor a partir de coordenadas:
\begin{lstlisting}
punto p{1.0, 5.5, 7.2}; // Construye en coordenadas 1.0, 5.5, 7.2
\end{lstlisting}

\item Funciones miembro para obtener las coordenadas:
\begin{lstlisting}
double cx = p.x();
auto cy = p.y();
auto cz = p.z();
\end{lstlisting}

\item Funciones miembro para modificar las coordenadas:
\begin{lstlisting}
p.x(2.5);
p.y(3.9);
p.z(4.2);
\end{lstlisting}

\item Copia de puntos:
\begin{lstlisting}
punto p{1.0, 2.0, 3.0};
punto q{p};
punto r;
r = q;
\end{lstlisting}

\item Distancia entre puntos:
\begin{lstlisting}
double d = distancia(p,q);
\end{lstlisting}

\end{itemize}
\end{enumerate}
