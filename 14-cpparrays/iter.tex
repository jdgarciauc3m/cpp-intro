\section{Arrays y aritmética de iterador}

\begin{frame}[t,fragile]{Iteradores a posiciones de un array}
\begin{itemize}
  \item Un \textgood{iterador} permite \textmark{acceder directamente} 
        a una posición de un 
        \cppid{std::array}
    \begin{itemize}
      \item La operación \cppid{begin()} devuelve un iterador a la 
            \textmark{primera posición} del \cppid{std::array}.
\begin{lstlisting}
std::array<double,4> v;
auto i = v.begin();
\end{lstlisting}

      \mode<presentation>{\vfill\pause}
      \item El operador \cppkey{*} permite acceder al 
            elemento \textmark{apuntado} por el \textgood{iterador}.
\begin{lstlisting}
*i = 1.5;
double x = v[0]; // x == 1.5
\end{lstlisting}
    \end{itemize}

  \mode<presentation>{\vfill\pause}
  \item Se puede \textgood{iniciar} un iterador a \textmark{cualquier posición} del 
        \cppid{std::array}.
\begin{lstlisting}
auto p = v.begin() + 1; // p apunta a v[1]
p = v.begin() + 3; // p apunta a v[3]
\end{lstlisting}
\end{itemize}
\end{frame}

\begin{frame}[t,fragile]{Límites de un array}
\begin{itemize}
  \item También se puede apuntar a la \textmark{posición siguiente a la última}
        con \cppid{end()}.
    \begin{itemize}
      \item Pero no a posiciones anteriores o posteriores.
    \end{itemize}
\begin{lstlisting}
auto p = v.end(); // OK. Posición siguiente a la última
p = v.begin() + v.size(); // OK. Posición siguiente a la última.
p = v.begin() - 1; // Comportamiento no definido
p = v.end() + 1; // Comportamiento no definido
\end{lstlisting}
\end{itemize}
\end{frame}

\begin{frame}[t,fragile]{Aritmética de iteradores}
\begin{itemize}
  \item Se puede \textgood{avanzar} con un iterador en un \cppid{std::array}, 
        \textmark{sumando un entero} al puntero.
\begin{lstlisting}
std::array<double,4> v {};
auto p = v.begin() + 1; // Apunta a v[1]
p = p + 2; // Apunta a v[3];
++p; // Apunta a v[4]
\end{lstlisting}

  \mode<presentation>{\vfill\pause}
  \item Igualmente se puede \textgood{retroceder};
\begin{lstlisting}
--p; // Apunta a v[3]
p = p - 2; // Apunta a v[1]
p -=1; // Apunta a v[0]
\end{lstlisting}
\end{itemize}
\end{frame}

\begin{frame}[t,fragile]{Recorrido con iteradores}
\begin{itemize}
  \item Se puede utilizar un \textgood{iterador} para \textmark{recorrer}
        un \cppid{std::array}.
\begin{lstlisting}
std::array v {1, 2, 3, 4, 5, 6, 7, 8, 9};
for (auto p=v.begin(); p!=v.end(); ++p) {
  std::cout << *p << '\n';
}
\end{lstlisting}

  \mode<presentation>{\vfill\pause}
  \item Se puede utilizar también un \textgood{iterador inverso} para 
        \textmark{recorrer de fin a principio}.
\begin{lstlisting}
std::array v {1, 2, 3, 4, 5, 6, 7, 8, 9};
for (auto p=v.rbegin(); p!=v.rend(); ++p) {
  std::cout << *p << '\n';
}
\end{lstlisting}
\end{itemize}
\end{frame}

\begin{frame}[t,fragile]{Distancia entre dos posiciones}
\begin{itemize}
  \item Se puede \textgood{calcular la distancia} entre dos iteradores 
        como número de elementos.
    \begin{itemize}
      \item Solamente si apuntan a objetos del mismo array.
    \end{itemize}
\begin{lstlisting}
std::array<double> v {}, w {};
auto p = v.begin();
auto q = v.end();
auto n = q - p; // n == 4
n = std::distance(p,q); // n == 4
auto m = std::distance(v.begin(), w.end()); // No definido
\end{lstlisting}
\end{itemize}
\end{frame}

\begin{frame}[t,fragile]{Paso como parámetro}
\begin{itemize}
  \item Se puede pasar un array como argumento a una función.

  \begin{columns}[T]

  \column{.5\textwidth}
\begin{block}{Paso por valor}
\begin{lstlisting}
double f(std::array<double,5> vec) {
  double r = 0.0;
  for (auto x : vec) {
    r += x;
  }
  return r;
}
\end{lstlisting}
\end{block}

  \mode<presentation>{\pause}
  \column{.55\textwidth}
\begin{block}{Paso por referencia}
\begin{lstlisting}
double f(const std::array<double,5> & vec) {
  double r = 0.0;
  for (auto & x : v) {
    r += x;
  }
  return r;
}
\end{lstlisting}
\end{block}
  \end{columns}

  \mode<presentation>{\vfill\pause}
  \item Es necesario fijar el tamaño como parte del tipo del parámetro.
    \begin{itemize}
      \item \textgood{Alternativa}: Usar \cppid{std::vector} o 
            \textemph{programación genérica}.
    \end{itemize}

\end{itemize}
\end{frame}
