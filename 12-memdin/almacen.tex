\section{El almacén libre}

\subsection{Asignación de memoria}

\begin{frame}{Memoria del almacén libre}
\begin{itemize}
  \item El \alert{almacén libre} contiene la memoria libre que un programa
        puede adquirir y liberar.
    \begin{itemize}
      \item Típicamente implementado mediante el \emph{montículo}.
    \end{itemize}
  \item \alert{IMPORTANTE}: C++ no es un lenguaje con gestión automática de recursos.
    \begin{itemize}
      \item Si se adquiere un recurso, éste se debe liberar.
      \item Si se adquire memoria del almacén libre, ésta se debe devolver.
    \end{itemize}
\end{itemize}
\begin{quote}
C++ Is my favorite garbage collected language because it generates so little garbage.

\textbf{Bjarne Stroustrup}
\end{quote}
\end{frame}

\begin{frame}[fragile]{Operador de asignación de memoria}
\begin{itemize}
  \item El operador \cppkey{new} permite asignar memoria del almacén libre.
\begin{lstlisting}
int * p = new int; // Asigna memoria para un int
char * q = new char[10]; // Asigna memoria para 10 char
\end{lstlisting}
  \item \emph{Efecto}:
    \begin{itemize}
      \item El operador \cppkey{new} devuelve un puntero al inicio de la memoria asignada.
      \item Una expresión \cppkey{new} \cppid{T} devuelve un valor de tipo \cppid{T*}.
    \end{itemize}
  \item No se puede saber el tamaño de la memoria asignada después de evaluar la expresión \cppkey{new}.
\end{itemize}
\end{frame}

\subsection{Acceso e memoria mediante punteros}

\begin{frame}[fragile]{Operadores de acceso}
\begin{itemize}
  \item Dos operadores de acceso:
    \begin{itemize}
      \item \alert{Relativo}: Operador \cppkey{[]} (indexación).
        \begin{itemize}
          \item Accede de forma indexada en una secuencia de valores.
\begin{lstlisting}
int * v = new int[10];
v[0] = 42;
v[1] = 84;
\end{lstlisting}
        \end{itemize}

      \item \alert{Indirecto}: Operador \cppkey{*} (desreferencia).
        \begin{itemize}
          \item Si se apunta a un valor, accede al valor apuntado.
\begin{lstlisting}
int * p = new int;
*p = 42;
int x = *p; // x = 42
\end{lstlisting}

          \item Si se apunta a una secuencia de valores, accede al primer valor de la secuencia.
\begin{lstlisting}
int * v = new int[10];
*v = 42; // Equivale a v[0] = 42;
\end{lstlisting}
        \end{itemize}
    \end{itemize}
\end{itemize}
\end{frame}

\begin{frame}[fragile]{Problemas de acceso}
\begin{itemize}
  \item Una variable de tipo puntero no se inicia de forma automática a ningún valor.
    \begin{itemize}
      \item Si se desreferencia un puntero no iniciado se tiene un comportamiento no definido.
    \end{itemize}
\begin{lstlisting}
int * p;
*p = 42; // Comportamiento no definido.
p[0] = 42; // Comportamiento no definido.
\end{lstlisting}
  \item Una variable de tipo puntero iniciada a una secuencia solamente puede accederse dentro
        de sus límites establecidos.
\begin{lstlisting}
int * v = new int[10];
v[0] = 42; // OK
x = v[-1]; // No definido
x = v[15]; // No definido
v[10] = 0; // No definido
\end{lstlisting}
\end{itemize}
\end{frame}

\begin{frame}[fragile]{El puntero nulo}
\begin{itemize}
  \item Se puede iniciar un puntero al valor \emph{puntero-nulo} para indicar que no apunta a ningún objeto.
    \begin{itemize}
      \item Literal \cppkey{nullptr}.
    \end{itemize}
\begin{lstlisting}
int * p = nullptr;
char * q = nullptr;
if (p != nullptr) { /* ... */ }
if (q == nullptr) { /* ... */ }
\end{lstlisting}
  \item Otras alternativas heredades de C y C++98:
    \begin{itemize}
      \item Literal \cppid{0}.
      \item Macro predefinida \cppid{NULL}.
    \end{itemize}
  \item Un puntero es contextualmente convertible a booleano.
    \begin{itemize}
      \item Si aparece en el contexto en que se espera un booleano.
    \end{itemize}
\begin{lstlisting}
if (p) { /* ... */ } // if (p != nullptr) 
if (!q) { /* ... */ } // if (q == nullptr)
\end{lstlisting}
\end{itemize}
\end{frame}

\begin{frame}[fragile]{Asignación de memoria e iniciación}
\vspace{-0.5em}
\begin{itemize}
  \item El operador \cppkey{new} no inicia el objeto asignado.
\begin{lstlisting}
int * p = new int;
x = *p; // x tiene un valor desconocido
\end{lstlisting}
    \begin{itemize}
      \item Se puede indicar el valor inicial entre llaves.
    \end{itemize}
\begin{lstlisting}
p = new int{42}; // *p == 42
p = new int{}; // *p == 0
\end{lstlisting}
  \item Si se reserva una secuencia con \cppkey{new} no se inicia ninguno de los objetos.
\begin{lstlisting}
int * v = new int[10];
\end{lstlisting}
    \begin{itemize}
      \item Se puede indicar los valores iniciales entre llaves.
    \end{itemize}
\end{itemize}
\begin{lstlisting}
v = new int[4]{1,2,3,4}; // v[0] = 1, v[1] = 2, v[2] = 3, v[3] = 4
v = new int[4]{1, 2};    // v[0] = 1, v[1] = 2, v[2] = 0, v[3] = 0
v = new int[4]{};        // v[0] = 0, v[1] = 0, v[2] = 0, v[3] = 0
v = new int[4]{1,2,3,4,5}; // Error demasiados iniciadores
\end{lstlisting}
\end{frame}

\subsection{Desasignación de memoria}

\begin{frame}[fragile]{Operador de desasignación de memoria}
\begin{itemize}
  \item El operador \cppkey{delete} permite liberar memoria y marcarla como no asignada.
  \item Pude aplicarse solamente a:
    \begin{itemize}
      \item Memoria devuelta por el operador \cppkey{new} y actualmente asignada.
      \item El punter nulo.
    \end{itemize}
\begin{lstlisting}
int * p = new int{10};
*p = 20;
delete p; // Libera p
\end{lstlisting}
  \item Es un error invocar dos veces a \cppkey{delete} sobre un mismo puntero.
\begin{lstlisting}
int * p = new int{10};
delete p; // Libera p
delete p; // Comportamiento no definido
\end{lstlisting}
\end{itemize}
\end{frame}

\begin{frame}[fragile]{Desasignación de arrays}
\begin{itemize}
  \item Existe una versión diferente para liberar \emph{arrays}.
\begin{lstlisting}
int * p = new int{10};
int * v = new int[10];
delete p; // Libera p
delete []v;
\end{lstlisting}
  \item \alert{Importante}: Se debe usar la versión correcta de desasignación.
    \begin{itemize}
      \item Si se reserva memoria con \cppkey{new T} debe liberarse con \cppkey{delete}.
      \item Si se reserva memoria con \cppkey{new T[n]} debe liberarse con \cppkey{delete[]}.
    \end{itemize}
\begin{lstlisting}
int * p = new int{10};
int * v = new int[10];
delete []p; // Comportamiento no definido
delete v; // Comportamiento no definido
\end{lstlisting}
\end{itemize}
\end{frame}

\begin{frame}[fragile]{Razones para desasignar}
\begin{itemize}
  \item Si se reserva memoria y no se libera esta queda asignada.
\begin{lstlisting}
void f() {
  int * v = new int[1024*1024];
  // ...
}
\end{lstlisting}
    \begin{itemize}
      \item Cada vez que se invoca a \cppid{f()} se pierden 8 MB (si \cppkey{sizeof(int)}\cppid{==8}).
    \end{itemize}
  \item Problemas con los goteos de memoria:
    \begin{itemize}
      \item En cada asignación de memoria puede requerirse más tiempo.
      \item Si el programa se ejecuta durante mucho tiempo, puede acabar agotándose la memoria.
    \end{itemize}
  \item Se se agota la memoria se lanza la excepción \cppid{bad\_alloc}.
\end{itemize}
\end{frame}

