\section{Tipos definidos por el usuario}

\begin{frame}[t]{Tipos}
\begin{itemize}
  \item Cada objeto manipulado tiene un tipo asociado.

  \mode<presentation>{\vfill\pause}
  \item Un \textgood{tipo} tiene asociada:
    \begin{itemize}
      \item Una \textmark{representación} de los datos necesarios 
            en la memoria del computador.
      \item Un conjunto de \textmark{operaciones} 
            que pueden realizarse sobre un objeto.
    \end{itemize}

  \mode<presentation>{\vfill\pause}
  \item Los tipos pueden ser:
    \begin{itemize}
      \item \textmark{Primitivos}: \cppkey{int}, \cppkey{char}, \ldots
      \item \textmark{Definidos por la biblioteca}: \cppid{string}, \cppid{vector}, \ldots
      \item \textmark{Definidos por el usuario}: \cppid{punto}, \cppid{fecha}, \ldots
    \end{itemize}
\end{itemize}
\end{frame}

\begin{frame}[t,fragile]{Definición de tipos}
\begin{itemize}
  \item Dos mecanismos básicos de \textgood{definición de nuevos tipos} en C++:
    \begin{itemize}

      \mode<presentation>{\vfill\pause}
      \item \textmark{Clase}: 
            Define un tipo mediante su representación y conjunto de operaciones.
\begin{lstlisting}
class punto {
  punto(double cx, double cy);
  // ...
private:
  double x, y;
};
\end{lstlisting}

      \mode<presentation>{\vfill\pause}
      \item \textmark{Enumerados}: 
            Define un tipo como un conjunto de valores admisibles.
\begin{lstlisting}
enum class color { rojo, verde, azul };
\end{lstlisting}
    \end{itemize}
\end{itemize}
\end{frame}
