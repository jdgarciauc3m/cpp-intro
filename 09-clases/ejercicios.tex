\section{Ejercicios}

\begin{enumerate}

\item Escribe una clase \cppid{racional} que representa un número faccionario como
      una fracción con un numerador y un denominador enteros. El tipo de datos
      debe soportar, al menos, las siguientes operaciones:

\begin{itemize}

  \item Construcción a partir del numerador y el denominador.
  \item Suma y resta de números racionales.
  \item Producto de números racionales.
  \item División de números racionales.

\end{itemize}

\textmark{Nota}: En todas las operaciones se ofrecerá como resultado la fracción
                 más simplificada posible (p.ej $\frac{1}{2}$ en vez $\frac{4}{8}$).

\item Escribe una clase \cppid{matriz} que represente una matriz numérica bidimencional.
      Utiliza para representar los valores números reales en doble precisión.
      Puedes utilizar internamente un \cppid{std::vector} del tamaño adecuado
      para representar los valores, de manera que cada fila se almacena dentro
      del vector a continuación de la siguiente (primero la fila 0, depués
      la fila 1, \ldots).

      De esta forma para determinar la posición que ocupa el elemento $m_{i,j}$
      en una matriz $A$ de $m$ filas y $n$ columnas, se realizará el siguiente
      cálculo:
\[
pos = i \times m + j
\]


      El tipo de datos debe soportar las siguientes operaciones:

\begin{itemize}
  \item Constructor a partir de un número de filas y columnas.
  \item Obtención del número de filas y número de columnas.
  \item Asignación de un valor a una posición $(i,j)$.
  \item Consulta de un valor que está en una posición.
  \item Suma de dos matrices que deben tener el mismo número de filas y de columnas.
  \item Producto de dos matrices.
\end{itemize}

      Escribe un conjunto de pruebas unitarias para este tipo de datos.

\end{enumerate}
