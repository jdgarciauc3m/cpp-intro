\section{Clases}

\begin{frame}[t,fragile]{¿Qué es una clase?}
\begin{itemize}
  \item Tipo \textemph{definido por el usuario}.
    \begin{itemize}
      \item \textmark{Representación} $\rightarrow$ \textgood{Datos miembro}.
      \item \textmark{Operaciones} $\rightarrow$ \textgood{Funciones miembro}
    \end{itemize}

  \mode<presentation>{\vfill\pause}
  \item Construcciones del lenguaje:

\begin{columns}[T]

\column{.1\textwidth}
\column{.3\textwidth}
    \begin{itemize}
      \item Estructura.
\begin{lstlisting}
struct fecha {
  // Miembros
};
\end{lstlisting}
    \end{itemize}

\column{.3\textwidth}
    \begin{itemize}
      \item Clase
\begin{lstlisting}
class fecha {
  // Miembros
};
\end{lstlisting}
    \end{itemize}

\column{.3\textwidth}

\end{columns}


  \mode<presentation>{\vfill\pause}
  \item \textmark{Alternativas}:
    \begin{itemize}
      \item Estructura + Funciones sobre estructura.
      \item Estructura con funciones miembro.
      \item Clase.    
    \end{itemize}
\end{itemize}
\end{frame}


\begin{frame}{Estructura y funciones sobre estructura}
\mode<presentation>{\vspace{-.8em}}
\begin{block}{fecha.h}
\lstinputlisting{09-clases/fecha1/fecha.h}
\end{block}
\begin{block}{main.cpp}
\lstinputlisting[firstline=6]{09-clases/fecha1/main.cpp}
\end{block}
\end{frame}

\mode<presentation>{\vspace{-.25em}}
\begin{frame}
\begin{block}{fecha.cpp}
\mode<presentation>{
\lstinputlisting[lastline=18]{09-clases/fecha1/fecha.cpp}
}
\mode<article>{
\lstinputlisting{09-clases/fecha1/fecha.cpp}
}
\end{block}
\end{frame}

\begin{frame}[fragile]{Función miembro}
\begin{itemize}
  \item Es una \textgood{función} que puede invocarse 
        \textbad{solamente} para una \textgood{variable} del tipo definido.
    \begin{itemize}
      \item Recibe implícitamente un parámetro del tipo definido.
      \item Puede acceder a los datos miembro de la variable.
      \item Se invoca usando la notación de acceso.
        \begin{itemize}
          \item \cppid{variable.funcion(param)}.
        \end{itemize}
    \end{itemize}
\begin{lstlisting}
struct punto {
  double x, y;
  double modulo();
};

void f() {
  punto p { 2.0, 3.5 };
  double x = p.modulo(); // Invocación
  // ...
}
\end{lstlisting}
\end{itemize}
\end{frame}

\begin{frame}{Estructura con fucniones miembro}
\begin{block}{fecha.h}
\lstinputlisting{09-clases/fecha2/fecha.h}
\end{block}
\begin{block}{main.cpp}
\lstinputlisting[firstline=6]{09-clases/fecha2/main.cpp}
\end{block}
\end{frame}

\begin{frame}
\begin{block}{fecha.cpp}
\mode<presentation>{
\lstinputlisting[lastline=18]{09-clases/fecha2/fecha.cpp}
}
\mode<article>{
\lstinputlisting{09-clases/fecha2/fecha.cpp}
}
\end{block}
\end{frame}

\begin{frame}[t,fragile]{Visibilidad}
\mode<presentation>{\vspace{-0.25cm}}
\begin{itemize}
  \item Los miembros de una estructura (o una clase) tienen asociado 
        un \textmark{nivel de visibilidad}.
    \begin{itemize}
      \item \cppkey{public}: El miembro puede ser accedido por cualquiera.
      \item \cppkey{private}: El miembro puede ser accedido solamente por otros miembros.
      \item \cppkey{protected}: Relacionado con herencia.
    \end{itemize}
  \item Visibilidad por defecto:
    \begin{itemize}
      \item Los miembros de una estructura son públicos por defecto.
      \item Los miembros de una clase son privados por defecto.
    \end{itemize}
\end{itemize}
\vspace{-0.25cm}
\begin{lstlisting}
class X {
  int x; // x es privado
  double y; // y es privado
public:
  void f(int n); // f es pública
  void g(double x); // g es pública
private:
  void h(); // h es privada
};
\end{lstlisting}
\end{frame}

\begin{frame}[t,fragile]{Alcance de clase}
\begin{itemize}
  \item Las entidades declaradas dentro de una clase son 
        \textgood{accesibles} dentro de la definición de la clase.
  \item También puede accederse cualificando el acceso.
\end{itemize}
\begin{lstlisting}
class punto {
public:
  punto(double cx, double cy);
  double modulo() { return sqrt(x*x+y*y); }
  void simetria_x();
private:
  double x;
  double y;
};

punto::simetria_x() {
  y = -y;
}

\end{lstlisting}
\end{frame}

\begin{frame}[t,fragile]{Interfaz e implementación}
\begin{itemize}
  \item Los \textgood{niveles de visiblidad} permiten especificar y separar:
    \begin{itemize}
      \item La \textmark{interfaz pública} de la clase.
      \item Los \textmark{detalles privados} de implementación.
    \end{itemize}
\end{itemize}
\begin{lstlisting}
class ejemplo {
public:
  // Miembros públicos
  // Funciones, tipos y datos
private:
  // Miembros privados
  // Funciones, tipos y datos
};
\end{lstlisting}
\end{frame}

\begin{frame}[t]{Una clase fecha}
\begin{block}{fecha.h}
\lstinputlisting{09-clases/fecha3/fecha.h}
\end{block}
\end{frame}

\begin{frame}[t]{Uso de funciones}
\begin{block}{main.cpp}
\lstinputlisting{09-clases/fecha3/main.cpp}
\end{block}
\end{frame}

\begin{frame}[t]{Implementación de funciones}
\begin{block}{fecha.cpp}
\lstinputlisting[lastline=19]{09-clases/fecha3/fecha.cpp}
\end{block}
\end{frame}

\begin{frame}[t]{Implementación de funciones}
\begin{block}{fecha.cpp}
\lstinputlisting[firstline=21,lastline=39]{09-clases/fecha3/fecha.cpp}
\end{block}
\end{frame}

\begin{frame}[t]{Implementación de funciones}
\begin{block}{fecha.cpp}
\lstinputlisting[firstline=40,lastline=47]{09-clases/fecha3/fecha.cpp}
\end{block}
\end{frame}

\begin{frame}[t]{Implementación de funciones}
\begin{block}{fecha.cpp}
\lstinputlisting[firstline=49]{09-clases/fecha3/fecha.cpp}
\end{block}
\end{frame}

\begin{frame}[t,fragile]{Constructor}
\begin{itemize}
  \item Un \textgood{constructor} es una función miembro especial.
    \begin{itemize}
      \item Se usa para \textmark{iniciar objetos} del tipo definido por la clase.
      \item La sintaxis refuerza la invocación del constructor.
        \begin{itemize}
          \item Es obligatorio pasar agumentos del constructor al definir una variable.
        \end{itemize}
      \item Es una función miembro con el nombre de la clase y sin tipo de retorno.
    \end{itemize}
\begin{lstlisting}
class punto {
public:
  punto(double cx, double cy);
  // ...
private:
  double x;
  double y;
};
\end{lstlisting}
\end{itemize}
\end{frame}

\begin{frame}[t,fragile]{Invocación al constructor}
\begin{itemize}
  \item Dos notaciones para \textmark{invocar al constructor}.
\begin{lstlisting}
punto p1(2.3, 3.5); // Clásica
punto p2{2.3, 3.5}; // C++11
\end{lstlisting}

  \mode<presentation>{\vfill\pause}
  \item Si una clase tiene constructor \textbad{se deben} suministrar 
        \textmark{argumentos}.
\begin{lstlisting}
punto p; // Error faltan argumentos
\end{lstlisting}

  \mode<presentation>{\vfill\pause}
  \item Se puede iniciar \textmark{copiando otro objeto}.
\begin{lstlisting}
punto p1{2.5, 3.5};
punto p2 {p1};
punto p3 = p1;
punto p4 = punto(1.5, 2.5);
punto p5 = punto{2.5, 3.5};
\end{lstlisting}
    \begin{itemize}
      \item La copia por defecto es miembro a miembro.
    \end{itemize}
\end{itemize}
\end{frame}

\begin{frame}[t,fragile]{Implementación del constructor}
\begin{itemize}
  \item El contructor \textbad{debe iniciar} todos los \textmark{datos miembro} del tipo.
\begin{lstlisting}
punto::punto(double cx, double cy) {
  x = cx;
  y = cy;
}
\end{lstlisting}

  \mode<presentation>{\vfill\pause}
  \item Se puede usar una \textgood{lista de iniciación} de miembros.
\begin{lstlisting}
punto::punto(double cx, double cy) : x{cx}, y{cy}
{
}
\end{lstlisting}
    \begin{itemize}
      \item Evita el uso de un miembro antes de iniciarse.
      \item Fácilmente extensible con herencia.
    \end{itemize}
\end{itemize}
\end{frame}

\begin{frame}[t]{Construcción de fechas}
\begin{block}{fecha.h}
\lstinputlisting{09-clases/fecha4/fecha.h}
\end{block}
\end{frame}

\begin{frame}[t]{Implementación de constructor}
\begin{block}{fecha.cpp}
\lstinputlisting[lastline=10]{09-clases/fecha4/fecha.cpp}
\end{block}
\end{frame}

\begin{frame}[t]{Uso de constructor}
\begin{block}{main.cpp}
\lstinputlisting{09-clases/fecha4/main.cpp}
\end{block}
\end{frame}

