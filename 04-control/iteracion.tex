\section{Iteración}

\subsection{Bucles \textbf{while}}

\begin{frame}[t,fragile]{Mientras se cumple una condición}
\begin{itemize}
  \item Un bucle \cppkey{while} repite una sentencia mientras
        se cumple una condición.
\begin{lstlisting}
int opcion;
std::cin >> opcion;
while (opcion <0 || opcion >4)
  std::cin >> opcion;
std::cout << "Opción seleccionada: " << opcion << "\n";
\end{lstlisting}

\mode<presentation>{\vfill\pause}
  \item La sentencia puede ser un bloque compuesto.
\begin{lstlisting}
int opcion;
std::cin >> opcion;
while (opcion <0 || opcion >4) {
  std::cout << "No válida.\nOpción:";
  std::cin >> opcion;
}
std::cout << "Opción seleccionada: " << opcion << "\n";
\end{lstlisting}
\end{itemize}
\end{frame}

\begin{frame}[t,fragile]{Repetición con contador}
\begin{itemize}
  \item Imprime la lista de los cuadrados de los números naturales
        menores de 10.
\end{itemize}
\begin{block}{Lista de cuadrados}
\begin{lstlisting}
#include <iostream>

int main() {
  int i=0;
  while (i<10) {
    std::cout << i << "^2 = " << i*i << "\n";
    ++i;
  }
}
\end{lstlisting}
\end{block}
\end{frame}

\subsection{Bucles \textbf{do-while}}

\begin{frame}[t,fragile]{Evaluando la condición al final}
\begin{itemize}
  \item Similar a los bucles \cppkey{while} pero evaluando la
        condición de terminación al final de cada iteración.
\end{itemize}
\begin{lstlisting}
int opcion;
do {
  std::cout << "Opción:";
  std::cin >> opcion;
} 
while (opcion <0 || opcion >4);
std::cout << "Opción seleccionada: " << opcion << "\n";
\end{lstlisting}
\end{frame}

\subsection{Bucles \textbf{for}}

\begin{frame}[t,fragile]{Bucles \textbf{for}}
\begin{itemize}
  \item Especialmente útiles para repetir un número de veces.
\end{itemize}
\begin{block}{Lista de cuadrados}
\begin{lstlisting}
#include <iostream>

int main() {
  int i=0;
  for (int i=0; i<10; ++i) {
    std::cout << i << "^2 = " << i*i << "\n";
  }
}
\end{lstlisting}
\end{block}
\end{frame}

\begin{frame}[t,fragile]{Bucles \textbf{for} basadas en rango}
\begin{itemize}
  \item Permiten recorrer una lista de valores.
\begin{lstlisting}
for (int x : {1, 2, 3, 4}) {
  cout << i << '\n';
}
\end{lstlisting}
  \item Se pueden combinar con \cppkey{auto}.
\begin{lstlisting}
for (auto x : {1, 2, 3, 4}) {
  cout << i << '\n';
}
\end{lstlisting}

\end{itemize}
\end{frame}
