\section{Funciones}

\begin{frame}[t,fragile]{Funciones}
\begin{itemize}
  \item Una \textgood{función} es un subprograma que se puede invocar
        desde distintos lugares de un programa.
\end{itemize}
\begin{lstlisting}
#include <iostream>

void imprime_cuadrado(int i) {
  std::cout << i << "^2 = ";
  std::cout << i*i << "\n";
}

int main() {
  imprime_cuadrado(2);
  imprime_cuadrado(5);
}
\end{lstlisting}
\end{frame}

\begin{frame}[t,fragile]{Anatomía de una función}
\begin{itemize}
  \item \textgood{Tipo de retorno}: Tipo del resultado de la función.
    \begin{itemize}
      \item Si no hay resultado $\rightarrow$ \cppkey{void}.
    \end{itemize}
\begin{lstlisting}[escapechar=@]
@\color{red}\textbf{void}@ imprime_cuadrado(int i) {
\end{lstlisting}

  \mode<presentation>{\vfill\pause}
  \item \textgood{Nombre de la función}
\begin{lstlisting}[escapechar=@]
void @\color{red}\textbf{imprime\_cuadrado}@(int i) {
\end{lstlisting}

  \mode<presentation>{\vfill\pause}
  \item \textgood{Parámetros de la función}
\begin{lstlisting}[escapechar=@]
void imprime_cuadrado@\color{red}\textbf{(int i)}@ {
\end{lstlisting}

  \mode<presentation>{\vfill\pause}
  \item \textgood{Cuerpo de la función}
\begin{lstlisting}[escapechar=@]
void imprime_cuadrado(int i) @\color{red}\textbf{\{}@
  std::cout << i << "^2 = ";
  std::cout << i*i << "\n";
@\color{red}\textbf{\}}@
\end{lstlisting}
\end{itemize}
\end{frame}

\begin{frame}[t,fragile]{Resultado de una función}
\begin{itemize}
  \item Una función puede devolver un resultado.
    \begin{itemize}
      \item Hay que especificar el tipo del valor de retorno.
      \item Se debe utilizar la sentencia \cppkey{return} para devolver el valor.
    \end{itemize}
\end{itemize}

\mode<presentation>{\vfill\pause}
\begin{lstlisting}
double cuadrado(double x) {
  return x * x;
}

void f() {
  std::cout << "42^2 = " << cuadrado(42.0) << "\n";
}
\end{lstlisting}
\end{frame}
