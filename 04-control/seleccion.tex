\section{Selección}

\subsection{Sentencia \textbf{if}}

\begin{frame}[t,fragile]{Ejecución condicional}
\begin{itemize}
  \item Sentencia \cppkey{if}: Ejecuta dependiendo de una condición.
    \begin{itemize}
      \item La condición debe ser una expresión convertible a
            \textmark{boolean} entre paréntesis
    \end{itemize}
\begin{lstlisting}
if (edad>18) 
  std::cerr << "Mayor de edad\n";
\end{lstlisting}

  \mode<presentation>{\vfill\pause}
  \item Alternativa \cppkey{else}: Acción a ejecutar cuando no se cumple 
        la condición.
\begin{lstlisting}
if (edad>18) 
  std::cerr << "Mayor de edad\n";
else
  std::cerr << "Menor de edad\n";
\end{lstlisting}
\end{itemize}
\end{frame}

\begin{frame}[t,fragile]{Sentencias compuestas}
\begin{itemize}
  \item Tanto la sentencia asociada al \cppkey{if} (\textmark{sentencia then})
        como la sentencia asociada a la parte \cppkey{else} (\textmark{sentencia else})
        pueden ser un bloque de sentencias.
    \begin{itemize}
      \item Una secuencia de sentencias entre llaves.
    \end{itemize}
\begin{lstlisting}
if (a>b) {
  maximo = a;
  minimo = b;
}
else {
  maximo = b;
  minimo = a;
}
\end{lstlisting}
\end{itemize}
\end{frame}

\begin{frame}[t,fragile]{Selección encadenada}
\begin{itemize}
  \item Tanto la \textmark{parte then} como la \textmark{parte else} pueden
        ser a su vez una sentencia de selección.
    \begin{itemize}
      \item Permite encadenar selecciones.
    \end{itemize}
\begin{lstlisting}
if (talla=='S') {
  std::cout << "pequeña\n";
}
else if (talla=='M') {
  std::cout << "mediana\n";
}
else if (talla=='L') {
  std::cout << "grande\n";
}
else {
  std::cout << "desconocida\n";
}
\end{lstlisting}
\end{itemize}
\end{frame}

\subsection{Sentencia \textbf{switch}}

\begin{frame}[t,fragile]{Alternativa múltiple}
\begin{itemize}
  \item Permite elegir entre varias alternativas dependiendo del valor
        de una expresión.
\begin{lstlisting}
switch (talla) {
  case 'S':
    std::cout << "pequeña\n";
    break;
  case 'M':
    std::cout << "mediana\n";
    break;
  case 'L':
    std::cout << "grande\n";
    break;
  default:
    std::cout << "desconocida\n";
}
\end{lstlisting}
\end{itemize}
\end{frame}

\begin{frame}[t,fragile]{Tipo de expresión}
\begin{itemize}
  \item El tipo de la expresión de selección debe ser un tipo entero,
        carácter o 
        \mode<presentation>{enumerado.}
        \mode<article>{enumerado~\footnote{Los enumerados se presentarán en el tema~\ref{cap:enum}}.}
\end{itemize}

\begin{columns}[T]

\column[t]{.5\textwidth}
\begin{block}{OK}
\begin{lstlisting}
int x;
std::cin >> x;
switch (x) {
  case 0:
    std::cout << "cero\n";
    break;
  case 1:
    std::cout << "uno\n";
    break;
  default:
    std::cout << "otro\n";
}
\end{lstlisting}
\end{block}

\mode<presentation>{\pause}
\column[t]{.5\textwidth}
\begin{itemize}
  \item No puede ser un tipo cadena como \cppid{std::string}.
\end{itemize}
\begin{block}{Error}
\begin{lstlisting}
std::string s;
std::cin >> s;
switch (s) { // Error. s es cadena
  //...
}
\end{lstlisting}
\end{block}
\end{columns}
\end{frame}

\begin{frame}[t,fragile]{Etiquetas}
\begin{itemize}
  \item Las etiquetas en los \cppkey{case} deben ser expresiones constantes.
    \begin{itemize}
      \item No se puede poner una variable en un \cppkey{case}.
    \end{itemize}
\end{itemize}

\begin{columns}[T]

\column{.5\textwidth}
\begin{block}{OK}
\begin{lstlisting}
char c;
std::cin >> c;
switch (c) {
  case 'M':
    std::cout << "mediana\n";
    break;
  //...
}
\end{lstlisting}
\end{block}

\column{.5\textwidth}
\begin{block}{Error}
\begin{lstlisting}
char c;
std::cint >> c;
const char med1 = 'M';
constexpr char med2 = 'M';

switch (c) {
  case med1: // Error
    std::cout << "mediana\n";
    break;
  case med2: // OK
    std::cout << "mediana\n";
    break;
  //...
}
\end{lstlisting}
\end{block}

\end{columns}
\end{frame}

\begin{frame}[t,fragile]{Repetición}
\mode<presentation>{\vspace{-1em}}
\begin{columns}[T]

\column{.5\textwidth}
\begin{itemize}
  \item Una etiqueta puede aparecer una única vez.
\end{itemize}

\begin{block}{Etiquetas repetidas}
\begin{lstlisting}
char c;
std::cin >> c;
switch (c) {
  case 'l':
    std::cout << "lunes\n";
    break;
  case 'm':
    std::cout << "martes\n";
    break;
  case 'm': // Error: etiqueta repetida
    std::cout << "miércoles\n";
    break;
  //...  
}
\end{lstlisting}
\end{block}

\mode<presentation>{\pause}
\column{.5\textwidth}
\begin{itemize}
  \item Se pueden usar varias etiquetas para una rama.
\end{itemize}
\begin{block}{Rama múltiple}
\begin{lstlisting}
char c;
std::cin >> c;
switch (c) {
  case 'a':
  case 'e':
  case 'i':
  case 'o':
  case 'u':
    std::cout << "vocal\n";
    break;
  default:
    std::cout << "consonante\n"
}
\end{lstlisting}
\end{block}

\end{columns}
\end{frame}

\begin{frame}[t,fragile]{Terminación de ramas}
\begin{itemize}
  \item La última sentencia de una rama debe ser \cppkey{break}.
    \begin{itemize}
      \item En otro caso se continúa con la siguiente rama.
    \end{itemize}
\end{itemize}

\mode<presentation>{\vfill}
\begin{columns}[T]

\column{.5\textwidth}
\begin{block}{Ramas en cascada}
\begin{lstlisting}
switch (nota) {
  case 'S':
    std::cout << "Excelente\n";
  case 'N':
    std::cout << "Buen trabajo\n";
    break;
  case 'A':
    std::cout << "Aprobado\n";
    break;
  default:
    std::cout << "Sigue trabajando\n";
}
\end{lstlisting}
\end{block}

\column{.5\textwidth}
{\scriptsize
\begin{itemize}
  \item \textmark{'S'}:
\begin{lstlisting}[style=terminal,basicstyle=\tiny]
Excelente
Buen trabajo
\end{lstlisting}
  \item \textmark{'N'}:
\begin{lstlisting}[style=terminal,basicstyle=\tiny]
Excelente
Buen trabajo
\end{lstlisting}
  \item \textmark{'A'}:
\begin{lstlisting}[style=terminal,basicstyle=\tiny]
Aprobado
\end{lstlisting}
  \item Otra letra:
\begin{lstlisting}[style=terminal,basicstyle=\tiny]
Sigue trabajando
\end{lstlisting}
\end{itemize}
}
\end{columns}
\end{frame}

\begin{frame}[t,fragile]{Ramas en cascada}
\begin{itemize}
  \item Cuando una rama no termina en \cppkey{break} es probable que se
        obtenga una advertencia al compilar.
    \begin{itemize}
      \item Se puede usar el atributo \cppkey{[[fallthrough]]}.
      \mode<article>{
        \item Indica intención de pasar a la siguiente rama.
       }
    \end{itemize}
\end{itemize}

\begin{block}{Ramas en cascada y [[fallthrough]]}
\begin{lstlisting}
switch (nota) {
  case 'S':
    std::cout << "Excelente\n";
    [[fallthrough]];
  case 'N':
    std::cout << "Buen trabajo\n";
    break;
  case 'A':
    std::cout << "Aprobado\n";
    break;
  default:
    std::cout << "Sigue trabajando\n";
}
\end{lstlisting}
\end{block}
\end{frame}
