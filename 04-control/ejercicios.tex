\section{Ejercicios}

\begin{enumerate}

\item Escribe un programa que imprima las soluciones de una ecuación de segundo grado

\[
a x^2 + b x + c = 0
\]

El programa debe pedir los valores para los coeficientes \cppid{a}, \cppid{b} y
\cppid{c} e imprimir los resultados.

\textmark{Nota}: Para hallar la raíz cuadrada de un valor puedes usar la función
\cppid{std::sqrt()} (detalles en 
\url{https://en.cppreference.com/w/cpp/numeric/math/sqrt}).

\item Escribe un programa que imprima una tabla con todas las letras minúsculas
      de la \cppstr{'a'} a la \cppstr{'z'}. Para cada letra, imprime en la misma
      línea su código numérico.

\textmark{Nota}: Para obtener el código numérico de una variable de tipo \cppkey{char}
         puedes utilizar el operador \cppkey{static\_cast}
         (detalles en \url{https://en.cppreference.com/w/cpp/language/static_cast}).

\end{enumerate}
