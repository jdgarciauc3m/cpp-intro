\section{Ejercicios}

\begin{enumerate}

\item Escribe un programa que imprima las soluciones de una ecuación de segundo grado

\[
a x^2 + b x + c = 0
\]

El programa debe pedir los valores para los coeficientes \cppid{a}, \cppid{b} y
\cppid{c} e imprimir los resultados.

\textmark{Nota}: Para hallar la raíz cuadrada de un valor puedes usar la función
\cppid{std::sqrt()} (detalles en 
\url{https://en.cppreference.com/w/cpp/numeric/math/sqrt}).

\item Escribe un programa que imprima una tabla con todas las letras minúsculas
      de la \cppstr{'a'} a la \cppstr{'z'}. Para cada letra, imprime en la misma
      línea su código numérico.

\textmark{Nota}: Para obtener el código numérico de una variable de tipo \cppkey{char}
         puedes utilizar el operador \cppkey{static\_cast}
         (detalles en \url{https://en.cppreference.com/w/cpp/language/static_cast}).

\item Escribe un programa que pida un valor real \cppid{x} y determine su seno y su
      coseno mediante el desarrollo en serie de MacLaurin para los 10 primeros términos
      de la serie. Compara el valor obtenido con el de las funciones \cppid{std::sin()} 
      y \cppid{std::cos()}.

\textmark{Nota}: Los desarrollos en serie correspondientes son:

\[
cos(x) = 1 - \frac{x^2}{2!} + \frac{x^4}{4!} - \ldots + \frac{(-1)^n x^{2n}}{(2n)!} + \ldots
=
\sum_{k=0}^{+\infty} \frac{(-1)^k x^{2k}}{(2k)!}
\]

\[
sin(x) = x - \frac{x^3}{3!} + \frac{x^5}{5!} - \ldots + \frac{(-1)^n x{2n+1}}{(2n+1)!} + \ldots
=
\sum_{k=0}^{+\infty} \frac{(-1)^k x^{2k+1}}{(2k+1)!}
\]

\end{enumerate}
