\section{Clases de alcance}

\begin{frame}[fragile]{Alcance}
\begin{itemize}
  \item Un \alert{alcance} es una región de código fuente.
    \begin{itemize}
      \item Normalmente delimitada por un par de llaves.
      \item El nombre de una entidad es válido desde su punto de declaración hasta el
            final del alcance en que se produjo dicha delcaración.
      \item Se pueden anidar alcances.
    \end{itemize}
\end{itemize}
\begin{lstlisting}
int x; // Alcance global

void f() {
  int y; // Alcance local
  y = x;
  if (x == 0) {
    int z = 2; // Alcance anidado
    // ...
  }
}
\end{lstlisting}
\end{frame}

\begin{frame}{Clases de alcance}
\begin{itemize}
  \item \alert{Alcance global}: No incluido en ningún otro.
  \item \alert{Alcance de espacio de nombres}: Alcance con un nombre asociado.
    \begin{itemize}
      \item Puede estar anidado en el alcance global o en otro espacio de nombres.
    \end{itemize}
  \item \alert{Alcance de clase}: Alcance de la definición de una clase.
  \item \alert{Alcance local}: Alcance de un bloque delimitado por un par de llaves.
  \item \alert{Alcance de sentencia}: Alcance vinculado a algunas sentencias (p.ej. bucles-\emph{for}).
\end{itemize}
\end{frame}

\begin{frame}[fragile]{Alcance global}
\begin{itemize}
  \item Las entidades declaradas en el alcance global son accesibles desde cualquier entidad del programa.
  \item Una entidad con alcance global puede ser accedida desde su punto de declaración hasta el final de la unidad
        de traducción en que se declaró.
\end{itemize}
\begin{lstlisting}
#include <iostream>

int x; // Global

void f() {
  std::cout << x << std::endl;
}
\end{lstlisting}
\end{frame}

\begin{frame}[fragile]{Alcance de espacio de nombres}
\begin{itemize}
  \item La entidades declaradas en un espacio de nombres son accesibles dentro del mismo.
  \item Una entidad puede ser accedida desde su punto de declaración hasta el final de la unidad de traducción
        siempre que se esté dentro del espacio de nombres.
  \item También puede accederse cualificando el acceso.
\end{itemize}
\begin{lstlisting}
namespace geom {
  double area(double l, double a);
  double volumen(double nx, double ny, double nz) {
    return area(nx,ny) * nz;
  }
} // Fin de espacio de nombres

void f() {
  double a1 = geom::area(2,3);
  double a2 = volumen(2,3,4); // Error: volumen no esta en alcance
  // ...
}
\end{lstlisting}
\end{frame}

\begin{frame}[fragile]{Alcance de clase}
\begin{itemize}
  \item Las entidades declaradas dentro de una clase son accesible dentro de la definición de la clase.
  \item También puede accederse cualificando el acceso.
\end{itemize}
\begin{lstlisting}
class punto {
public:
  punto(double cx, double cy);
  double modulo() { return sqrt(x*x+y*y); }
  void simetria_x();
private:
  double x;
  double y;
};

punto::simetria_x() {
  y = -y;
}

\end{lstlisting}
\end{frame}

\begin{frame}[fragile]{Alcance local}
\begin{itemize}
  \item Las entidades declaradas dentro de una función son accesibles solamente dentro de la función.
  \item Una entidad puede ser accedida desde su punto de declaración hasta el final del bloque en que se encuentra
        dicha declaración.
  \item Los parámetros se consideran locales al bloque de primer nivel de la función.
\end{itemize}
\begin{lstlisting}
void f(int x, int y) {
  int z = (x>y) ? x : y;
  if (z>0) {
    int t = x - y;
    // ...
  } // Fin de alcance de t
} // Fin de alcance de x, y, z
\end{lstlisting}
\end{frame}

\begin{frame}[fragile]{Alcance de sentencia}
\begin{itemize}
  \item La entidades declaradas dentro de los paréntesis de sentencias \cppkey{for}, \cppkey{while}, \cppkey{if} o \cppkey{switch}
        tienen alcance de sentencia.
  \item Una entidad puede ser accedida solamente dentro de la sentencia correspondiente.
\end{itemize}
\begin{lstlisting}
void f(int n) {
  for (int i=0; i<n; ++i) {
    std::cout << i;
    std::cout << " -> ";
    int cuad = i*i;
    std::cout << cuad << std::endl;
  } // Fin de alcance para i, cuad
}
\end{lstlisting}
\end{frame}


\begin{frame}[fragile]{Espacio de nombres}
\begin{itemize}
  \item Un \alert{espacio de nombres} permite agrupar un conjunto de entidades en un alcance con nombre.
\end{itemize}
\begin{lstlisting}
#include <iostream>

namespace geom {
  area_rect(double nx, double ny);
  area_circ(double r);
  // ...
}

void f() {
  std::cout << geom::area_rect(2.0, 3.5) << std::endl;
}
\end{lstlisting}
\end{frame}

\begin{frame}[fragile]{Declaraciones de uso}
\begin{itemize}
  \item Declara un sinónimo local para una entidad de otro espacio de nombres.
\end{itemize}
\begin{lstlisting}
#include <iostream>

namespace geom {
  area_rect(double nx, double ny);
  area_circ(double r);
  // ...
}

void f() {
  using std::cout;
  using std::endl;
  using geom::area_rect;
  cout << area_rect(2.0, 3.5) << endl;
}
\end{lstlisting}
\end{frame}

\begin{frame}[fragile]{Directiva de uso}
\begin{itemize}
  \item Hace que se busquen los nombres no encontrados en otro espacio de nombres.
\end{itemize}
\begin{lstlisting}
#include <iostream>

namespace geom {
  area_rect(double nx, double ny);
  area_circ(double r);
  // ...
}

void f() {
  using namespace std;
  using namespace geom;
  cout << area_rect(2.0, 3.5) << endl;
}
\end{lstlisting}
\end{frame}
