\section{Errores y excepciones}

\begin{frame}{Errores}
\begin{quote}
Cuiusvis hominis est errare, nullius nisi insipientis in errore perseverare.

\textbf{Marco Tulio Cicerón}
\end{quote}

\pause

\begin{quote}
Beware of bugs in the above code; I have only proved it correct, not tried it.

\textbf{Donald Knuth}
\end{quote}
\end{frame}


\begin{frame}{Tipos de errores}
\begin{itemize}
  \item En tiempo de desarrollo.
    \begin{itemize}
      \item Detectados durante la fase de desarrollo de software.
      \item \alert{Errores de compilación}: Violación de las reglas del lenguaje que
      se puede detectar dentro de una unidad de traducción.
      \item \alert{Errores de enlace}: Incosistencias entre distintas unidades de 
      traducción compiladas detectadas durante la fase de enlace.
    \end{itemize}

  \item En tiempo de ejecución.
    \begin{itemize}
      \item Detectados durante la ejecución del software.
      \item \alert{Externos}: Detectados por hardware, sistema operativo, software externo, \ldots
      \item \alert{Biblioteca}: Detectados por comprobaciones realizadas por la biblioteca.
      \item \alert{Internos}: Detectados (o no) por código desarrollado.
    \end{itemize}

  \item{Objetivo final}: Poder traducir el código a un programa que produzca los resultados deseados.
\end{itemize}
\end{frame}

\begin{frame}{Lenguajes estáticos versus dinámicos}
\begin{itemize}
  \item Alternativas:
    \begin{itemize}
      \item \alert{Lenguajes estáticos}: Comprobaciones realizadas en tiempo de traducción.
        \begin{itemize}
          \item Se pueden detectar más errores antes de comenzar la ejecución.
          \item Pueden generar código con mejor rendimiento.
          \item Las comprobaciones presentan límites teóricos y prácticos.
        \end{itemize}
      \item \alert{Lenguajes dinámicos}: Comprobaciones realizadas en tiempo de ejecución.
        \begin{itemize}
          \item Los errores se detectan al ejecutar programa.
          \item Mayor peligro de errores no detectados en producto final.
          \item Menor rendimiento derivado de más comprobaciones en tiempo de ejecución.
        \end{itemize}
    \end{itemize}
  \item La mayoría de los lenguajes presentan una combinación de las dos aproximaciones.
\end{itemize}
\end{frame}

\begin{frame}[fragile]{Errores de compilación}
\begin{lstlisting}
double cuadrado(double x);
\end{lstlisting}
\begin{itemize}
  \item Errores de compilación.
\begin{lstlisting}
double a = cuadrado(2,0); // Error: 2.0
double b = cuadrado(3) // Falta punto y coma
\end{lstlisting}
  \item Errores de tipo.
\begin{lstlisting}
double c = cuadrado("hola"); // No se pude convertir a double
\end{lstlisting}
  \item Hay otros errores que el compilador no está obligado a detectar.
    \begin{itemize}
      \item Activar advertencias del compilador.
        \begin{itemize}
          \item \verb|-Wall -Weffc++ -Werror -Wpedantic -Wextra|
        \end{itemize}
      \item Usar herramientas de análisis estático de código.
    \end{itemize}
\end{itemize}
\end{frame}

\begin{frame}[fragile]{Errores de enlace}
\begin{itemize}
  \item Cuando el enlazador (\emph{linker}) no encuentra algo.
\lstinputlisting{05-errores/enlace/main.cpp}
  \item Salida del compilador.
\begin{lstlisting}[language=bash,basicstyle=\scriptsize\ttfamily]
/tmp/ccGYMTH0.o: In function `main':
main.cpp:(.text+0x31): undefined reference to `cuadrado(int)'
collect2: error: ld devolvio el estado de salida 1
make: *** [test] Error 1
\end{lstlisting}
  \mode<article>{
  \item Hay más variedad de errores de enlace en C++ de los que se puede encontrar
  en otros lenguajes como C. Esto se debe, entre otras razones, a que se suele
  construir software bastante más complejo usando C++.
  }
\end{itemize}
\end{frame}
