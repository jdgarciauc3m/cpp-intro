\section{Precondiciones}

\begin{frame}[fragile]{Precondiciones}
\begin{itemize}
  \item Una \alert{precondición} es una condición que debe cumplirse
        antes de ejecutar una función.
\begin{lstlisting}
void imprime(vector<string> personas, vector<int> edades);
// Pre: personas.size() == edades.size()
\end{lstlisting}
  \item Alternativas:
    \begin{itemize}
      \item Comprobar todas las precondiciones.
        \begin{itemize}
          \item Requiere una acción si se incumple (¿excepción?).
          \item No siempre es factible.
          \item Computacinalmente más costoso.
        \end{itemize}
      \item No realizar las comprobaciones.
        \begin{itemize}
          \item El llamante es responsable de cumplir la precondición.
          \item Computacionalmente menos costoso.
          \item Puede dar lugar a errores.
        \end{itemize}
    \end{itemize}
\end{itemize}
\end{frame}

\mode<presentation>{

\begin{frame}
\begin{block}{check.cpp}
\lstinputlisting[lastline=17]{04-errores/prepost/check.cpp}
\ldots
\end{block}
\end{frame}

\begin{frame}
\begin{block}{check.cpp}
\ldots
\lstinputlisting[firstline=19,lastline=31]{04-errores/prepost/check.cpp}
\ldots
\end{block}
\end{frame}

\begin{frame}
\begin{block}{check.cpp}
\ldots
\lstinputlisting[firstline=33]{04-errores/prepost/check.cpp}
\end{block}
\end{frame}

}

\mode<article>{
\begin{frame}{Comprobación de precondiciones}
\begin{block}{check.cpp}
\lstinputlisting{04-errores/prepost/check.cpp}
\end{block}
\end{frame}
}

\begin{frame}[t,fragile]{Salida}
\begin{lstlisting}[style=terminal]
Error: Distinta longitud
Carlos -> 10
Daniel -> 43
José -> 67
Manuel -> 98
Maria -> 8
\end{lstlisting}
\begin{itemize}
  \item La primera invocación genera una excepción porque las longitudes son distintas.
  \item Se añade un valor al \cppid{vector e}.
  \item La segunda invocación funciona correctamente.
\end{itemize}
\end{frame}


\begin{frame}{Ausencia de comprobación}
\begin{itemize}
  \item La precondición no se comprueba.
  \mode<article>{
    \begin{itemize}
      \item Es responsabilidad del llamante cumplirla.
    \end{itemize}
  }
\end{itemize}
\pause
\begin{block}{nocheck.cpp}
\mode<presentation>{
\lstinputlisting[firstline=1,lastline=14]{04-errores/prepost/nocheck.cpp}
\ldots
}
\mode<article>{
\lstinputlisting{04-errores/prepost/nocheck.cpp}
}
\end{block}
\end{frame}

\mode<presentation>{
\begin{frame}
\begin{block}{nocheck.cpp}
\ldots
\lstinputlisting[firstline=15]{04-errores/prepost/nocheck.cpp}
\end{block}
\end{frame}
}

\begin{frame}[fragile]{Salida}
\begin{lstlisting}[style=terminal]
Carlos -> 10
Daniel -> 43
José -> 67
Manuel -> 98
Maria -> 0
Carlos -> 10
Daniel -> 43
José -> 67
Manuel -> 98
Maria -> 8
\end{lstlisting}
\begin{itemize}
  \item \alert{Peligro}: El primer valor para María no está garantizado.
  \item Otras implementaciones podrían causar errores
\end{itemize}
\end{frame}

\subsection{Postcondiciones}

\begin{frame}[fragile]{Postcondiciones}
\begin{itemize}
  \item Una \alert{postcondición} es una condición que debe cumplirse tras
        la ejecución de una función.
  \item Pueden ser más complejas y costosas de comprobar.
\begin{lstlisting}
vector<int> copia_ordenado(vector<int> v);
// Post: Resultado es copia ordenada de v.
\end{lstlisting}
  \item Es una buena práctica documentarlas.
  \item El incumplimiento de una postcondición puede venir dado por:
    \begin{itemize}
      \item Incumplimiento de una precondición.
      \item Error de programación.
    \end{itemize}
  \item No se suelen comprobar.
\end{itemize}
\end{frame}
