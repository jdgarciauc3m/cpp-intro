\section{Depuración}

\begin{frame}{Razones para depurar}
\begin{itemize}
  \item Hay diversas razones para depurar un programa:
    \begin{itemize}
      \item Se obtienen resultados no esperados.
      \item El programa termina de forma abrupta.
      \item Comprender mejor el funcionamiento del programa.
    \end{itemize}
  \item Alternativas de depuración:
    \begin{itemize}
      \item Uso de un entorno de depuración.
      \item Inserción de sentencias de impresión en el código.
    \end{itemize}
  \item Recuerda:
    \begin{itemize}
      \item La alternativa a usar depende mucho de las circunstancias.
      \item Es complicado depurar ciertas aplicaciones: multi-hilo, fuentes de eventos externas, tiempo real.
    \end{itemize}
\end{itemize}
\end{frame}

\begin{frame}{Herramientas de depuración}
\begin{itemize}
  \item Algunos ejemplos:
    \begin{itemize}
      \item gdb.
      \item \cppid{Code::Blocks}.
      \item DDD.
      \item Eclipse CDT.
      \item KDevelop.
      \item Nemiver.
    \end{itemize}
  \item \url{http://www.drdobbs.com/testing/13-linux-debuggers-for-c-reviewed/240156817}.
\end{itemize}
\end{frame}

\subsection{Pruebas}

\begin{frame}{Pruebas}
\begin{itemize}
  \item Las pruebas son parte esencial del proceso de desarrollo de software.
  \item Existen muchas herramientas de pruebas unitarias para C++.
  \item Altamente recomendables:
    \begin{itemize}
      \item GTest (Google).
      \item CPPUnit.
      \item CUTE (HSR Rapperswil).
    \end{itemize}
  \item Recomendaciones:
    \begin{itemize}
      \item Escribir pruebas a la vez que el código (o incluso antes).
      \item Ejecutar todas las pruebas después de cada construcción.
    \end{itemize}
\end{itemize}
\end{frame}
