\section{Ejemplo: Biblioteca de calendario}

\mode<presentation> {

\begin{frame}
\begin{block}{mes.h}
\lstinputlisting[lastline=21]{12-interfaz/fecha7/mes.h}
\end{block}
\end{frame}

\begin{frame}
\begin{block}{mes.h}
\lstinputlisting[firstline=23]{12-interfaz/fecha7/mes.h}
\end{block}
\end{frame}

}

\mode<article> {
\begin{frame}
\begin{block}{mes.h}
\lstinputlisting{12-interfaz/fecha7/mes.h}
\end{block}
\end{frame}
}

\mode<presentation> {

\begin{frame}
\begin{block}{mes.cpp}
\lstinputlisting[lastline=13]{12-interfaz/fecha7/mes.cpp}
\end{block}
\end{frame}

\begin{frame}
\begin{block}{mes.cpp}
\lstinputlisting[firstline=15, lastline=23]{12-interfaz/fecha7/mes.cpp}
\end{block}
\end{frame}

\begin{frame}
\begin{block}{mes.cpp}
\lstinputlisting[firstline=25]{12-interfaz/fecha7/mes.cpp}
\end{block}
\end{frame}

}

\mode<article> {

\begin{frame}
\begin{block}{mes.cpp}
\lstinputlisting{12-interfaz/fecha7/mes.cpp}
\end{block}
\end{frame}

}

\mode<presentation> {

\begin{frame}
\begin{block}{fecha.h}
\lstinputlisting[lastline=19]{12-interfaz/fecha7/fecha.h}
\end{block}
\end{frame}

\begin{frame}
\begin{block}{fecha.h}
\lstinputlisting[firstline=21]{12-interfaz/fecha7/fecha.h}
\end{block}
\end{frame}

}

\mode<article> {
\begin{frame}
\begin{block}{fecha.h}
\lstinputlisting{12-interfaz/fecha7/fecha.h}
\end{block}
\end{frame}
}

\mode<presentation>{

\begin{frame}
\begin{block}{fecha.cpp}
\lstinputlisting[lastline=20]{12-interfaz/fecha7/fecha.cpp}
\end{block}
\end{frame}

\begin{frame}
\begin{block}{fecha.cpp}
\lstinputlisting[firstline=22,lastline=36]{12-interfaz/fecha7/fecha.cpp}
\end{block}
\end{frame}

\begin{frame}
\begin{block}{fecha.cpp}
\lstinputlisting[firstline=38]{12-interfaz/fecha7/fecha.cpp}
\end{block}
\end{frame}

}

\mode<article>{
\begin{frame}
\begin{block}{fecha.cpp}
\lstinputlisting{12-interfaz/fecha7/fecha.cpp}
\end{block}
\end{frame}
}
