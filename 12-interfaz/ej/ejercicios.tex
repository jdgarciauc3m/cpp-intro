\section{Ejercicios}

\begin{enumerate}

\item
Diseña los tipos de datos para representar una opción financiera, teniendo en cuenta los siguientes requisitos:

\begin{itemize}

\item Una opción puede ser de dos tipos: \textgood{call} (opción de compra) o
      \textgood{put} opción de venta. Utiliza un \textmark{tipo enumerado} para
      representar los tipos de opción.

\item Una opción tiene asociado una \textgood{prima} que es el valor al que se
      ha comprado la opción.

\item Una opción tiene asociado un \textgood{precio} que es el valor al que se
      compra o se vende el activo subuyacente a la compra.

\item Una opción ofrece dos operaciones importantes:

  \begin{itemize}

    \item \cppid{valor\_vencimiento(valor\_subyacente)}: Determina el valor al
          vencimiento de la opción dependiendo del valor del activo subyacente.

          Si se trata de una opción de compra, el valor será la diferencia entre
          el \cppid{valor\_subyacente} y el \cppid{precio} si este es positivo o
          \cppid{0} en otro caso.

          Si se trata de una opción de venta, el valor será la diferencia entre
          el \cppid{precio} y el \cppid{valor\_subyacente} si este es positivo o
          \cppid{0} en otro caso. 

    \item \cppid{beneficio\_vencimiento(valor\_subyacente)}: Determina el
          beneficio al vencimiento de la opción dependiendo del valor del activo
          subyacente.

          El beneficio depende del valor al vencimiento. Si el valor al vencimiento
          es mayor que la \cppid{prima}, el beneficio es la diferencia entre ambos.
          En otro caso, el beneficio es \cppid{0}.

  \end{itemize}

\end{itemize}

Escribe un programa que lea de la entrada estándar líneas con datos de opciones
y valores subyacentes y escriba a la salida estándar el valor y el beneficio al
rendimiento de cada una de las opciones.

El formato de las líneas incluirá en este orden: tipo, prima, precio, valorsuby.
Por ejemplo:

  \begin{itemize}

    \item PUT 100.0 90.0 95.0
    \item CALL 90.5 95.0 83.5
    \item \ldots

  \end{itemize}

\end{enumerate}
