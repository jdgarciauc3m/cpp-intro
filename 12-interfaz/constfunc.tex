\section{Inmutabilidad}

\begin{frame}[t,fragile]{Modificabilidad}
\begin{itemize}
  \item Un \textgood{objeto} puede ser:
    \begin{itemize}
      \item \textgood{Modificable}: Si se puede alterar su valor.
      \item \textgood{Constante}: Si no se puede modificar su valor.
    \end{itemize}

  \mode<presentation>{\vfill\pause}
  \item Una \textgood{función} puede aceptar:
    \begin{itemize}
      \item Parámetros por \textmark{referencia no-constante}:
        \begin{itemize}
          \item Solamente argumentos modificables.
        \end{itemize}
\begin{lstlisting}
void f(T & x);
\end{lstlisting}

      \item Parámetros por \textmark{valor} o por \textmark{referencia constante}:
        \begin{itemize}
          \item Argumentos modificables y no modificables.
        \end{itemize}
\begin{lstlisting}
void g(const T & x);
void h(T x);
\end{lstlisting}
    \end{itemize}
\end{itemize}
\end{frame}

\begin{frame}[t,fragile]{Parámetros por referencia no-constante}
\begin{itemize}
  \item \textbad{No se puede pasar}:
    \begin{itemize}
      \item Objetos constantes.
      \item Resultados de expresiones (objetos temporales).
    \end{itemize}
\end{itemize}
\begin{lstlisting}
void f(int & x) {
  ++x;
  cout << x << endl;
}

void g() {
  int z = 1;
  const int t = 2;
  f(z);
  f(t); // Error: No puede pasar objeto constante
  f(z+t); // Error: No puede pasar objeto temporal
}
\end{lstlisting}
\end{frame}

\begin{frame}[t,fragile]{Parámetros por referencia constante}
\begin{itemize}
  \item \textgood{Si se pueden pasar}:
    \begin{itemize}
      \item Objetos constantes.
      \item Resultados de expresiones (objetos temporales).
    \end{itemize}
\end{itemize}
\begin{lstlisting}
void f(const int & x) {
  ++x; // Error no se puede modificar x
  cout << x << endl;
}

void g() {
  int z = 1;
  const int t = 2;
  f(z);
  f(t); 
  f(z+t); 
}
\end{lstlisting}
\end{frame}

\begin{frame}[fragile]{Parámetros y funciones miembro}
\begin{itemize}
  \item Cuando se invoca una función miembro se pasa implícitamente una 
        \textgood{referencia al objeto} para el que se invoca la función 
        (el objeto \cppkey{*this}).
\begin{lstlisting}
struct fecha {
  // ...
  int dia() { return dia_; }
  // ...
  int dia_; // Un dato miembro y una función miembro no se pueden llamar igual
};

int que_dia(fecha & f) { return f.dia_; }

fecha cumple{4, id_mes::septiembre, 1969 };
int d1 = cumple.dia();    // d1 == que_dia(f)
int d2 = que_dia(cumple); // d2 == f.dia()
const fecha inicio{1, id_mes::enero, 1969 };
int d3 = inicio.dia();    // Error: inicio es const
int d4 = que_dia(inicio); // Error: inicio es const
\end{lstlisting}
\end{itemize}
\end{frame}

\begin{frame}[t,fragile]{Función miembro constante}
\begin{itemize}
  \item Una \textmark{función miembro} marcada como \textemph{constante} 
        recibe de forma implícita una \textemph{referencia constante}
        al objeto para el que se invoca la función (el objeto \cppid{*this}).
    \begin{itemize}
      \item Su implementación \textbad{no pude modificar} ningún dato miembro.
      \item \textgood{Se puede invocar} para objetos 
            modificables y para objetos constantes.
    \end{itemize}
\begin{lstlisting}
class punto {
public:
  punto();
  double x() const { return x_; }
  double y() const { return y_; }
private:
  double x_, y_;
};
punto p1{2.0, 2.0};
auto x = p1.x();
const punto p2{0.0, 0.0};
auto y = p2.y();
\end{lstlisting}
\end{itemize}
\end{frame}

\begin{frame}[t,fragile]{Resultados no ignorables}
\begin{itemize}
  \item Normalmente, el resultado de una función se puede ignorar.
    \begin{itemize}
      \item \textgood{Evitable} usando marcador \cppkey{[[nodiscard]]}.
    \end{itemize}
\end{itemize}

\begin{columns}[T]

\pause
\column{.4\textwidth}

\begin{lstlisting}
class punto {
public:
  punto();
  double x() const { return x_; }
  double y() const { return y_; }
private:
  double x_, y_;
};

punto p1{2.0, 2.0};
p1.x(); // OK
double x = p1.x(); // OK
\end{lstlisting}

\pause
\column{.6\textwidth}
\begin{lstlisting}
class punto {
public:
  punto();
  [[nodiscard]] double x() const { return x_; }
  [[nodiscard]] double y() const { return y_; }
private:
  double x_, y_;
};

punto p1{2.0, 2.0};
p1.x(); // Error
double x = p1.x(); // OK
\end{lstlisting}

\end{columns}

\end{frame}

\begin{frame}[t,fragile]{Miembros de acceso: ¿Siempre?}
\begin{lstlisting}
struct punto1 {
  double x, y;
};

class punto2 {
public:
  punto2() : x_{0}, y_{0} {}
  punto2(double x, double y) : x_{x}, y_{y} {}
  
  void x(double x) { x_ = x; }
  double x() const { return x_; }
  void y(double y) { y_ = y; }
  double y() const { return y_; }

private:
  double x_, y_;
};
\end{lstlisting}
\begin{itemize}
  \item Eficiencia / Mantenibilidad / Evolución.
\end{itemize}
\end{frame}
