\section{Definiciones en-línea}

\begin{frame}[t,fragile]{Función en-linea}
\begin{itemize}
  \item Una definición \textgood{en-línea} es una \textmark{sugerencia} 
        al compilador para que expanda
        el cuerpo de una función, en vez de generar el código de llamada.
    \begin{itemize}
      \item El compilador \textmark{puede ignorar} la sugerencia.
    \end{itemize}
\begin{lstlisting}
inline int cuadrado(int n) { 
  return n*n; 
}

int x = cuadrado(3); // Podría generar x = 9
int y = cuadrado(x); // Podria generar x = y*y
\end{lstlisting}

  \mode<presentation>{\vfill\pause}
  \item Permite generar código pequeño, rápido y mantenible.
    \begin{itemize}
      \item Preferible al uso de macros en C.
    \end{itemize}
\end{itemize}
\end{frame}

\begin{frame}[t,fragile]{Definiciones en-línea: Reglas}
\begin{itemize}
\item La \textgood{definición} de una función \cppkey{inline} 
      debe haberse visto \textmark{antes de su primer uso}.

\begin{lstlisting}
int cuadrado(int n);

x = cuadrado(2);

inline int cuadrado(int n) { return n*n; } // Error
\end{lstlisting}

\mode<presentation>{\vfill\pause}
\item Si varias unidades de traducción contienen la misma función \cppkey{inline}
      su definición debe ser idéntica.

\begin{lstlisting}
// f1.cpp
inline int cuadrado(int n) { return n*n; }


// f2.cpp
inline int cuadrado(int n) { return 2*n*n/2; }
\end{lstlisting}
\end{itemize}

\end{frame}

\begin{frame}[t,fragile]{Organización y funciones en línea}
\begin{itemize}
  \item Colocar las definiciones de funciones \cppkey{inline} en archivos de cabecera.
    \begin{itemize}
      \item Definición vista antes de uso.
      \item Misma definición vista por todas las unidades de traducción.
    \end{itemize}
\end{itemize}
\begin{block}{fecha.h}
\begin{lstlisting}
#ifndef CALENDARIO_FECHA_H
#define CALENDARIO_FECHA_H

// ...

inline bool es_bisiesto(int a) {
  return ((a % 4 == 0 && a % 100 != 0) || (a % 400 == 0));
}

// ...

#endif
\end{lstlisting}
\end{block}
\end{frame}

\begin{frame}[t,fragile]{Funciones miembro definidas dentro de clase}
\begin{itemize}
  \item Una función miembro \textmark{definida dentro de una clase} se considera 
        una función miembro \cppkey{inline}.
\begin{lstlisting}
class fecha {
public:
  // ...
  fecha() : dia{1}, mes{mes_id::enero}, anyo{2000} {}
  // ...
private:
  int dia;
  mes_id mes;
  int anyo;
};
\end{lstlisting}
\end{itemize}
\end{frame}

\begin{frame}[t,fragile]{Equivalencia}
\begin{itemize}
  \item Es \textmark{equivalente} a definir la función en línea 
        \textmark{justo después} de la definición de la clase.
\end{itemize}
\begin{lstlisting}
class fecha {
public:
  // ...
  fecha() : dia{1}, mes{mes_id::enero}, anyo{2000} {}
  // ...
private:
  int dia;
  mes_id mes;
  int anyo;
};

inline fecha::fecha() :
  dia{1},
  mes{mes_id::enero},
  anyo{2000}
{
}
\end{lstlisting}
\end{frame}

