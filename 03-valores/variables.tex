\section{Variables, valores y tipos}

\begin{frame}[t,fragile]{Variables, valores y objetos}
\begin{lstlisting}
int x = 65;
\end{lstlisting}
\begin{itemize}
  \item \textmark{Variable}: Nombre de un objeto en un programa.
    \begin{itemize}
      \item \cppid{x} es una variable.
    \end{itemize}

  \mode<presentation>{\vfill}
  \item \textmark{Valor}: Secuencia de bits que se interpreta de acuerdo con las reglas de un tipo.
    \begin{itemize}
      \item El valor \verb+00000000 00000000 00000000 01000001+ represetan el número \cppid{65} para el tipo \cppkey{int}.
      \item El valor \verb+01000001+ represetna el carácter \cppid{'A'} para el tipo \cppkey{char}.
    \end{itemize}

  \mode<presentation>{\vfill}
  \item \textmark{Objeto}: Secuencia de uno o más \emph{bytes} que contiene un valor de un tipo.
    \begin{itemize}
      \item Un objeto es lugar en la memoria en que se almacena un valor.
      \item Una variable permite referenciar un objeto.
        \mode<article>{
          \item \textmark{IMPORTANTE}: El concepto de objeto no tiene que ver necesariamente con la \emph{orientación a objetos}.
        }
    \end{itemize}
\end{itemize}
\end{frame}

\begin{frame}[t]{Tipo}
\begin{itemize}
  \item \textmark{Tipo}: Definición del conjunto de valores y operaciones admisibles para un objeto.
    \begin{itemize}

      \mode<presentation>{\vfill}
      \item \textmark{Tipos fundamentales}: Definidos por el lenguaje.
        \begin{itemize}
          \item \textmark{\emph{Booleanos}}: \cppkey{bool}.
          \item \textmark{Carácter}: \cppkey{char}, \cppkey{signed char}, \cppkey{unsigned char}, \cppkey{wchar\_t}, \cppkey{char16\_t}, \cppkey{char32\_t}.
          \item \textmark{Enteros}: \cppkey{short}, \cppkey{int}, \cppkey{long}, \cppkey{long long}.
            \mode<article>{
              \begin{itemize}
                \item Pueden llevar prefijo \cppkey{signed} o \cppkey{unsigned}.
                \item Si no se indica nada, se toma \cppkey{signed}.
              \end{itemize}
            }
          \item \textmark{Coma flotante}: \cppkey{float}, \cppkey{double}, \cppkey{long double}.
        \end{itemize}

      \mode<presentation>{\vfill}
      \item \textmark{Tipos predefinidos}: Definidos por al biblioteca estándar.
        \begin{itemize}
          \item Algunos ejemplos \cppid{string}, \cppid{vector}, \ldots
        \end{itemize}

      \mode<presentation>{\vfill}
      \item \textmark{Tipos definidos por el usuario}: Definidos por otra biblioteca o por el programa.
    \end{itemize}
\end{itemize}
\end{frame}

\begin{frame}[t,fragile]{\texttt{bool}}
\begin{itemize}
  \item Literales: \cppkey{false} y \cppkey{true}.
  \item Operaciones:
    \begin{itemize}
      \item Asignación.
\begin{lstlisting}
x = true;
\end{lstlisting}
      \item Comparación de igualdad.
\begin{lstlisting}
bool b1=true, b2=false;
x = {b1 == b2};
x = {b1 != b2};
\end{lstlisting}
      \item Comparación relacional
\begin{lstlisting}
bool b1=true, b2=false;
x = {b1 < b2};
x = {b1 <= b2};
x = {b1 > b2};
x = {b1 >= b2};
\end{lstlisting}
    \end{itemize}
\end{itemize}
\end{frame}

\begin{frame}[t,fragile]{Tipos carácter}
\begin{itemize}
  \item Literales entre comillas simples son de tipo \cppkey{char}.
    \begin{itemize}
      \item \cppid{'a'}, \cppid{'x'}, \ldots
      \mode<article>{
        \item También hay formatos para literales \cppkey{char16\_t}, \cppkey{char32\_t} y \cppkey{wchar\_t}.
      }
    \end{itemize}
  \item Secuencias de escape para caracteres especiales:
    \begin{itemize}
      \item Caracteres especiales: \cppkey{\textbackslash{}n}, \cppkey{\textbackslash{}r}, 
        \cppkey{\textbackslash{}t}, \cppkey{\textbackslash{}v},
        \cppkey{\textbackslash{}b}, \cppkey{\textbackslash{}f}, 
        \cppkey{\textbackslash{}a}.
      \item Secuencias de escape:
        \cppkey{\textbackslash{}\textbackslash}, 
        \cppkey{\textbackslash{}?}, 
        \cppkey{\textbackslash{}'}, 
        \cppkey{\textbackslash{}''}, 
    \end{itemize}
  \item Operaciones:
    \begin{itemize}
      \item Asignación.
\begin{lstlisting}
x = 'a';
\end{lstlisting}
      \item Comparaciones.
\begin{lstlisting}
bool b;
char c1 = 'a', c2 = 'b';
b = { c1 != c2};
b = { c1 >= c2};
\end{lstlisting}
    \end{itemize}
\end{itemize}
\end{frame}

\begin{frame}[t]{Tipos enteros}
\begin{itemize}
  \item \textgood{Literales}:
    \begin{itemize}

      \mode<presentation>{\vfill}
      \item \textmark{Base} en los literales:
        \begin{itemize}
          \item \cppid{1000} $\rightarrow$ Base 10.
          \item \cppid{01000} $\rightarrow$ Base 8.
          \item \cppid{0x1000} $\rightarrow$ Base 16.
        \end{itemize}

      \mode<presentation>{\vfill}
      \item \textmark{Tipo} en los literales:
        \begin{itemize}
          \item \cppid{1234} $\rightarrow$ \cppkey{int}.
          \item \cppid{1234u} $\rightarrow$ \cppkey{unsigned int}.
          \item \cppid{1234l} $\rightarrow$ \cppkey{long int}.
          \item \cppid{1234ul} $\rightarrow$ \cppkey{unsigned long int}.
          \item \cppid{1234ll} $\rightarrow$ \cppkey{long long int}.
          \item \cppid{1234ull} $\rightarrow$ \cppkey{unsigned long long int}.
        \end{itemize}
    \end{itemize}
\end{itemize}
\end{frame}

\begin{frame}[fragile,t]{Operaciones sobre tipos enteros}
\begin{itemize}
      \item \textmark{Asignación}: \cppkey{=}.
\begin{lstlisting}
x = 42;
\end{lstlisting}

      \mode<presentation>{\vfill}
      \item \textmark{Comparaciones}:
        \cppkey{==},
        \cppkey{!=},
        \cppkey{<},
        \cppkey{<=},
        \cppkey{>},
        \cppkey{>=}.

\begin{lstlisting}
int y=2, z=3;
bool b = {x > y};
\end{lstlisting}

      \mode<presentation>{\vfill}
      \item \textmark{Operaciones aritméticas}: 
        \cppkey{+}, \cppkey{-}, \cppkey{*}, \cppkey{/}, \cppkey{\%}.
\begin{lstlisting}
int y=2, z=3;
int x = y + z;
\end{lstlisting}
\end{itemize}
\end{frame}

\begin{frame}[fragile,t]{Operaciones sobre tipos enteros}
\begin{itemize}
  \item \textmark{Operaciones aritméticas con asignación}.
    \cppkey{+=}, \cppkey{-=}, \cppkey{*=}, \cppkey{/=}, \cppkey{\%=}.
\begin{lstlisting}
int y=2, z=3;
y *= z;
\end{lstlisting}

  \mode<presentation>{\vfill}
  \item \textmark{Incremento/decremento}: \cppkey{++}, \cppkey{-{}-}.
\begin{lstlisting}
int n = 10;
int x;
x = ++n;
x = n++;
x = --n;
x = n--;
\end{lstlisting}
\end{itemize}
\end{frame}


\begin{frame}[t,fragile]{Coma flotante}
\begin{itemize}
  \item \textgood{Literales}:
    \begin{itemize}
      \item \cppid{1.5} $\rightarrow$ \cppkey{double}.
      \item \cppid{1.5f} $\rightarrow$ \cppkey{float}.
      \item \cppid{1.5l} $\rightarrow$ \cppkey{long double}.
    \end{itemize}
  \item \textgood{Operaciones}:
\begin{columns}

\column{.5\textwidth}

\begin{itemize}
      \item \textmark{Asignación}.
\begin{lstlisting}
x = 1.5;
\end{lstlisting}
      \item \textmark{Comparaciones}:
        \cppkey{==},
        \cppkey{!=},
        \cppkey{<},
        \cppkey{<=},
        \cppkey{>},
        \cppkey{>=}.
      \item \textmark{Operaciones aritméticas}: 
        \cppkey{+}, \cppkey{-}, \cppkey{*}, \cppkey{/}.
\begin{lstlisting}
double y=2.5, z=1.5;
double x = y + z;
\end{lstlisting}
\end{itemize}

\column{.5\textwidth}

\begin{itemize}
      \item \textmark{Operaciones aritméticas con asignación}:
        \cppkey{+=}, \cppkey{-=}, \cppkey{*=}, \cppkey{/=}.
\begin{lstlisting}
double y=2.5, z=1.5;
y *= z;
\end{lstlisting}
      \item \textmark{Incremento/decremento}: \cppkey{++}, \cppkey{-{}-}.
\begin{lstlisting}
double x=1.5;
x++;
\end{lstlisting}
\end{itemize}

\end{columns}
\end{itemize}
\end{frame}

\begin{frame}[t,fragile]{Cadenas (\texttt{std::string})}
\begin{itemize}
  \item \textgood{Literales}:
    \begin{itemize}
      \item Entre comillas dobles: \cppid{"Hola"}
    \end{itemize}

  \mode<presentation>{\vfill}
  \item \textgood{Operaciones}:
    \begin{itemize}
      \item \textmark{Asignación}:
\begin{lstlisting}
string x;
x = "Hola";
\end{lstlisting}
      \item \textmark{Concatenación}:
\begin{lstlisting}
string n="Daniel", a="Garcia";
string nc = n + c;
string nc2 = n;
nc2 += c;
\end{lstlisting}
      \item \textmark{Comparaciones}:
        \cppkey{==},
        \cppkey{!=},
        \cppkey{<},
        \cppkey{<=},
        \cppkey{>},
        \cppkey{>=}.
    \end{itemize}
\end{itemize}
\end{frame}

