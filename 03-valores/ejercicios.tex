\section{Ejercicios}

\begin{enumerate}

\item El operador \cppkey{sizeof} determina el tamaño en bytes de una variable o de
un tipo de datos:

\begin{lstlisting}
int x;
auto sz1 = sizeof(x);
auto sz2 = sizeof(int);
\end{lstlisting}

Escribe un programa que imprima por pantalla el tamaño en bytes para cada uno de los
tipos fundamentales del lenguaje C++.

\item Todos los compiladores tienen un conjunto de advertencias (\emph{warnings})
      que se pueden activar, para realizar comprobaciones adicionales sobre el
      código. 

      Investiga el significado de los siguientes \emph{flags} en el
      compilador \textmark{gcc}:
\begin{itemize}
  \item \cppkey{-Wall}
  \item \cppkey{-Wextra}
  \item \cppkey{-Werror}
  \item \cppkey{-Wpedantic}
  \item \cppkey{-pedantic-errors}
  \item \cppkey{-Wconversion}
\end{itemize}

\item La herramienta \textmark{CMake} permite añadir opciones de compilación mediante
      \cppkey{add\_compile\_options}
      (consulta \url{https://cmake.org/cmake/help/latest/command/add_compile_options.html}).

      Crea un proyecto con un programa que lea de la entrada estándar un número
      real en doble precisión. Asigna los valores enteros inferiores y superiores
      a dos variables enteras (\cppid{alto} y \cppid{bajo}). Finalmente imprime
      los valores de dichas variables.

      \textmark{Nota}: El proyecto debe incluir las opciones de compilación mencionadas
      en el ejercicio anterior.

      Las siguientes funciones pueden ser útiles:
      \begin{itemize}
        \item \cppid{ceil()} 
              (consulta \url{https://en.cppreference.com/w/cpp/numeric/math/ceil}).
        \item \cppid{floor()} 
              (consulta \url{https://en.cppreference.com/w/cpp/numeric/math/floor}).
      \end{itemize}

      Resuelve los problemas ocasionados por advertencias del compilador.

\item Considera la siguiente cuestión:

      ¿Cómo se puede prevenir en la conversión de reales a enteros los casos que
      no son representables por el correspondiente tipo entero?
     

\end{enumerate}
