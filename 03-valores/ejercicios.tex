\section{Ejercicios}

\begin{enumerate}

\item El operador \cppkey{sizeof} determina el tamaño en bytes de una variable o de
un tipo de datos:

\begin{lstlisting}
int x;
auto sz1 = sizeof(x);
auto sz2 = sizeof(int);
\end{lstlisting}

Escribe un programa que imprima por pantalla el tamaño en bytes para cada uno de los
tipos fundamentales del lenguaje C++.

\item Escribe un programa que imprima las soluciones de una ecuación de segundo grado

\[
a x^2 + b x + c = 0
\]

El programa debe pedir los valores para los coeficientes \cppid{a}, \cppid{b} y
\cppid{c} e imprimir los resultados.

\textmark{Nota}: Para hallar la raíz cuadrada de un valor puedes usar la función
\cppid{std::sqrt()} (detalles en 
\url{https://en.cppreference.com/w/cpp/numeric/math/sqrt}).


\end{enumerate}
