\section{Entrada salida básica}

\begin{frame}[fragile]{Entrada/salida estándar}
\begin{itemize}
  \item Archivo de cabecera: \mode<presentation>{\cppid{<iostream>}}
    \mode<article>{
      \begin{itemize}
        \item Contiene todas las declaraciones de la entrada/salida estándar.
      \end{itemize}
    }
  \item Todas las declaraciones en espacio de nombres \cppid{std}.
  \item Objetos globales:
    \begin{itemize}
      \item \cppid{cin}: Entrada estándar.
      \item \cppid{cout}: Salida estándar.
      \item \cppid{cerr}: Salida de errores.
        \mode<presentation>{
          \begin{itemize}
            \item Va a la salida de errores sin usar un búfer intermedio.
          \end{itemize}
        }
      \item \cppid{clog}: Salida de log.
        \mode<presentation>{
          \begin{itemize}
            \item Va a la salida de errores usando un búfer intermedio.
          \end{itemize}
        }
    \end{itemize}
  \item Operadores:
    \begin{itemize}
      \item Operadores:
        \begin{itemize}
          \item Operador \cppid{<{}<} vuelca un dato en un flujo de salida.
\begin{lstlisting}
std::cout << x;
\end{lstlisting}
          \item Operador \cppid{>{}>} obtiene un dato de un flujo de entrada.
\begin{lstlisting}
std::cin >> x;
\end{lstlisting}
        \end{itemize}
    \end{itemize}
\end{itemize}
\end{frame}

\begin{frame}[t,fragile]{Ejemplo}
\begin{block}{main1.cpp}
\lstinputlisting{ejemplos/03-valores/hola-nombre/main1.cpp}
\end{block}
\begin{itemize}
  \item Declaraciones en espacio de nombres \cppid{std}.
    \mode<article>{
      \begin{itemize}
        \item Todos los símbolos de la biblioteca estándar están en este espacio de nombres.
      \end{itemize}
    }
  \item Cadena: Tipo de biblioteca $\rightarrow$ \cppid{string}.
    \mode<article>{
      \begin{itemize}
        \item Es una \emph{clase} definida en la biblioteca estándar.
        \item C++ también tiene el tipo \cppkey{char}\cppid{*}, pero \cppid{string}
          ofrece una interfaz más simple que se prefiere.
      \end{itemize}
    }
  \item \cppid{endl}/\cppstr{"\textbackslash{}n"}: Valor para fin de línea.
    \mode<article>{
      \begin{itemize}
        \item \cppid{endl} está definido en la biblioteca estándar como un valor
          que se traduce en un salto de línea.
        \item Además fuerza a vaciar el contenido del búfer asociado (\emph{flushing}).
          \begin{itemize}
            \item También se puede usar \cppid{"\textbackslash{}n"}, 
                  pero entonces no se vacía el búfer asociado.
          \end{itemize}
      \end{itemize}
    }
\end{itemize}
\end{frame}

\begin{frame}[t,fragile]{Ejemplo}
\begin{block}{main2.cpp}
\lstinputlisting{ejemplos/03-valores/hola-nombre/main2.cpp}
\end{block}
\begin{itemize}
  \item \cppkey{using namespace} evita repetición de cualificación de nombres.
  \item \textbad{Cuidado}: No uses \cppkey{using namespace} fuera de una función.
  \mode<article>{
    \begin{itemize}
      \item Puede dar lugar a situaciones difíciles de resolver.
    \end{itemize}
  }
\end{itemize}
\end{frame}
