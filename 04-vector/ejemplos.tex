\section{Ejemplos}

\begin{frame}[t]{Ejemplo: Estadísticas}
\begin{itemize}
  \item \alert{Objetivo}: Leer de la entrada estándar una secuencia de calificaciones
    y volcar en la salida estándar la calificación mínima, la máxima y la calificación media.
    \begin{itemize}
      \item Finalizar la lectura si se llega a fin de fichero.
      \item Finalizar la lectura si no se lee un valor correctamente (p. ej. letras en lungar de números).
      \item Se desconoce (y  no se pregunta) el número de valores.
    \end{itemize}
\end{itemize}
\end{frame}

\mode<presentation> {

\begin{frame}
\begin{block}{notas.cpp}
\lstinputlisting[lastline=16]{04-vector/vector/notas.cpp}
\ldots
\end{block}
\end{frame}

\begin{frame}
\begin{block}{notas.cpp}
\ldots
\lstinputlisting[firstline=17]{04-vector/vector/notas.cpp}
\end{block}
\end{frame}

}

\mode<article> {
\begin{frame}
\begin{block}{notas.cpp}
\lstinputlisting{04-vector/vector/notas.cpp}
\end{block}
\end{frame}
}

\begin{frame}[t]{Ejemplo: Palabras únicas}
\begin{itemize}
  \item \alert{Objetivo}: Volcar la lista ordenada de palabras únicas de un texto.
    \begin{itemize}
      \item El texto se lee de la entrada estándar hasta fin de fichero.
      \item La lista de palabras se imprime en la salida estándar.
    \end{itemize}
\end{itemize}
\end{frame}


\mode<presentation> {

\begin{frame}
\begin{block}{unique.cpp}
\lstinputlisting[lastline=15]{04-vector/vector/unique.cpp}
\ldots
\end{block}
\end{frame}

\begin{frame}
\begin{block}{unique.cpp}
\ldots
\lstinputlisting[firstline=16]{04-vector/vector/unique.cpp}
\end{block}
\end{frame}

}

\mode<article> {

\begin{frame}
\begin{block}{unique.cpp}
\lstinputlisting{04-vector/vector/unique.cpp}
\end{block}
\end{frame}

}
