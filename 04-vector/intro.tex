\section{Introducción}

\begin{frame}{Colecciones de valores}
\begin{itemize}
  \item \cppid{vector} permite almacenar y procesar un conjunto de valores
  de un mismo tipo.
    \mode<article>{
      \begin{itemize}
        \item C++ también tiene \emph{arrays} pero son demasiado limitados
          y simples. Se revisará su uso más adelante.
      \end{itemize}
    }
  \item Un \cppid{vector}:
    \begin{itemize}
      \item Tiene una secuencia de elementos.
      \item Se puede acceder a los elementos por su índice.
      \item Incluye información de su tamaño.
    \end{itemize}
\begin{tikzpicture}
\tikzset{
    bloque/.style={rectangle,draw=black, top color=white, bottom color=blue!50,
                   very thick, inner sep=0.5em, minimum size=0.6cm, text centered, font=\tiny},
    flecha/.style={->, >=latex', shorten >=1pt, thick},
    etiqueta/.style={text centered, font=\tiny} 
}  
\node[bloque] (bsize) {5};
\node[bloque,right=0cm of bsize] (bptr) { };
\node[bloque,below right=0.5cm and 0.75cm of bptr] (v0) {1};
\node[bloque,right=0cm of v0] (v1) {2};
\node[bloque,right=0cm of v1] (v2) {4};
\node[bloque,right=0cm of v2] (v3) {8};
\node[bloque,right=0cm of v3] (v4) {16};
\draw[flecha] (bptr) -- (v0);
\node[etiqueta, left=0.1cm of bsize] {v:};
\node[etiqueta, above=0cm of bsize] {size()};
\node[etiqueta, above=0cm of v0] {v[0]};
\node[etiqueta, above=0cm of v1] {v[1]};
\node[etiqueta, above=0cm of v2] {v[2]};
\node[etiqueta, above=0cm of v3] {v[3]};
\node[etiqueta, above=0cm of v4] {v[4]};
\end{tikzpicture}
  \item Alternativa a usar arrays directamente.
\end{itemize}
\end{frame}

\begin{frame}{Uso básico}
\begin{columns}[t]

\column{.5\textwidth}
\lstinputlisting{04-vector/vector/vec1.cpp}

\column{.5\textwidth}
\begin{itemize}
  \item Archivo de cabecera:
    \mode<presentation>{
      \cppid{<vector>}
    }
  \item Se debe indicar el tipo del elemento.
    \begin{itemize}
      \item Todos del mismo tipo.
    \end{itemize}
  \item Parámetro del constructor:
    \mode<presentation>{
      \alert{Tamaño inicial}.
    }
  \item No se puede acceder a indices más allá del tamaño.
\end{itemize}
\end{columns}
\end{frame}

\begin{frame}{Vectores y tipos}
\lstinputlisting{04-vector/vector/vec2.cpp}
\end{frame}

\mode<article>{
\begin{itemize}
  \item Cada vector debe indicar su tipo de elemento.
  \item Se realiza comprobación de tipos.
    \begin{itemize}
      \item No se pueden mezclar tipos.
    \end{itemize}
\end{itemize}
}

\begin{frame}[t,fragile]{Vectores e iniciación}
\begin{itemize}
  \item Un vector con tamaño inicia todos sus valores al valor por defecto del tipo.
    \begin{itemize}
      \item Valores numéricos: \mode<presentation>{\cppid{0}}
      \item Valores de cadena: \mode<presentation>{\cppid{``''}}
    \end{itemize}
  \item Si no se indica tamaño inicial, el vector tiene tamaño \cppid{0}.
\begin{lstlisting}
vector<double> v; // Vector con 0 elementos
\end{lstlisting}
  \item Se puede suministrar un valor inicial distinto.
\begin{lstlisting}
vector<double> v{100, 0.5}; // 100 posiciones iniciadas a 0.5
\end{lstlisting}
\end{itemize}
\end{frame}

\begin{frame}{Iniciación en la declaración}
\lstinputlisting{04-vector/vector/vec3.cpp}
\end{frame}

