\section{Recorrido}

\begin{frame}[fragile]{Recorrido de un vector}
\begin{itemize}
\item Se puede consultar el tamaño de un vector mediante la \emph{función miembro} \cppid{size}.
\begin{lstlisting}
cout << v.size();
\end{lstlisting}
\item \cppid{size()} permite definir un bucle para recorrer los elementos de un vector.
\begin{lstlisting}
for (int i=0; i<v.size(); ++i) {
  cout << "v[" << i << "] = " << v[i] << endl;
}
\end{lstlisting}
  \item \textmark{Importante}: Hay mejores maneras de recorrer un \cppid{vector}.
    \begin{itemize}
      \item Se verán a lo largo del curso.
    \end{itemize}
\end{itemize}
\end{frame}

\begin{frame}[t,fragile]{Recorrido basado en rango}
\begin{itemize}
  \item Se puede usar un recorrido basado en rango para un vector.
\begin{lstlisting}
vector<int> v1 { 1, 2, 3, 4 };
for (auto x : v1) {
  cout << x << endl;
}

vector<string> v2 { "Carlos", "Daniel", "José", "Manuel" };
for (auto x : v2) {
  cout << x << endl;
}
\end{lstlisting}
\end{itemize}
\end{frame}

