\section{Errores y excepciones}

\begin{frame}{Errores}
\begin{quote}
Cuiusvis hominis est errare, nullius nisi insipientis in errore perseverare.

\textbf{Marco Tulio Cicerón}
\end{quote}

\mode<presentation>{\vfill\pause}

\begin{quote}
Beware of bugs in the above code; I have only proved it correct, not tried it.

\textbf{Donald Knuth}
\end{quote}
\end{frame}

\begin{frame}[t]{Tipos de errores}
\begin{itemize}
  \item En \textmark{tiempo de desarrollo}.
    \begin{itemize}
      \item Detectados durante la fase de desarrollo de software.
      \item Más baratos de corregir.
      \item Soporte de herramientas para la detección.
    \end{itemize}

  \mode<presentation>{\vfill\pause}
  \item En \textmark{tiempo de ejecución}.
    \begin{itemize}
      \item Detectados durante la ejecución del software.
      \item Más caros de corregir.
      \item Necesidad de poder reproducirlos.
    \end{itemize}

  \mode<presentation>{\vfill\pause}
  \item \textgood{Objetivo final}: 
        Poder traducir el código a un programa que produzca los resultados deseados.
\end{itemize}
\end{frame}

\begin{frame}[t]{Errores en tiempo de desarrollo}
\begin{itemize}
  \item \textbad{Errores de compilación}: 
        Violación de las reglas del lenguaje que
        se puede detectar dentro de una unidad de traducción.

  \mode<presentation>{\vfill\pause}
  \item \textbad{Errores de enlace}: 
        Incosistencias entre distintas unidades de 
        traducción compiladas detectadas durante la fase de enlace.

  \mode<presentation>{\vfill\pause}
  \item \textbad{Otros errores detectables}:
        Errores y/o violaciones de reglas detectados por herramientas
        como \textmark{analizadores estáticos de código} y
        \textmark{analizadores dinámicos de código}.

  \mode<presentation>{\vfill\pause}
  \item \textbad{Errores durante el proceso de pruebas}:
        Errores detectados durante pruebas unitarias, de integración
        o de sistema.
\end{itemize}
\end{frame}

\begin{frame}[t]{Errores en tiempo de ejecución}
\begin{itemize}
  \item \textbad{Externos}: 
        Detectados por hardware, sistema operativo, software externo, \ldots
    \begin{itemize}
      \item División por cero.
      \item Acceso ilegal a memoria.
    \end{itemize}

  \mode<presentation>{\vfill\pause}
  \item \textbad{Biblioteca}: 
        Detectados por comprobaciones realizadas por la biblioteca.
    \begin{itemize}
      \item Error en reserva de memoria.
    \end{itemize}

  \mode<presentation>{\vfill\pause}
  \item \textbad{Internos}: 
        Detectados (o no) por código desarrollado.
    \begin{itemize}
      \item Acceso indebido a un dato.
    \end{itemize}
\end{itemize}
\end{frame}

\begin{frame}[t]{Tipos de errores}
\begin{itemize}
  \item \textmark{Especificaciones de baja calidad}:
        No está claro qué hacer en determinados casos.

  \mode<presentation>{\vfill\pause}
  \item \textmark{Programas incompletos}:
        Faltan casos por tener en cuenta.

  \mode<presentation>{\vfill\pause}
  \item \textmark{Argumentos no esperados}:
        Una función recibe un valor para el que no está preparada.

  \mode<presentation>{\vfill\pause}
  \item \textmark{Entradas no esperadas}:
        Se recibe un valor no esperado de un archivo o
        la interfaz de usuario.

  \mode<presentation>{\vfill\pause}
  \item \textmark{Estado inesperado}:
        El programa se encuentra en un estado en el que no puede 
        realizar una operación.

  \mode<presentation>{\vfill\pause}
  \item \textmark{Errores lógicos}:
        El programa no hace lo que se espera.
\end{itemize}
\end{frame}

\begin{frame}[t]{Lenguajes estáticos versus dinámicos}
\begin{itemize}
  \item \textgood{Lenguajes estáticos}: Comprobaciones realizadas en 
        \textmark{tiempo de traducción}.
    \begin{itemize}
      \item Se pueden detectar más errores antes de comenzar la ejecución.
      \item Pueden generar código con mejor rendimiento.
      \item Las comprobaciones presentan límites teóricos y prácticos.
    \end{itemize}

  \mode<presentation>{\vfill\pause}
  \item \textgood{Lenguajes dinámicos}: Comprobaciones realizadas en 
        \textmark{tiempo de ejecución}.
    \begin{itemize}
      \item Los errores se detectan al ejecutar programa.
      \item Mayor peligro de errores no detectados en producto final.
      \item Menor rendimiento derivado de más comprobaciones en tiempo de ejecución.
    \end{itemize}

  \mode<presentation>{\vfill\pause}
  \item La mayoría de los lenguajes presentan una 
        \textmark{combinación de las dos aproximaciones}.
\end{itemize}
\end{frame}

