\section{Depuración}

\begin{frame}[t]{Razones para depurar}
\begin{itemize}
  \item Hay diversas \textmark{razones} para \textgood{depurar} un programa:
    \begin{itemize}
      \item Se obtienen resultados no esperados.
      \item El programa termina de forma abrupta.
      \item Comprender mejor el funcionamiento del programa.
    \end{itemize}

  \mode<presentation>{\vfill\pause}
  \item \textgood{Alternativas} de depuración:
    \begin{itemize}
      \item Uso de un entorno de depuración.
      \item Inserción de sentencias de impresión en el código.
    \end{itemize}

  \mode<presentation>{\vfill\pause}
  \item \textmark{Recuerda}:
    \begin{itemize}
      \item La alternativa a usar depende mucho de las circunstancias.
      \item Es complicado depurar ciertas aplicaciones: 
            multi-hilo, fuentes de eventos externas, tiempo real.
    \end{itemize}
\end{itemize}
\end{frame}

\begin{frame}[t]{Herramientas de depuración}
\begin{itemize}
  \item Algunos ejemplos:
    \begin{itemize}
      \item gdb.
      \item Code::Blocks.
      \item DDD.
      \item Eclipse CDT.
      \item KDevelop.
      \item Nemiver.
      \item \textgood{CLion}.
    \end{itemize}
\end{itemize}
\end{frame}
