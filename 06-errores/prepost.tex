\section{Precondiciones y postcondiciones}

\subsection{Precondiciones}

\begin{frame}[t,fragile]{Precondiciones}
\begin{itemize}
  \item Una \textmark{precondición} es una condición que debe cumplirse
        antes de ejecutar una función.
\begin{lstlisting}
// pre: personas.size() == edades.size()
void imprime(vector<string> personas, vector<int> edades);
\end{lstlisting}

  \mode<presentation>{\vfill\pause}
  \item \textmark{Alternativas}:
    \begin{itemize}

      \item No realizar las comprobaciones.
        \begin{itemize}
          \item El llamante es responsable de cumplir la precondición.
        \end{itemize}

      \item Comprobar todas las precondiciones.
        \begin{itemize}
          \item La precondición se comprueba al entrar en la función.
        \end{itemize}
    \end{itemize}
\end{itemize}
\end{frame}

\begin{frame}[t,fragile]{Sin comprobación}
\begin{itemize}
  \item El llamante es responsable de cumplir la precondición.


  \mode<presentation>{\vfill}
  \item \textmark{Aspectos a considerar}:
    \begin{itemize}
      \item Puede dar lugar a errores si se invoca incumpliendo
            una precondición.
      \item No tiene coste computacional.
      \item Debe documentarse.
    \end{itemize}
\end{itemize}


\mode<presentation>{\vfill}
\begin{lstlisting}
// pre: x>=0
double sqrt(double x);
\end{lstlisting}
\end{frame}

\mode<presentation>{

\begin{frame}[t]{Ausencia de comprobación}
\begin{block}{nocheck.cpp}
\lstinputlisting[firstline=1,lastline=12]{ejemplos/06-errores/prepost/nocheck.cpp}
\ldots
\end{block}
\end{frame}

\begin{frame}[t]{Ausencia de comprobación}
\begin{block}{nocheck.cpp}
\ldots
\lstinputlisting[firstline=14]{ejemplos/06-errores/prepost/nocheck.cpp}
\end{block}
\end{frame}

}

\mode<article>{
\begin{frame}[t]{Ausencia de comprobración}
\begin{block}{nocheck.cpp}
\lstinputlisting{ejemplos/06-errores/prepost/nocheck.cpp}
\end{block}
\end{frame}
}

\begin{frame}[t,fragile]{Salida}
\begin{lstlisting}[style=terminal]
Carlos -> 10
Daniel -> 43
José -> 67
Manuel -> 98
Maria -> 0
Carlos -> 10
Daniel -> 43
José -> 67
Manuel -> 98
Maria -> 8
\end{lstlisting}
\begin{itemize}
  \item \textbad{Peligro}: El primer valor para María no está garantizado.
  \item Otras implementaciones podrían causar errores
\end{itemize}
\end{frame}


\begin{frame}[t,fragile]{Comprobación defensiva}
\begin{itemize}
  \item La precondición se comprueba al entrar en la función.

  \mode<presentation>{\vfill}
  \item \textmark{Aspectos a considerar}:
    \begin{itemize}
      \item Coste computacional asociado a las comprobaciones.
      \item No siempre es factible.
      \item Requiere una acción en respuesta al incumplimiento.
    \end{itemize}
\end{itemize}

\mode<presentation>{\vfill}
\begin{lstlisting}
// throws: invalid_index if i>=max_index
std::string find_name(int i);
\end{lstlisting}
\end{frame}

\mode<presentation>{

\begin{frame}[t]{Comprobración defensiva}
\mode<presentation>{\vspace{-1em}}
\begin{block}{check.cpp}
\lstinputlisting[lastline=15]{ejemplos/06-errores/prepost/check.cpp}
\ldots
\end{block}
\end{frame}

\begin{frame}[t]{Comprobación defensiva}
\begin{block}{check.cpp}
\ldots
\lstinputlisting[firstline=16,lastline=30]{ejemplos/06-errores/prepost/check.cpp}
\ldots
\end{block}
\end{frame}

\begin{frame}[t]{Comprobación defensiva}
\begin{block}{check.cpp}
\ldots
\lstinputlisting[firstline=31]{ejemplos/06-errores/prepost/check.cpp}
\end{block}
\end{frame}

}

\mode<article>{
\begin{frame}[t]{Comprobación defensiva}
\begin{block}{check.cpp}
\lstinputlisting{ejemplos/06-errores/prepost/check.cpp}
\end{block}
\end{frame}
}

\begin{frame}[t,fragile]{Salida}
\begin{lstlisting}[style=terminal]
Error: Distinta longitud
Carlos -> 10
Daniel -> 43
José -> 67
Manuel -> 98
Maria -> 8
\end{lstlisting}
\begin{itemize}
  \item La primera invocación genera una excepción porque las longitudes son distintas.
  \item Se añade un valor al \cppid{vector e}.
  \item La segunda invocación funciona correctamente.
\end{itemize}
\end{frame}

\begin{frame}[t]{Comparación de alternativas}
\begin{itemize}
  \item \textgood{Sin comprobaciones}:
    \begin{itemize}
      \item Traslada la responsabilidad al llamante.
      \item Código muy eficiente.
      \item Si se viola una precondición $\Rightarrow$ Comportamiento no definido.
    \end{itemize}

  \mode<presentation>{\vfill\pause}
  \item \textgood{Comprobación defensiva}:
    \begin{itemize}
      \item Responsabilidad del llamado.
      \item Sustituye la precondición por un comportamiento definido.
      \item Robustez a cambio de rendimiento.
      \item Suele ser difícil recuperarse.
    \end{itemize}

  \mode<presentation>{\vfill\pause}
  \item \textmark{Otro enfoque}:
    \begin{itemize}
      \item Biblioteca de programación por contratos:
            \url{https://github.com/jdgarciauc3m/min-contracts}
    \end{itemize}
\end{itemize}
\end{frame}

\begin{frame}[t]{Programación basada en contratos}
\begin{block}{check.cpp}
\lstinputlisting[firstline=7,lastline=14]{ejemplos/06-errores/prepost/contract_check.cpp}
\end{block}
\end{frame}

\subsection{Postcondiciones}

\begin{frame}[t,fragile]{Postcondiciones}
\begin{itemize}
  \item Una \textmark{postcondición} es una condición que debe cumplirse tras
        la ejecución de una función.

  \mode<presentation>{\pause\vfill}
  \item Pueden ser más complejas y costosas de comprobar.
\begin{lstlisting}
vector<int> copia_ordenado(vector<int> v);
// Post: Resultado es copia ordenada de v.
\end{lstlisting}

  \mode<presentation>{\pause\vfill}
  \item Es una buena práctica documentarlas.

  \mode<presentation>{\pause\vfill}
  \item El incumplimiento de una postcondición dado por:
    \begin{itemize}
      \item Incumplimiento de una precondición.
        \begin{itemize}
          \item No es una postcondición real.
        \end{itemize}
      \item Error de implementación.
        \begin{itemize}
          \item Recomendable comprobarlas.
        \end{itemize}
    \end{itemize}
\end{itemize}
\end{frame}

\begin{frame}[t]{Contratos: Postcondiciones}
\begin{block}{Postcondición de valor absoluto}
\lstinputlisting[firstline=16,lastline=20]{ejemplos/06-errores/prepost/contract_check.cpp}
\end{block}
\end{frame}

\subsection{Aserciones}

\begin{frame}[t,fragile]{Aserciones}
\begin{itemize}
  \item Una \textmark{aserción} es un predicado que debe cumplirse
        en un punto determinado del código.

  \mode<presentation>{\vfill\pause}
  \item Razones para utilizar \textmark{aserciones}:
    \begin{itemize}
      \item Puntos de control en algoritmos.
      \item Mayor eficiencia en postcondiciones complejas.
    \end{itemize}

  \mode<presentation>{\vfill\pause}
  \item Se puede usar la macro \cppid{assert()}.
    \begin{lstlisting}
    #include <cassert>
    \end{lstlisting}

  \item Se puede usar \cppid{CONTRACT\_ASSERT()}.
    \begin{lstlisting}
    #include <mincontracts/mincontracts.hpp>
    \end{lstlisting}
\end{itemize}
\end{frame}

\begin{frame}[t,fragile]{Aserciones}
\begin{block}{Usando Aserciones}
\ldots
\begin{lstlisting}
std::vector<double> v = get_vector();
for (auto x : v) {
  CONTRACT_ASSERT(x>0.0);
  procesa(x);
}
\end{lstlisting}
\ldots
\end{block}
\end{frame}
