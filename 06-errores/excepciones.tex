\section{Excepciones}

\begin{frame}[fragile]{Detección de errores}
\begin{itemize}
  \item Una función no puede siempre realizar su tarea.
  \item Alternativas:
    \begin{itemize}
      \item Comprobar antes de llamar.
      \item Comprobar dentro de la función.
        \begin{itemize}
          \item Realizar acción apropiada en caso de error.
          \item Devolver código de error.
          \item Usar excepciones.
        \end{itemize}
    \end{itemize}
\end{itemize}
\end{frame}

\begin{frame}[fragile]{Comprobar antes de llamar}
\begin{lstlisting}
void imprime_velocidad(double s, double t) {
  cout << s/t << endl;
}

void f() {
  double s = lee_espacio();
  double t = lee_tiempo();
  if (t > 0.0) {
    imprime_velocidad(s,t);
  }
  else {
    cerr << "Error: Tiempo negativo" << endl;
  }
}
\end{lstlisting}
\end{frame}

\begin{frame}[fragile]{Comprobar y realizar acción}
\begin{lstlisting}
void imprime_velocidad(double s, double t) {
  if (t > 0.0) {
    cout << s/t << endl;
  }
  else {
    cerr << "Error: Tiempo negativo" << endl;
  }
}

void f() {
  double s = lee_espacio();
  double t = lee_tiempo();
  imprime_velocidad(s,t);
}
\end{lstlisting}
\end{frame}

\begin{frame}[fragile]{Comprobar y devolver código de error}
\begin{lstlisting}
int imprime_velocidad(double s, double t) {
  if (t > 0.0) {
    cout << s/t << endl;
    return 0;
  }
  return -1;
}

void f() {
  double s = lee_espacio();
  double t = lee_tiempo();
  int r = imprime_velocidad(s,t);
  if (r<0) {
    cerr << "Error: Tiempo negativo" << endl;
  }
}
\end{lstlisting}
\end{frame}

\begin{frame}[fragile]{Excepciones}
\begin{itemize}
  \item \alert{IMPORTANTE}: El modelo de excepciones de C++ tiene diferencias con otros lenguajes.
  \item Resumen básico:
    \begin{itemize}
      \item Una excepción puede ser de cualquier tipo definido por el usuario.
\begin{lstlisting}
class tiempo_negativo {};
\end{lstlisting}
      \item Cuando una función detecta una situación excepcional lanza (\cppkey{throw}) una excepción.
\begin{lstlisting}
void imprime_velocidad(double s, double t) {
  if (t > 0.0) {
    cout << s/t << endl;
  }
  else {
    throw tiempo_negativo{};
}
\end{lstlisting}
    \end{itemize}
\end{itemize}
\end{frame}

\begin{frame}[fragile]{Tratamiento de excepciones}
\begin{itemize}
  \item Resumen básico:
    \begin{itemize}
      \item El llamante puede tratar la excepción con un bloque \cppkey{try}-\cppkey{catch}.
\begin{lstlisting}
void f() {
  double s = lee_espacio();
  double t = lee_tiempo();
  try {
    imprime_velocidad(s,t);
  }
  catch (tiempo_negativo) {
    cerr << "Error: Tiempo negativo" << endl;
  }
}
\end{lstlisting}
      \item No es obligatorio tratar una excepción.
\begin{lstlisting}
void f() {
  double s = lee_espacio(), t = lee_tiempo();
  imprime_velocidad(s,t);
}
\end{lstlisting}
    \end{itemize}
\end{itemize}
\end{frame}

\begin{frame}[fragile]{Excepciones estándar}
\begin{itemize}
  \item Existen varios tipos de excecpión predefinidos en la biblioteca estándar.
    \begin{itemize}
      \item \cppid{out\_of\_range}, \cppid{invalid\_argument}, \ldots
      \item Todos generalizados en el tipo \cppid{exception}.
      \item Todos tienen una operación \cppid{what()}.
    \end{itemize}
  \mode<article>{
  \item Se puden poner varias claúsulas \cppkey{catch}.
  }
\end{itemize}
\begin{lstlisting}
int main()
try {
  f();
  return 0;
}
catch (out_of_range & e) {
  cerr << "Out of range:" << e.what() << endl;
  return -1;
}
catch (exception & e) {
  cerr << "Excepción: " << e.what() << endl;
  return -2;
}
\end{lstlisting}
\end{frame}

\begin{frame}[fragile]{Captura completa}
\begin{itemize}
  \item Se puede incluir una claúsula que captura cualquier excepción.
    \begin{itemize}
      \item Incluso las no estándar.
    \end{itemize}
\end{itemize}
\begin{lstlisting}
int main()
try {
  f();
  return 0;
}
catch (exception & e) {
  cerr << "Excepción: " << e.what() << endl;
  return -1;
}
catch (...) {
  cerr << "Excepción no esperada" << endl;
  return -10;
}
\end{lstlisting}
\end{frame}
