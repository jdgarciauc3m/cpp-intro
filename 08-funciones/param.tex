\section{Paso de parámetros}

\subsection{Paso por valor}

\begin{frame}[fragile]{Paso por valor}
\begin{itemize}
  \item Se pasa a la función una \textmark{copia del argumento} 
        especificado en la llamada.
\begin{columns}[T]
\column{.3\textwidth}
\begin{lstlisting}
int incrementa(int n) {
  ++n;
  return n;
}
\end{lstlisting}

\column{.4\textwidth}
\begin{lstlisting}
void f() {
  int x = 5;
  int a = incrementa(x);
  int b = incrementa(x);
  // ...
}
\end{lstlisting}
\end{columns}

  \mode<presentation>{\vfill\pause}
  \item Se pueden pasar \textmark{literales} y \textmark{temporales} 
        como argumentos.
\begin{lstlisting}
void g() {
  int a = incrementa(3);
  int b = incrementa(a+2);
  // ...
}
\end{lstlisting}

\end{itemize}
\end{frame}

\subsection{Paso por referencia}

\begin{frame}[fragile]{Paso por referencia constante}
\begin{itemize}
  \item El paso por valor siempre realiza una copia.
    \begin{itemize}
      \item Si el objeto pasado es grande puede ser muy costoso.
    \end{itemize}
  \item \textgood{Paso por referencia constante}: 
        Pasa la \textmark{dirección del objeto} pero 
        \textbad{impide su alteración} dentro de la función.
    \begin{itemize}
      \item Conceptualmente equivalente a paso por valor.
    \end{itemize}
\begin{columns}[T]
\column{.58\textwidth}
\begin{lstlisting}
double maxref(const vector<doulbe> & v) {
  if (v.size() == 0) throw invalid_argument;
  double r = v[0];
  for (int i=1; i<v.size(); ++i) {
    if (r>v[i]) r=v[i];
  }
  return r;
}
\end{lstlisting}

\column{.42\textwidth}
\begin{lstlisting}
void f() {
  vector<double> vec(10000000);
  // ...
  double m = maxref(vec);
  // ...
}
\end{lstlisting}
\end{columns}
\end{itemize}
\end{frame}

\begin{frame}[fragile]{Paso por referencia}
\begin{itemize}
  \item \textbad{Elimina la restricción} de \textmark{no modificar} 
        el parámetro dentro de la función.
\begin{lstlisting}
void rellena(vector<int> & v, int n) {
  for (int i=0; i!=n; ++i) { v.push_back(i); }
}
\end{lstlisting}
  \item No se está pasando una copia.
    \begin{itemize}
      \item Se tiene acceso al propio objeto y se puede modificar.
    \end{itemize}
\begin{lstlisting}
int incrementa(int & n) {
  ++n;
  return n;
}

void f() {
  int x = 5;
  int a = incrementa(x);
  int b = incrementa(x);
  // ...
}
\end{lstlisting}
\end{itemize}
\end{frame}

\begin{frame}[fragile]{Ejemplo: intercambio de valores}
\begin{lstlisting}
void intercambia(int & x, int & y) {
  int tmp = x;
  x = y;
  y = tmp;
}

void f() {
  int a = 5, b=15;
  intercambia(a,b);
}
\end{lstlisting}
\begin{itemize}
  \item La biblioteca estańdar ofrece intercambio de datos para 
        todos los tipos predefinidos.
    \begin{itemize}
      \item \cppid{std::swap()}.
    \end{itemize}
\end{itemize}
\end{frame}
