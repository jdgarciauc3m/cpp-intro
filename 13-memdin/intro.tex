\section{Introducción}

\begin{frame}{Pila y montículo}
\begin{itemize}
  \item La mayor parte de las operaciones pueden realizarse con objetos alojados en la pila.
  \item Más sencillos de usar:
    \begin{itemize}
      \item Reglas claras sobre el alcance.
      \item No es necesario hacer uso del concepto de puntero.
    \end{itemize}
  \item Los objetos almacenados en el montículo ofrecen más flexibilidad.
    \begin{itemize}
      \item Gestión del tamaño en tiempo de ejecución.
      \item Uso más eficiente de la memoria.
      \item Coste adicional en tiempo de ejecución.
    \end{itemize}
  \item Algunos tipos (como \cppid{vector} o \cppid{string}) usan internamente el montículo.
\end{itemize}
\end{frame}

\begin{frame}[fragile]{Punteros}
\begin{itemize}
  \item Un \alert{puntero} es un objeto que almacena la dirección de otro objeto.
    \begin{itemize}
      \item Tiene asociado un tipo y solamente puede apuntar a objetos de ese tipo.
    \end{itemize}
\begin{lstlisting}
int * pi; // pi es un puntero a entero
char * pc; // pc es un puntero a carácter
\end{lstlisting}
  \item El operador \cppkey{\&} permite obtener la dirección de un objeto.
\begin{lstlisting}
int i = 42;
int * pi = &i;
char c = 'a';
char * pc = & c;
\end{lstlisting}
  \item El operador \cppkey{*} permite acceder al objeto apuntado por un puntero.
\begin{lstlisting}
int x = *pi; // x = 42
*pc = 'z'; // c = 'z'
\end{lstlisting}
\end{itemize}
\end{frame}
