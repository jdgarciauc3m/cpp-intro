\section{Punteros y referencias}

\begin{frame}[fragile]{Comparación: punteros/referencias}
\begin{itemize}
  \item Un puntero almacena la dirección de memoria de un objeto.
    \begin{itemize}
      \item Se puede modificar la dirección de memoria y el valor del objeto.
    \end{itemize}
\begin{lstlisting}
int * p = nullptr; 
p = f(); // Modificación de dirección
*p = 42; // Modificación de objeto.
\end{lstlisting}
  \item Una referencia mantiene un vínculo con una variable
    \begin{itemize}
      \item No se puede modificar la dirección de memoria, pero si el valor del objeto.
    \end{itemize}
\begin{lstlisting}
int & r = x; // r es una referencia a x
r = 12; // x == 12
\end{lstlisting}
\end{itemize}
\end{frame}

\begin{frame}[fragile]
\begin{itemize}
  \item Un puntero se puede declarar sin valor inicial, pero una referencia debe vincularse
        a un objeto al declararse.
\begin{lstlisting}
int x;
int * p;
p = &x; // p apunta a x
int & r = x;
int & s; // Error: Falta valor
\end{lstlisting}
  \item Una asignación a una referencia cambia el objeto.
  \item Una asignación a un puntero cambia la dirección.
\begin{lstlisting}
int x = 0;
int & r = x;
int * p = &x;
r = 42; // x == 42
p = &y; // p apunta ahora a y
\end{lstlisting}
\end{itemize}
\end{frame}
