\section{Arrays}

\subsection{Objetos de tipo array}

\begin{frame}[fragile]{Arrays fuera de memoria dinámica}
\begin{itemize}
  \item Como variables globales:
\begin{lstlisting}
constexpr int max1 = 100;
constexpr int max2 = 50;
double v1[max1];
double v2[max2];
double v3[25];
\end{lstlisting}
  \item Como variables locales:
\begin{lstlisting}
void f() {
  double v[10];
  // ...
}
\end{lstlisting}
  \item Diferencias con \cppid{std::vector}:
    \begin{itemize}
      \item El tamaño de un array debe ser conocido en tiempo de compilación.
      \item No se puede obtener el número de elementos de un array.
      \item Los arrays locales no se inician a ningún valor.
    \end{itemize}
\end{itemize}
\end{frame}

\begin{frame}[fragile]
\vspace{-0.5em}
\begin{itemize}
  \item No se puede definir un array usando como tamaño un parámetro de función.
\begin{lstlisting}
void f(int n) {
 double v[n]; // Error
 v[0] = 1.0;
 // ...
}
\end{lstlisting}
  \item Pero si se puede hacer con memoria dinámica
\begin{lstlisting}
void f(int n) {
 double * v = new double[n];
 v[0] = 1.0;
 // ...
 delete []v; // Si se olvida hay goteo
}
\end{lstlisting}
  \item O con \cppid{std::vector}:
\begin{lstlisting}
void f(int n) {
 vector<double> v(n);
 v[0] = 1.0;
 // ...
}
\end{lstlisting}
\end{itemize}
\end{frame}

\subsection{Iniciación de arrays}

\begin{frame}[fragile]{Iniciadores}
\vspace{-0.5em}
\begin{itemize}
  \item Un array puede iniciarse con una lista de valores.
\begin{lstlisting}
double v[4] = { 1.0, 2.0, 3.5, 4.7 };
char c[4] = { 'c', '+', '+', 0 };
\end{lstlisting}
  \item Si un array se declara con iniciador, se puede omitir el tamaño para
        que se calcule automáticamente.
\begin{lstlisting}
double v[] = {1.2, 2.4, 3.6 }; // Tamaño = 3
char c[] = { 'H', 'o', 'l', 'a' }; // Tamaño = 4
\end{lstlisting}
  \item Si se especifican menos iniciadores de los necesarios, se completa con \cppid{0}.
\begin{lstlisting}
double v[4] = {1.5, 2.5}; // v[2]==0.0, v[3]==0.0
\end{lstlisting}
  \item No se pueden especificar más iniciadores de los necesarios.
\begin{lstlisting}
char c[3] = { 'H', 'o', 'l', 'a' };
\end{lstlisting}
  \item Iniciación a 0.
\begin{lstlisting}
double v[10] = {};
\end{lstlisting}
\end{itemize}
\end{frame}

\begin{frame}[fragile]{Iniciación de cadenas}
\begin{itemize}
  \item Un literal de cadena equivale a un array de caracteres constantes con un carácter adicional para el terminador. 
\begin{lstlisting}
const char c1[5] = "Hola";
const char c2[5] = {'H', 'o', 'l', 'a', 0 };
\end{lstlisting}
  \item Se puede asignar un literal de cadena a un array no constante.
\begin{lstlisting}
char c[] = "Hola";
c[0] = 'C';
c[2] = 's';
\end{lstlisting}
  \item Se puede iniciar con un literal un puntero a carácter constante.
\begin{lstlisting}
const char * p = "Hola";
\end{lstlisting}
  \item Pero no un puntero a carácter no constante.
\begin{lstlisting}
char * p = "Hola"; // Error
\end{lstlisting}
\end{itemize}
\end{frame}

\begin{frame}[fragile]{Otros tipos de cadenas}
\begin{itemize}
  \item Cadenas en modo \emph{crudo}:
\begin{lstlisting}
cout << R"(\t")"; // Imprime '\', 't', '"'
\end{lstlisting}
  \item Cadenas UTF:
    \begin{itemize}
      \item Cadenas UTF-8.
        \begin{itemize}
          \item \cppid{u8"Hola"};
        \end{itemize}
      \item Cadenas UTF-16.
        \begin{itemize}
          \item \cppid{u"Hola"};
        \end{itemize}
      \item Cadenas UTF-32.
        \begin{itemize}
          \item \cppid{U"Hola"};
        \end{itemize}
    \end{itemize}
\end{itemize}
\end{frame}

\begin{frame}[fragile]{Iniciación de array como dato miembro}
\begin{itemize}
  \item Un dato miembro que sea de tipo array se puede iniciar con una lista de valores.
\begin{lstlisting}
class conversor {
private:
  double coef[4];
public:
  conversor(double x, double y, double z, double t) : coef{x,y,z,t} {}
  double convierte(double a) {
    return coef[0] * a + coef[1] * a + coef[2] * a + coef[3] * a;
  }
};

void f() {
  conversor c{1.0, 0.0, -1, 2};
  cout << c.convierte(3) << endl;
}
\end{lstlisting}
\end{itemize}
\end{frame}

\subsection{Arrays y aritmética de punteros}

\begin{frame}[fragile]{Punteros a posiciones de un array}
\begin{itemize}
  \item Un objeto de tipo array se puede usar como un puntero a la primera posición del array.
\begin{lstlisting}
double v[4];
*v = 1.5;
double x = v[0]; // x == 1.5
\end{lstlisting}
  \item Se puede iniciar un puntero a cualquier posición del array.
\begin{lstlisting}
double v[4];
double * p = &v[1];
p = &v[3];
\end{lstlisting}
  \item También se puede apuntar a la posición siguiente a la última.
    \begin{itemize}
      \item Pero no a posiciones anteriores o posteriores.
    \end{itemize}
\begin{lstlisting}
double v[4];
double * p = &v[-1]; // Comportamiento no definido
double * p = &v[7]; // Comportamiento no definido
\end{lstlisting}
\end{itemize}
\end{frame}

\begin{frame}[fragile]{Aritmética de punteros}
\begin{itemize}
  \item Se puede avanzar con un puntero en un array, sumando un entero al puntero.
\begin{lstlisting}
double v[4] = {}
double * p = v + 1; // Apunta a v[1]
p = p + 2; // Apunta a v[3];
++p; // Apunta a v[4]
\end{lstlisting}
  \item Igualmente se puede retroceder;
\begin{lstlisting}
--p; // Apunta a v[3]
p = p - 2; // Apunta a v[1]
p -=1; // Apunta a v[0]
\end{lstlisting}
  \item Se puede calcular la distancia entre dos punteros como número de elementos.
    \begin{itemize}
      \item Solamente si apuntan a objetos del mismo array.
    \end{itemize}
\begin{lstlisting}
double v[4] = {};
int n = &v[4] - &v[0]; // n == 4
int m = &v[4] - &w[2]; // No definido
\end{lstlisting}
\end{itemize}
\end{frame}

\begin{frame}[fragile]{Conversión}
\begin{itemize}
  \item Existe una conversión implícita del nombre de un array a puntero.
\begin{lstlisting}
double v[4] = {};
f(v); // Pasa a f el puntero &v[0]
\end{lstlisting}
  \item Esto impide el paso por valor de arrays a funciones.
    \begin{itemize}
      \item Objetivo: Evitar paso accidental por copia de arrays grandes.
    \end{itemize}
\begin{lstlisting}
double f(double vec[], int n) {
  double r{};
  for (int i = 0; i<n; ++i) {
    r += vec[i];
  }
  return r;
}
\end{lstlisting}
\end{itemize}
\end{frame}

\begin{frame}[t,fragile]{Conversión}
\begin{itemize}
  \item El resultado de la conversión es un valor, no una variable
\begin{lstlisting}
double v[4] = {};
v = new double[4]; // Error: v no es una variable punter
\end{lstlisting}
\end{itemize}
\end{frame}

\subsection{Copia de arrays}

\begin{frame}[fragile]{Copia de arrays}
\begin{itemize}
  \item La asignación de punteros copia direcciones.
\begin{lstlisting}
double v[4]={}, w[4]={};
double * p = v; // p apunta a v
\end{lstlisting}
  \item No se puede colocar el nombre de un array en la parte izquierda de una asignación.
\begin{lstlisting}
v = w; // Error: v no es una variable
\end{lstlisting}
  \item Alternativas:
    \begin{itemize}
      \item Copia explícita elemento a elemento.
\begin{lstlisting}
for (int i=0; i<4; ++i) v[i] = w[i];
\end{lstlisting}
      \item Copia de memoria.
\begin{lstlisting}
memcpy(v, w, 4 * sizeof(double)); // Cuidado
\end{lstlisting}
      \item Algoritmo de copia.
\begin{lstlisting}
std::copy_n(w, 4, v); // Preferida
\end{lstlisting}
    \end{itemize}
\end{itemize}
\end{frame}

\subsection{Paso de parámetros array}

\begin{frame}[fragile]{Cuestión de estilo}
\begin{itemize}
  \item Estilo alternativo a paso de dirección más tamaño:
\begin{lstlisting}
double suma(double * b, double * e) {
  double r{};
  while (b!=e) {
    r+= *b++;
  }
  return r;
}
\end{lstlisting}
  \item Invocación:
\begin{lstlisting}
void f() {
  double v[] = { 1, 2, 3, 4 };
  cout << suma(v, v+4) << endl;
}
\end{lstlisting}
  \item Enfoque: Paso de un rango semiabierto.
    \begin{itemize}
      \item Inicio: Puntero a primer elemento.
      \item Fin: Puntero al siguiente del último elemento.
    \end{itemize}
\end{itemize}
\end{frame}

\begin{frame}[fragile]{Constantes arrays y punteros}
\begin{itemize}
  \item El calificador \cppkey{const} indica la inmutabilidad de un objeto.
    \begin{itemize}
      \item Forma parte del tipo
    \end{itemize}
\begin{lstlisting}
const int max = 4; // max es const int
const double v[max] = { 1, 2, 3, 4 }; // v[0] es const double
\end{lstlisting}
  \item En el caso de un puntero realmente hay dos objetos:
    \begin{itemize}
      \item El puntero (dirección de memoria).
      \item El objeto apuntado.
    \end{itemize}
\end{itemize}
\end{frame}

\begin{frame}[fragile]
\begin{itemize}
  \item Puntero a valor constante:
\begin{lstlisting}
const double * p = v; // p apunta a v[0]
p = v + 2; // OK. Modificación del puntero
*p = 10; // Error: elementos constantes
\end{lstlisting}
  \item Puntero a valor constante (notación alternativa):
\begin{lstlisting}
double const * p = v; // p apunta a v[0]
p = v + 2; // OK. Modificación del puntero
*p = 10; // Error: elementos constantes
\end{lstlisting}
  \item Puntero constante a valor variable:
\begin{lstlisting}
double * const p = v; // p apunta a v[0]
p = v + 2; // Error no se puede apuntar a otro sitio
*p = 10; // OK: Modificación de objeto apuntado
\end{lstlisting}
  \item Puntero constante a valor constante:
\begin{lstlisting}
const double * const p = v; // p apunta a v[0]
p = v + 2; // Error no se puede apuntar a otro sitio
*p = 10; // OK: Modificación de objeto apuntado
\end{lstlisting}
\end{itemize}
\end{frame}

\begin{frame}[fragile]{Parámetros de función constantes}
\begin{itemize}
  \item \cppkey{const} forma parte del tipo $\rightarrow$ se puede usar para restringir las operaciones que se
        pueden realizar sobre un parámetro.
\begin{lstlisting}
void copia(char * dest, const char * orig) {
  while (*orig) {
    *dest++ = *orig++;
  }
}
\end{lstlisting}
\end{itemize}
\end{frame}

\subsection{Problemas con punteros}

\begin{frame}{Lista de problemas}
\begin{itemize}
  \item Indirección de un puntero nulo.
  \item Indirección de un puntero no iniciado.
  \item Indirección de un puntero al final de un rango (uno después del final).
  \item Indirección de un puntero desasignado.
  \item Indirección de un puntero a un objeto que ha salido de alcance.
\end{itemize}
\end{frame}
