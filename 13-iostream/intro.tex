\section{Introducción}

\begin{frame}{Visión general}
\begin{itemize}
  \item La entrada/salida se gestiona a varios niveles:
    \begin{enumerate}
      \item Dispostivo.
      \item Driver de dispositivo.
      \item Servicios del sistema operativo.
      \item Bibliotecas (sistema y \emph{run-time}).
      \item Aplicación.
    \end{enumerate}
  \item C++ ofrece una biblioteca propia de entrada/salida.
    \begin{itemize}
       \item Distinta de POSIX / Win32 / \ldots
       \item Entrada/salida portable.
    \end{itemize}
\end{itemize}
\end{frame}

\begin{frame}{Flujos de entrada y salida}
\begin{itemize}
  \item Un flujo representa una secuencia de caracteres que entran o salen
        de la aplicación.
  \item Tiene asociado un origen o destino: archivo, dispositivo, \ldots
  \item Dos tipos de flujos:
    \begin{itemize}
      \item \cppid{ostream}: Transforma valores de un tipo en secuencias de caracteres y los envía
            al destino asociado al flujo.
      \item \cppid{istream}: Transforma secuencias de caracteres procedentes de un origen
            en valores de un tipo.
    \end{itemize}
  \item Un flujo tiene internamente asociado un búfer.
\end{itemize}
\end{frame}


