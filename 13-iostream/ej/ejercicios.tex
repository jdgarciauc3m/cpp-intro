\section{Ejercicios}

\begin{enumerate}

\item Desarrolle un tipo de datos \cppid{opcion} que representa una opción financiera
      definida por los siguientes datos miembro:
  \begin{itemize}
    \item \cppid{tipo}: Puede ser \cppid{call} o \cppid{put}. Utilice un tipo enumerado.
    \item \cppid{prima}: Valor en doble precisión.
    \item \cppid{precio}: Valor en doble precisión.
  \end{itemize}

  Defina los operadores \cppkey{<{}<} y \cppkey{>{}>} sobre flujos para poder
  realizar lecturas y escrituras de archivos de texto.

  Defina la operación \cppid{write()} que escribirá en un archivo binario 
  (abierto previamente como un \cppid{ofstream}) la
  representación de una opción de la siguiente manera:
    \begin{itemize}
      \item \cppid{tipo} (1 byte): \cppstr{'C'} para \cppid{call} y 
            \cppstr{'P'} para \cppid{put}.
      \item \cppid{prima} (8 bytes): Representación binara del valor en doble precisión.
      \item \cppid{precio} (8 bytes): Representación binaria del valor en doble precisión
    \end{itemize}

  Defina la operación \cppid{read()} que leerá de un archivo binario
  (abierto previamente como un \cppid{ifstream}) la representación de una
  opción escrita con una operación \cppid{write()}.

\end{enumerate}
