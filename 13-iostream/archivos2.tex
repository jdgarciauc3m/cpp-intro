\section{Manipulación de archivos}

\begin{frame}[fragile]{Modos de apertura}
\begin{itemize}
  \item Apertura por defecto.
    \begin{itemize}
      \item \cppid{ifstream}: Abre para lectura.
      \item \cppid{ofstream}: Abre para escritura.
    \end{itemize}
  \item Modos de apertura:
    \begin{itemize}
      \item \cppid{ios\_base::app}: Abre para añadir al final del fichero.
      \item \cppid{ios\_base::ate}: Se posiciona al final del fichero.
      \item \cppid{ios\_base::binary}: Abre en modo binario.
      \item \cppid{ios\_base::in}: Abre para lectura.
      \item \cppid{ios\_base::out}: Abre para escritura.
      \item \cppid{ios\_base::trunc}: Abre y trunca el fichero (tamaño 0).
    \end{itemize}
  \item Los modos se pueden combinar:
\begin{lstlisting}
ofstream f1{"f1.txt", ios_base::app};
ofstream f2{"f2.txt", ios_base::in | ios_base::out};
\end{lstlisting}
\end{itemize}
\end{frame}


\begin{frame}[fragile]{Entrada/salida binaria}
\begin{itemize}
  \item Si se abre un archivo en modo binario no se usan los operadores \cppid{<{}<} y \cppid{>{}>}.
    \begin{itemize}
      \item \cppid{flujo.write(buf, nbytes)}.
      \item \cppid{flujo.read(buf, nbytes)}.
    \end{itemize}
  \item Es necesario convertir el dato que se desea operar a la dirección de una secuencia de bytes.
\begin{lstlisting}
template <class T>
char * como_bufer(T & x) {
  void * dir = &x;
  return static_cast<char*>(dir);
}
\end{lstlisting}
  \item Utilización:
\begin{lstlisting}
int x = 42;
flujo.write(como_bufer(x), sizeof(x));
\end{lstlisting}
\end{itemize}
\end{frame}

\begin{frame}[t]{Ejemplo}
\begin{itemize}
  \item Leer una secuencia de números de la entrada estándar, hasta fin de archivo.
  \item Guardar los números en formato binario de doble precisión en un archivo.
  \item El archivo binario tiene una cabecera indicando la cantidad de números que hay en el archivo.
\end{itemize}
\end{frame}

\begin{frame}[fragile]{Posiciones}
\begin{itemize}
  \item Un archivo tiene asociado dos punteros:
    \begin{itemize}
      \item \alert{get pointer}: Posición del siguiente carácter a leer.
      \item \alert{put pointer}: Posición del siguiente carácter a escribir.
    \end{itemize}
  \item Consulta de posiciones:
\begin{lstlisting}
n1 = flujo.tellg(); // Puntero de lectura
n2 = flujo.tellp(); // Puntero de escritura
\end{lstlisting}
  \item Posicionamiento absoluto
\begin{lstlisting}
flujo.seekp(n1); // Puntero de lectura
flujo.seekg(n2); // Puntero de escritura
\end{lstlisting}
\end{itemize}
\end{frame}

\begin{frame}[fragile]{Posiciones relativas}
\begin{itemize}
  \item Dirección de posicionamiento
    \begin{itemize}
      \item \cppid{ios\_base::beg}: Desde el principio.
      \item \cppid{ios\_base::cur}: Desde la posición actual.
      \item \cppid{ios\_base::end}: Desde el final.
    \end{itemize}
  \item Posicionamiento relativo:
\begin{lstlisting}
// Mueve el puntero de lectura 5 posiciones atrás 
flujo.seekg(-5, ios_base::cur);

// Mueve el punter de escritura al final
flujo.seekp(0, ios::base::end); 
\end{lstlisting}
\end{itemize}
\end{frame}
