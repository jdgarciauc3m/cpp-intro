\section{Punteros primitivos y \textbf{unique\_ptr}}

\begin{frame}[t,fragile]{Representación de \textbf{unique\_ptr}}
\begin{itemize}
  \item Un objeto de tipo \cppid{unique\_ptr} almacena internamente
        un \textemph{puntero primitivo} que apunta a
        la \textmark{memoria asignada}.

  \mode<presentation>{\vfill\pause}
  \item Se puede acceder al puntero primitivo con la operación \cppid{get()}
\begin{lstlisting}
auto p = std::make_unique<double>(42);
auto q = p.get(); // p es un double*

auto v = std::make_unique<int[]>(5);
auto w = v.get(); // v es un int*
\end{lstlisting}


  \mode<presentation>{\vfill\pause}
  \item \cppid{get()} da acceso al puntero pero la memoria 
        \textbad{se seguirá liberando} al destruirse el \cppid{unique\_ptr}.
\begin{lstlisting}
void print_value(int * p);
//...
auto q = std::make_unique<int>(42);
print_value(q.get());
//...
\end{lstlisting}

\end{itemize}
\end{frame}

\begin{frame}[t,fragile]{Liberación del puntero interno}
\begin{itemize}
  \item Se puede obtener el puntero interno liberando al \cppid{unique\_ptr} de
        su gestión.
    \begin{itemize}
      \item Al destruirse no se libera la memoria.
      \item Cualquier llamada a \cppid{get()} devuelve \cppkey{nullptr}
    \end{itemize}
\begin{lstlisting}
void f() {
  auto p = std::make_unique<int>(42);
  //...
  auto q = p.release(); // q es int*
  auto r = p.get(); // r es nullptr
} // Cuidado: goteo de memoria
\end{lstlisting}

\end{itemize}
\end{frame}

\begin{frame}[t,fragile]{Modificación del puntero interno}
\begin{itemize}
  \item Se puede modificar el puntero interno de un \cppid{unique\_ptr}
        con \cppid{p.reset(ptr)}.
\begin{lstlisting}
std::unique_ptr<int[]> v = new int[100];
//...
v.reset(new int[200]);
\end{lstlisting}

  \mode<presentation>{\vfill\pause}
  \item Cuando se invoca a \cppid{p.reset(ptr)}:
    \begin{enumerate}
      \item Se guarda el valor actual en un puntero auxiliar \cppid{aux}.
      \item Se actualiza el valor actual a \cppid{ptr}.
      \item Si el puntero auxiliar es no nulo, se libera la memoria asignada.
    \end{enumerate}
\end{itemize}
\end{frame}

\begin{frame}[t,fragile]{Transferencia de la propiedad}
\begin{itemize}
  \item Se puede combinar \cppid{release()} y \cppid{reset()} para 
        \textmark{transferir la propiedad} de un recurso.
\begin{lstlisting}
auto p = std::make_unique<int>(42);
auto q = std::make_unique<int>(99);
p.reset(q.release()); // libera valor apuntado por p

int x = *p; // x == 99
auto ptr = q.get(); // q == nullptr
\end{lstlisting}
\end{itemize}
\end{frame}
