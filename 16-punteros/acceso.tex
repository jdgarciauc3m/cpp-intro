\section{Acceso a memoria mediante punteros}

\begin{frame}[t,fragile]{Operadores de acceso}
\begin{itemize}
  \item \textgood{Indirecto}: Operador \cppkey{*} (desreferencia).
    \begin{itemize}
      \item Si se apunta a un valor, accede al valor apuntado.
\begin{lstlisting}
int * p = new int;
*p = 42;
int x = *p; // x = 42
\end{lstlisting}


      \mode<presentation>{\vfill\pause}
      \item Si se apunta a una secuencia de valores, accede al primer valor de la secuencia.
\begin{lstlisting}
int * v = new int[10];
*v = 42; // Equivale a v[0] = 42;
\end{lstlisting}
    \end{itemize}

  \mode<presentation>{\vfill\pause}
  \item \textgood{Relativo}: Operador \cppkey{[]} (indexación).
    \begin{itemize}
      \item Accede de forma indexada en una secuencia de valores.
\begin{lstlisting}
int * v = new int[10];
v[0] = 42;
v[1] = 84;
\end{lstlisting}
    \end{itemize}

\end{itemize}
\end{frame}

\begin{frame}[t,fragile]{Problemas de acceso}
\begin{itemize}
  \item Una variable de tipo puntero \textbad{no se inicia de forma automática} 
        a ningún valor.
    \begin{itemize}
      \item Si se desreferencia un \textmark{puntero no iniciado} se tiene un 
            \textbad{comportamiento no definido}.
    \end{itemize}
\begin{lstlisting}
int * p;
*p = 42; // Comportamiento no definido.
p[0] = 42; // Comportamiento no definido.
\end{lstlisting}

  \mode<presentation>{\vfill\pause}
  \item Una variable de tipo puntero iniciada a una secuencia 
        \textbad{solamente} puede accederse 
        \textmark{dentro de sus límites} establecidos.
\begin{lstlisting}
int * v = new int[10];
v[0] = 42; // OK
x = v[-1]; // No definido
x = v[15]; // No definido
v[10] = 0; // No definido
\end{lstlisting}
\end{itemize}
\end{frame}

\begin{frame}[t,fragile]{El puntero nulo}
\begin{itemize}
  \item Se puede iniciar un puntero al valor \textemph{puntero-nulo} para indicar 
        que \textbad{no apunta a ningún objeto}.
    \begin{itemize}
      \item Literal \cppkey{nullptr}.
    \end{itemize}
\begin{lstlisting}
int * p = nullptr;
char * q = nullptr;
if (p != nullptr) { /* ... */ }
if (q == nullptr) { /* ... */ }
\end{lstlisting}

  \mode<presentation>{\vfill\pause}
  \item Otras alternativas heredadas de C y C++98:
    \begin{itemize}
      \item Literal \cppid{0}.
      \item Macro predefinida \cppid{NULL}.
    \end{itemize}

  \mode<presentation>{\vfill\pause}
  \item Un puntero es \textmark{contextualmente convertible} a \textemph{booleano}.
    \begin{itemize}
      \item Si aparece en el contexto en que se espera un booleano.
    \end{itemize}
\begin{lstlisting}
if (p) { /* ... */ } // if (p != nullptr) 
if (!q) { /* ... */ } // if (q == nullptr)
\end{lstlisting}
\end{itemize}
\end{frame}

\begin{frame}[t,fragile]{Asignación de memoria e iniciación}
\begin{itemize}
  \item El operador \cppkey{new} \textbad{no inicia} el objeto asignado.
\begin{lstlisting}
int * p = new int;
x = *p; // x tiene un valor desconocido
\end{lstlisting}
    \begin{itemize}
      \item Se puede indicar el \textmark{valor inicial} entre llaves.
    \end{itemize}
\begin{lstlisting}
p = new int{42}; // *p == 42
p = new int{}; // *p == 0
\end{lstlisting}

  \mode<presentation>{\vfill\pause}
  \item Si se reserva una secuencia con \cppkey{new} 
        \textbad{no se inician} los objetos.
\begin{lstlisting}
int * v = new int[10];
\end{lstlisting}
    \begin{itemize}
      \item Se puede indicar los \textmark{valores iniciales} entre llaves.
    \end{itemize}
\begin{lstlisting}
v = new int[4]{1,2,3,4}; // v[0] = 1, v[1] = 2, v[2] = 3, v[3] = 4
v = new int[4]{1, 2};    // v[0] = 1, v[1] = 2, v[2] = 0, v[3] = 0
v = new int[4]{};        // v[0] = 0, v[1] = 0, v[2] = 0, v[3] = 0
v = new int[4]{1,2,3,4,5}; // Error demasiados iniciadores
\end{lstlisting}
\end{itemize}
\end{frame}

