\section{Punteros y referencias}

\begin{frame}[t,fragile]{Punteros frente a referencias}
\begin{itemize}
  \item Un \textgood{puntero} almacena la \textmark{dirección de memoria} de un objeto.
    \begin{itemize}
      \item Se puede \textgood{modificar} 
            la \textmark{dirección de memoria} y el \textmark{valor} del objeto.
    \end{itemize}
\begin{lstlisting}
int * p = nullptr; 
p = f(); // Modificación de dirección
*p = 42; // Modificación de objeto.
\end{lstlisting}

  \mode<presentation>{\vfill\pause}
  \item Una \textgood{referencia} mantiene un \textmark{vínculo} con una variable
    \begin{itemize}
      \item \textbad{No se puede modificar} la \textmark{dirección de memoria}, 
    \end{itemize}
\begin{lstlisting}
int & r = x; // r es una referencia a x
&r = y; // Error
\end{lstlisting}
      \item Pero \textgood{si se puede modificar} el \textmark{valor del objeto}.
\begin{lstlisting}
r = 12; // x == 12
\end{lstlisting}
\end{itemize}
\end{frame}

\begin{frame}[fragile]{Iniciación asignación}
\begin{itemize}
  \item Un puntero se puede declarar sin valor inicial, 
        pero una referencia \textbad{debe vincularse}
        a un objeto al declararse.
\begin{lstlisting}
int x;
int * p;
p = &x; // p apunta a x
int & r = x;
int & s; // Error: Falta valor
\end{lstlisting}

  \mode<presentation>{\vfill\pause}
  \item Una \textmark{asignación a una referencia} \textgood{cambia el objeto}, 
        pero una \textmark{asignación a un puntero} \textgood{cambia la dirección}.
\begin{lstlisting}
int x = 0;
int & r = x;
int * p = &x;
r = 42; // x == 42
p = &y; // p apunta ahora a y
\end{lstlisting}
\end{itemize}
\end{frame}
