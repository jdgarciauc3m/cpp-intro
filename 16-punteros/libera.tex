\section{Desasignación de memoria}

\begin{frame}[t,fragile]{Operador de desasignación de memoria}
\begin{itemize}
  \item El operador \cppkey{delete} permite \textgood{liberar memoria} 
        y \textgood{marcarla} como no asignada.
  \item Pude aplicarse solamente a:
    \begin{itemize}
      \item Memoria devuelta por el operador \cppkey{new} y asignada.
      \item El puntero nulo.
    \end{itemize}
\begin{lstlisting}
int * p = new int{10};
*p = 20;
delete p; // Libera p
\end{lstlisting}

  \mode<presentation>{\vfill\pause}
  \item Es un \textbad{error} invocar dos veces a \cppkey{delete} sobre un mismo puntero.
\begin{lstlisting}
int * p = new int{10};
delete p; // Libera p
delete p; // Comportamiento no definido
p = nullptr;
delete p; // OK
\end{lstlisting}
\end{itemize}
\end{frame}

\begin{frame}[t,fragile]{Desasignación de arrays}
\begin{itemize}
  \item Existe una versión diferente para liberar \emph{arrays}.
\begin{lstlisting}
int * p = new int{10};
int * v = new int[10];
delete p; // Libera p
delete []v;
\end{lstlisting}

  \mode<presentation>{\vfill\pause}
  \item \textmark{Importante}: Se debe usar la \textbad{versión correcta} de desasignación.
    \begin{itemize}
      \item Si se reserva memoria con \cppkey{new T} debe liberarse con \cppkey{delete}.
      \item Si se reserva memoria con \cppkey{new T[n]} debe liberarse con \cppkey{delete[]}.
    \end{itemize}
\begin{lstlisting}
int * p = new int{10};
int * v = new int[10];
delete []p; // Comportamiento no definido
delete v; // Comportamiento no definido
\end{lstlisting}
\end{itemize}
\end{frame}

\begin{frame}[t,fragile]{Razones para desasignar}
\begin{itemize}
  \item Si se reserva memoria y no se libera esta \textmark{queda asignada}.
\begin{lstlisting}
void f() {
  int * v = new int[1024*1024];
  // ...
}
\end{lstlisting}
    \begin{itemize}
      \item Cada vez que se invoca a \cppid{f()} \textbad{se pierden} 8 MB 
            (si \cppkey{sizeof(int)}\cppid{==8}).
    \end{itemize}

  \mode<presentation>{\vfill\pause}
  \item \textbad{Problemas} con los goteos de memoria:
    \begin{itemize}
      \item En cada asignación de memoria puede requerirse \textbad{más tiempo}.
      \item Si el programa se ejecuta durante mucho tiempo, 
            puede acabar \textbad{agotándose} la memoria.
    \end{itemize}

  \mode<presentation>{\vfill}
  \item Se se agota la memoria se lanza la excepción \cppid{bad\_alloc}.
\end{itemize}
\end{frame}

