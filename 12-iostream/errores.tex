\section{Errores}

\begin{frame}[fragile]{Estado}
\begin{itemize}
  \item Un flujo puede estar en cuatro estados:
    \begin{itemize}
      \item \cppid{good}: La última operación tuvo éxito.
      \item \cppid{eof}: Se ha llegado a fin de archivo.
      \item \cppid{fail}: Error recuperable.
        \begin{itemize}
          \item \alert{Ejemplo}: Error de formato.
        \end{itemize}
      \item \cppid{bad}: Error severo.
        \begin{itemize}
          \item \alert{Ejemplo}: Error de disco.
        \end{itemize}
    \end{itemize}
  \item Funciones miembro de consulta: \cppid{good()}, \cppid{eof()}, \cppid{fail()}, \cppid{bad()}.
  \item Se puede restablecer el estado a \cppid{good}.
\begin{lstlisting}
flujo.clear(); // Estado pasa a good.
\end{lstlisting}
\end{itemize}
\end{frame}

\begin{frame}{Ejemplo}
\begin{itemize}
  \item Leer números enteros de un archivo.
  \item Si se encuentran caracteres no numéricos se ignoran y se sigue leyendo.
  \item Presentar un número por línea en la salida estándar.
\end{itemize}
\end{frame}

\begin{frame}
\begin{block}{main.cpp}
\lstinputlisting[firstline=15,lastline=36]{12-iostream/error/main.cpp}
\end{block}
\end{frame}

\begin{frame}[fragile]{Excecpción de entrada/salida}
\begin{itemize}
  \item La biblioteca de E/S no lanza normalmente excecpiones.
  \item Se puede configurar un flujo para que lance excepciones al cambiar de estado.
\begin{lstlisting}
flujo.exceptions(ios_base::badbit); // Lanza si pasa a bad
flujo.exceptions(ios_base::failbit); // Lanza si pasa a fail
flujo.exceptions(ios_base::badbit | ios_base::failbit); // Lanza si pasa a bad o fail
\end{lstlisting}
  \item En caso de producirse fallo se lanza \cppid{ios\_base::exception}.
  \item Se puede obtener la máscara de excepciones actual.
\begin{lstlisting}
auto m = flujo.exceptions();
m |= ios_base::failbit;
flujo.exceptions(m);
\end{lstlisting}
\end{itemize}
\end{frame}

\begin{frame}{Uso de excepciones en entrada/salida}
\begin{block}{main.cpp}
\lstinputlisting{12-iostream/except/main.cpp}
\end{block}
\end{frame}
