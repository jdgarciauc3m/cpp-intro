\section{Introducción}

\begin{frame}[t]{Pila y montículo}
\begin{itemize}
  \item La mayor parte de las operaciones pueden realizarse con objetos 
        \textmark{alojados en la pila}.
  \begin{itemize}
    \item \textgood{Más sencillos} de usar:
      \begin{itemize}
        \item Reglas claras sobre el alcance.
        \item No es necesario hacer uso del concepto de puntero.
      \end{itemize}
  \end{itemize}

  \mode<presentation>{\vfill\pause}
  \item Los objetos almacenados en el montículo ofrecen \textgood{más flexibilidad}.
    \begin{itemize}
      \item Gestión del tamaño en tiempo de ejecución.
      \item Uso más eficiente de la memoria.
      \item Coste adicional en tiempo de ejecución.
    \end{itemize}

  \mode<presentation>{\vfill\pause}
  \item Algunos tipos (como \cppid{vector} o \cppid{string}) usan internamente el montículo.
\end{itemize}
\end{frame}

\begin{frame}[t,fragile]{Punteros únicos}
\begin{itemize}
  \item Un \textmark{puntero único} (\cppid{unique\_ptr}) almacena la dirección
        donde se encuentra un objeto alojado en memoria dinámica.
    \begin{itemize}
      \item Tiene asociado un tipo y solamente puede apuntar a objetos de ese tipo.
    \end{itemize}
\end{itemize}
\begin{lstlisting}
auto pj = std::make_unique<int>(5); // pj apunta a un entero con valor 5
                                    // pj de tipo std::unique_ptr<int>
\end{lstlisting}

\mode<presentation>{\vfill\pause}
\begin{itemize}
  \item El operador \cppkey{*} permite acceder al objeto apuntado por un 
        \cppid{unique\_ptr}.
\end{itemize}
\begin{lstlisting}
auto pi = std::make_unique<int>(5);
int x = *pi; // x = 5

auto pc = std::make_unique<char>('a');
*pc = 'z'; 
char c = *pc; // c = 'z'
\end{lstlisting}
\end{frame}

\begin{frame}[t,fragile]{Punteros únicos y tipos no primitivos}
\begin{itemize}
  \item La función \cppid{make\_unique()} toma cualquier número de argumentos
        y los pasa al constructor correspondiente.
\begin{lstlisting}
class punto {
public:
  punto(double x, double y);
  //...
};

void f() {
  auto p = std::make_unique<punto>(1.0, 5.0); // Crea un punto{1.0,5.0}
  //...
}
\end{lstlisting}
\end{itemize}
\end{frame}
