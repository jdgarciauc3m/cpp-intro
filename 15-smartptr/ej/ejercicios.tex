\section{Ejercicios}

\begin{enumerate}

\item
Escribe un programa que lea una lista de números reales. El programa debe:

\begin{itemize}
  \item Imprimir el valor medio.
  \item Imprimir el valor absoluto de la diferencia entre cada valor y el promedio.
  \item Imprimir la desviación típica.
\end{itemize}

No utilice el tipo \cppid{std::vector}. En vez de esto utilice la siguiente estrategia:

\begin{itemize}
  \item Comience con un \cppid{std::unique\_ptr<double[]>} a una secuencia de 10 valores
        (variable \cppid{lista}) y mantenga dos variables enteras separadas 
        \cppid{capacidad} (inicialmente con el valor \cppid{10}) y \cppid{numvals} 
        (inicialmente con el valor \cppid{0}).

  \item Cada vez que tenga que añadir un número a la lista, colóquelo en la posición
        \cppid{lista[numvals]} en incremente \cppid{numvals}.

  \item Si después de añadir un número a la lista \cppid{numvals} ha alcanzado el valor
        de \cppid{capacidad} es hora para conseguir una \cppid{lista} más grande:
    \begin{enumerate}
      \item Aumente \cppid{capacidad} al doble de su valor actual.
      \item Reserve una nueva lista \cppid{nueva\_lista} con el tamaño definido por el 
            nuevo valor de \cppid{capacidad}.
      \item Copie todos los valores de la lista antigua a la lista nueva.
      \item Esta será la lista a utilizar a partir de ahora transfiera la memoria de
            \cppid{nueva\_lista} a \cppid{lista}.
    \end{enumerate}
\end{itemize}

\end{enumerate}
