\section{Asignación de memoria}

\begin{frame}[t]{Memoria del almacén libre}
\begin{itemize}
  \item El \textmark{almacén libre} contiene la memoria libre que un programa
        puede adquirir y liberar.
    \begin{itemize}
      \item Típicamente implementado mediante el \textgood{montículo} (\emph{heap}).
    \end{itemize}

  \mode<presentation>{\vfill\pause}
  \item \cppid{unique\_ptr} gestiona automáticamente un recurso de memoria
        dinámica.
    \begin{itemize}
      \item Se construye con el resultado de \cppid{std::make\_unique()}.
      \item Cuando se sale del alcance el objeto se destruye y se libera la memoria.
    \end{itemize}

  \mode<presentation>{\vfill}
  \item \textbad{IMPORTANTE}: Esto es distinto de la \textmark{recolección de basura}.
\end{itemize}
\end{frame}

\begin{frame}[t,fragile]{Función de reserva: \textbf{std::make\_unique()}}
\begin{itemize}
  \item La función \cppid{std::make\_unique()} permite asignar memoria del 
        almacén libre.
\end{itemize}
\begin{lstlisting}
auto p = std::make_unique<int>(); // Asigna memoria para un int iniciado a 0
auto q = std::make_unique<int>(5); // Asigna memoria para un int iniciado a 5
auto r = std::make_unique<char[]>(10); // Asigna memoria para 10 char
\end{lstlisting}

\begin{itemize}
  \mode<presentation>{\vfill\pause}
  \item \textgood{Efecto}:
    \begin{itemize}
      \item \cppid{std::make\_unique()} reserva memoria y devuelve 
            un \cppid{std::unique\_ptr}.
    \end{itemize}

  \mode<presentation>{\vfill\pause}
  \item \textbad{IMPORTANTE}: No se puede saber el tamaño de la memoria asignada 
        después de construir un \cppkey{std::unique\_ptr}.
    \begin{itemize}
      \item Podría ser mayor que el tamaño del objeto.
    \end{itemize}
\end{itemize}
\end{frame}
