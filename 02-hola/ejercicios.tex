\section{Ejercicios}

\begin{enumerate}

\item Si no tienes acceso a un sistema Linux, instala una versión reciente de Linux.
Se recomienda utilizar, por ejemplo, Ubuntu Linux. Sigue los pasos en el tutorial
suministrado por Ubuntu en la siguiente URL:

  \begin{itemize}
    \item \url{https://ubuntu.com/tutorials/install-ubuntu-desktop#1-overview}.
  \end{itemize}

\item Asegúrate de instalar la versión 10 de \cppid{g++}:

\begin{lstlisting}[style=terminal]
sudo add-apt-repository ppa:ubuntu-toolchain-r/test
sudo apt-get update
sudo apt install gcc-10
sudo apt install g++-10
\end{lstlisting}

\item Edita el programa en un archivo con el nombre \cppid{hola.cpp}. Desde la 
línea de comandos ejecute la siguiente lína para compilar el programa:

\begin{lstlisting}[style=terminal]
g++ hola.cpp -o hola
\end{lstlisting}

Ejecuta el programa resultante.

\item Obten una licencia para estudiantes del entorno de desarrollo \textmark{CLion}
en la siguiente URL:
\begin{itemize}
  \item \url{https://www.jetbrains.com/es-es/community/education/#students}
\end{itemize}

\item Ejecuta el entorno de desarrollo \textmark{CLion} y sigue los siguiente pasos:

\begin{enumerate}
\item Crea un proyecto con las siguientes características:
\begin{itemize}
  \item \textmark{Project type}: C++ Executable.
  \item \textmark{Language standard}: C++20.
\end{itemize}

\item Comprueba que se generan los siguientes archivos:
\begin{itemize}
  \item \cppid{main.cpp}.
  \item \cppid{CMakeLists.txt}
\end{itemize}

\item Construye el programa mediante la opción \textmark{Build | Build Project}. 
Comprueba que el programa se construye correctamente.

\item Ejecuta el programa (\textmark{Run | Run 'hola'}).
Comprueba que el programa se ejecuta correctmente. La salida debería ser:

\begin{lstlisting}[style=terminal]
Hello, World!

Process finished with exit code 0
\end{lstlisting}

\end{enumerate}

\end{enumerate}
