\section{Tipos definidos por el usuario}

\begin{frame}[t]{Tipos}
\begin{itemize}
  \item Cada objeto manipulado tiene un tipo asociado.
  \item Un \alert{tipo} tiene asociada:
    \begin{itemize}
      \item Una \alert{representación} de los datos necesarios en la memoria del computador.
      \item Un conjunto de \alert{operaciones} que pueden realizarse sobre un objeto.
    \end{itemize}
  \item Los tipos pueden ser:
    \begin{itemize}
      \item Primitivos: \cppkey{int}, \cppkey{char}, \ldots
      \item Definidos por la biblioteca: \cppid{string}, \cppid{vector}, \ldots
      \item Definidos por el usuario: \cppid{punto}, \cppid{fecha}, \ldots
    \end{itemize}
\end{itemize}
\end{frame}

\begin{frame}[t,fragile]{Definición de tipos}
\begin{itemize}
  \item Dos mecanismos básicos de definición de nuevos tipos en C++:
    \begin{itemize}
      \item \alert{Clase}: Define un tipo mediante su representación y conjunto de operaciones.
\begin{lstlisting}
class punto {
  punto(double cx, double cy);
  // ...
private:
  double x, y;
};
\end{lstlisting}
      \item \alert{Enumerados}: Define un tipo como un conjunto de valores admisibles.
\begin{lstlisting}
enum color { rojo, verde, azul };
\end{lstlisting}
    \end{itemize}
\end{itemize}
\end{frame}
