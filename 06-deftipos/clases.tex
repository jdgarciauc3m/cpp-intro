\section{Clases}

\begin{frame}[fragile]{¿Qué es una clase?}
\begin{itemize}
  \item Tipo definido por el usuario.
    \begin{itemize}
      \item \alert{Representación} $\rightarrow$ \textbf{Datos miembro}.
      \item \alert{Operaciones} $\rightarrow$ \textbf{Funciones miembro}
    \end{itemize}
  \item Construcciones del lenguaje:
    \begin{itemize}
      \item Estructura.
\begin{lstlisting}
struct fecha {
  // Miembros
};
\end{lstlisting}
      \item Clase
\begin{lstlisting}
class fecha {
  // Miembros
};
\end{lstlisting}
    \end{itemize}
  \item Alternativas:
    \begin{itemize}
      \item Estructura + Funciones sobre estructura.
      \item Estructura con funciones miembro.
      \item Clase.    
    \end{itemize}
\end{itemize}
\end{frame}

\begin{frame}{Estructura y funciones sobre estructura}
\vspace{-.8em}
\begin{block}{fecha.h}
\lstinputlisting{06-deftipos/fecha1/fecha.h}
\end{block}
\begin{block}{main.cpp}
\lstinputlisting[firstline=6]{06-deftipos/fecha1/main.cpp}
\end{block}
\end{frame}

\vspace{-.25em}
\begin{frame}
\begin{block}{fecha.cpp}
\mode<presentation>{
\lstinputlisting[lastline=18]{06-deftipos/fecha1/fecha.cpp}
}
\mode<article>{
\lstinputlisting{06-deftipos/fecha1/fecha.cpp}
}
\end{block}
\end{frame}

\begin{frame}[fragile]{Función miembro}
\begin{itemize}
  \item Es una función que puede invocarse solamente para una variable del tipo definido.
    \begin{itemize}
      \item Recibe implícitamente un parámetro del tipo definido.
      \item Puede acceder a los datos miembro de la variable.
      \item Se invoca usando la notación de acceso.
        \begin{itemize}
          \item \cppid{variable.funcion(param)}.
        \end{itemize}
    \end{itemize}
\begin{lstlisting}
struct punto {
  double x, y;
  double modulo();
};

void f() {
  punto p { 2.0, 3.5 };
  double x = p.modulo(); // Invocación
  // ...
}
\end{lstlisting}
\end{itemize}
\end{frame}

\begin{frame}{Estructura con fucniones miembro}
\begin{block}{fecha.h}
\lstinputlisting{06-deftipos/fecha2/fecha.h}
\end{block}
\begin{block}{main.cpp}
\lstinputlisting[firstline=6]{06-deftipos/fecha2/main.cpp}
\end{block}
\end{frame}

\begin{frame}
\begin{block}{fecha.cpp}
\mode<presentation>{
\lstinputlisting[lastline=18]{06-deftipos/fecha2/fecha.cpp}
}
\mode<article>{
\lstinputlisting{06-deftipos/fecha2/fecha.cpp}
}
\end{block}
\end{frame}

\begin{frame}[t,fragile]{Visibilidad}
\vspace{-0.25cm}
\begin{itemize}
  \item Los miembros de una estructura (o una clase) tienen asociado un nivel de visibilidad.
    \begin{itemize}
      \item \cppkey{public}: El miembro puede ser accedido por cualquiera.
      \item \cppkey{private}: El miembro puede ser accedido solamente por otros miembros.
      \item \cppkey{protected}: Relacionado con herencia.
    \end{itemize}
  \item Visibilidad por defecto:
    \begin{itemize}
      \item Los miembros de una estructura son públicos por defecto.
      \item Los miembros de una clase son privados por defecto.
    \end{itemize}
\end{itemize}
\vspace{-0.25cm}
\begin{lstlisting}
class X {
  int x; // x es privado
  double y; // y es privado
public:
  void f(int n); // f es pública
  void g(double x); // g es pública
private:
  void h(); // h es privada
};
\end{lstlisting}
\end{frame}

\begin{frame}[fragile]{Interfaz e implementación}
\begin{itemize}
  \item Los niveles de visiblidad permiten especificar y separar:
    \begin{itemize}
      \item La interfaz pública de la clase.
      \item Los detalles privados de implementación.
    \end{itemize}
\end{itemize}
\begin{lstlisting}
class ejemplo {
public:
  // Miembros públicos
  // Funciones, tipos y datos
private:
  // Miembros privados
  // Funciones, tipos y datos
};
\end{lstlisting}
\end{frame}

\begin{frame}{Una clase fecha}
\begin{block}{fecha.h}
\lstinputlisting{06-deftipos/fecha3/fecha.h}
\end{block}
\end{frame}

\begin{frame}
\begin{block}{main.cpp}
\lstinputlisting{06-deftipos/fecha3/main.cpp}
\end{block}
\end{frame}

\begin{frame}
\begin{block}{fecha.cpp}
\lstinputlisting[lastline=19]{06-deftipos/fecha3/fecha.cpp}
\end{block}
\end{frame}

\begin{frame}
\begin{block}{fecha.cpp}
\lstinputlisting[firstline=21,lastline=39]{06-deftipos/fecha3/fecha.cpp}
\end{block}
\end{frame}

\begin{frame}
\begin{block}{fecha.cpp}
\lstinputlisting[firstline=40,lastline=47]{06-deftipos/fecha3/fecha.cpp}
\end{block}
\end{frame}

\begin{frame}
\begin{block}{fecha.cpp}
\lstinputlisting[firstline=49]{06-deftipos/fecha3/fecha.cpp}
\end{block}
\end{frame}

\begin{frame}[fragile]{Constructor}
\begin{itemize}
  \item Un \alert{constructor} es una función miembro especial.
    \begin{itemize}
      \item Se usa para iniciar objetos del tipo definido por la clase.
      \item La sintaxis refuerza la invocación del constructor.
        \begin{itemize}
          \item Es obligatorio pasar agumentos del constructor al declarar una variable.
        \end{itemize}
      \item Es una función miembro con el nombre de la clase y sin tipo de retorno.
    \end{itemize}
\begin{lstlisting}
class punto {
public:
  punto(double cx, double cy);
  // ...
private:
  double x;
  double y;
};
\end{lstlisting}
\end{itemize}
\end{frame}

\begin{frame}[fragile]{Invocación al constructor}
\begin{itemize}
  \item Dos notaciones para invocar al constructor.
\begin{lstlisting}
punto p1(2.3, 3.5); // Clásica
punto p2{2.3, 3.5}; // C++11
\end{lstlisting}
  \item Si una clase tiene constructor se debe suministrar argumentos.
\begin{lstlisting}
punto p; // Error faltan argumentos
\end{lstlisting}
  \item Se puede iniciar copiando otro objeto.
\begin{lstlisting}
punto p1{2.5, 3.5};
punto p2 { p1 };
punto p3 = p1;
punto p4 = punto(1.5, 2.5);
punto p5 = punto{2.5, 3.5};
\end{lstlisting}
    \begin{itemize}
      \item La copia por defecto es miembro a miembro.
    \end{itemize}
\end{itemize}
\end{frame}

\begin{frame}[fragile]{Implementación del constructor}
\begin{itemize}
  \item El contructor debe iniciar todos los datos miembro del tipo.
\begin{lstlisting}
punto::punto(double cx, double cy) {
  x = cx;
  y = cy;
}
\end{lstlisting}
  \item Se puede usar una lista de iniciación de miembros.
\begin{lstlisting}
punto::punto(double cx, double cy) : x{cx}, y{cy}
{
}
\end{lstlisting}
    \begin{itemize}
      \item Evita el uso de un miembro antes de iniciarse.
      \item Fácilmente extensible con herencia.
    \end{itemize}
\end{itemize}
\end{frame}

\begin{frame}{Construcción de fechas}
\begin{block}{fecha.h}
\lstinputlisting{06-deftipos/fecha4/fecha.h}
\end{block}
\end{frame}

\begin{frame}
\begin{block}{fecha.cpp}
\lstinputlisting[lastline=10]{06-deftipos/fecha4/fecha.cpp}
\end{block}
\end{frame}

\begin{frame}
\begin{block}{main.cpp}
\lstinputlisting{06-deftipos/fecha4/main.cpp}
\end{block}
\end{frame}

