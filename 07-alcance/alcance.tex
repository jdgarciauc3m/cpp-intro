\section{Tipos de alcance}

\begin{frame}[t,fragile]{Alcance}
\begin{itemize}
  \item Un \textgood{alcance} (\emph{scope}) es una región de código fuente.
    \begin{itemize}
      \item Normalmente delimitada por un par de llaves.
      \item El nombre de una entidad es \textemph{válido} 
            desde su \textmark{punto de declaración} hasta el
            \textmark{final del alcance} en que se produjo dicha delcaración.
      \item Se pueden anidar alcances.
    \end{itemize}
\end{itemize}
\begin{lstlisting}
int x; // Alcance global

void f() {
  int y; // Alcance local
  y = x;
  if (x == 0) {
    int z = 2; // Alcance anidado
    // ...
  }
}
\end{lstlisting}
\end{frame}

\begin{frame}[t]{Tipos de alcance}
\begin{itemize}
  \item \textmark{Alcance global}: No incluido en ningún otro.
  \item \textmark{Alcance de espacio de nombres}: Alcance con un nombre asociado.
    \begin{itemize}
      \item Puede estar anidado en el alcance global o en otro espacio de nombres.
    \end{itemize}
  \item \textmark{Alcance de clase}: Alcance de la definición de una clase.
  \item \textmark{Alcance local}: Alcance de un bloque delimitado por un par de llaves.
  \item \textmark{Alcance de sentencia}: Alcance vinculado a algunas sentencias (p.ej. bucles-\emph{for}).
\end{itemize}
\end{frame}

\begin{frame}[t,fragile]{Alcance global}
\begin{itemize}
  \item Las entidades declaradas en el \textgood{alcance global} 
        son \textmark{accesibles} desde cualquier entidad del programa.
  \item Una entidad con \textgood{alcance global}
        puede ser accedida desde su \textmark{punto de declaración} 
        hasta el \textmark{final de la unidad de traducción} en que se declaró.
\end{itemize}
\begin{lstlisting}
#include <iostream>

int x; // Global

void f() {
  std::cout << x << "\n";
}
\end{lstlisting}
\end{frame}

\begin{frame}[t,fragile]{Alcance de espacio de nombres}
\begin{itemize}
  \item Declaración \textmark{dentro} de \textgood{espacio de nombre}:
    \begin{itemize}
      \item Accesible dentro de ese espacio (y anidados).
      \item Desde el punto de declaración hasta final de unidad de traducción
      \item Desde fuera de espacio con acceso cualificado.
    \end{itemize}
\end{itemize}

\mode<presentation>{\vfill}
\begin{lstlisting}
namespace geom {
  double area(double l, double a);
  double volumen(double nx, double ny, double nz) {
    return area(nx,ny) * nz;
  }
} // Fin de espacio de nombres

void f() {
  double a1 = geom::area(2,3);
  double a2 = volumen(2,3,4); // Error: volumen no esta en alcance
  // ...
}
\end{lstlisting}
\end{frame}

\begin{frame}[t,fragile]{Alcance local}
\begin{itemize}
  \item Dentro de una \textgood{función}:
    \begin{itemize}
      \item Accesibles solamente dentro de la función.
      \item Desde su punto de declaración hasta el final del bloque.
    \end{itemize}
  \item \textgood{Parámetros de función} $\Rightarrow$ se consideran 
        locales al bloque de primer nivel de la función.
\end{itemize}
\begin{lstlisting}
void f(int x, int y) {
  int z = (x>y) ? x : y;
  if (z>0) {
    int t = x - y;
    // ...
  } // Fin de alcance de t
} // Fin de alcance de x, y, z
\end{lstlisting}
\end{frame}

\begin{frame}[t,fragile]{Alcance de sentencia}
\begin{itemize}
  \item La entidades declaradas dentro de los paréntesis de sentencias 
        \cppkey{for}, \cppkey{if} o \cppkey{switch}
        tienen alcance de sentencia.
  \item Una entidad puede ser accedida solamente dentro de la sentencia correspondiente.
\end{itemize}
\begin{lstlisting}
void f(int n) {
  for (int i=0; i<n; ++i) {
    std::cout << i;
    std::cout << " -> ";
    int cuad = i*i;
    std::cout << cuad << "\n";
  } // Fin de alcance para i y cuad
}
\end{lstlisting}
\end{frame}


\begin{frame}[t,fragile]{Espacios de nombres}
\begin{itemize}
  \item Un \textgood{espacio de nombres} 
        permite agrupar un conjunto de entidades en un alcance con nombre.
\end{itemize}
\begin{lstlisting}
#include <iostream>

namespace geom {
  area_rect(double nx, double ny);
  area_circ(double r);
  // ...
}

void f() {
  std::cout << geom::area_rect(2.0, 3.5) << "\n";
}
\end{lstlisting}
\end{frame}

\begin{frame}[t,fragile]{Declaraciones de uso}
\begin{itemize}
  \item Declara un \textmark{sinónimo local} 
        para una entidad de otro espacio de nombres.
\end{itemize}
\begin{lstlisting}
#include <iostream>

namespace geom {
  area_rect(double nx, double ny);
  area_circ(double r);
  // ...
}

void f() {
  using std::cout;
  using geom::area_rect;
  cout << area_rect(2.0, 3.5) << "\n";
}
\end{lstlisting}
\end{frame}

\begin{frame}[t,fragile]{Directiva de uso}
\begin{itemize}
  \item Hace que se busquen los nombres no encontrados en otro espacio de nombres.
\end{itemize}
\begin{lstlisting}
#include <iostream>

namespace geom {
  area_rect(double nx, double ny);
  area_circ(double r);
  // ...
}

void f() {
  using namespace std;
  using namespace geom;
  cout << area_rect(2.0, 3.5) << "\n";
}
\end{lstlisting}
\begin{itemize}
  \item \textbad{Advertencia}: Nunca coloques una directiva de uso en un 
        archivo de cabecer en el espacio de nombres global.
\end{itemize}
\end{frame}
