\section{Paso de parámetros array}

\begin{frame}[t,fragile]{Cuestión de estilo}
\begin{itemize}
  \item \textmark{Estilo alternativo} a paso de dirección más tamaño:
\begin{lstlisting}
double suma(double * b, double * e) {
  double r{};
  while (b!=e) {
    r+= *b++;
  }
  return r;
}
\end{lstlisting}

  \mode<presentation>{\vfill\pause}
  \item \textgood{Invocación}:
\begin{lstlisting}
void f() {
  double v[] = { 1, 2, 3, 4 };
  cout << suma(v, v+4) << endl;
}
\end{lstlisting}

  \mode<presentation>{\vfill\pause}
  \item \textgood{Enfoque}: Paso de un \textmark{rango semiabierto}.
    \begin{itemize}
      \item \textmark{Inicio}: Puntero a primer elemento.
      \item \textmark{Fin}: Puntero al siguiente del último elemento.
    \end{itemize}
\end{itemize}
\end{frame}

\begin{frame}[t,fragile]{Constantes arrays y punteros}
\begin{itemize}
  \item El calificador \cppkey{const} indica la \textmark{inmutabilidad de un objeto}.
    \begin{itemize}
      \item Forma parte del tipo.
    \end{itemize}
\begin{lstlisting}
const int max = 4; // max es const int
const double v[max] = { 1, 2, 3, 4 }; // v[0] es const double
\end{lstlisting}

  \mode<presentation>{\vfill\pause}
  \item En el caso de un puntero realmente hay dos objetos:
    \begin{itemize}
      \item El \textmark{puntero} (dirección de memoria).
      \item El \textgood{objeto apuntado}.
    \end{itemize}

  \mode<presentation>{\vfill\pause}
  \item \textgood{Implicaciones}
    \begin{itemize}
      \item Si el puntero es constante $\Rightarrow$ 
            no se puede cambiar el lugar apuntado.
      \item Si el objeto apuntado es constante $\Rightarrow$
            no se puede cambiar el valor del objeto.
    \end{itemize}
\end{itemize}
\end{frame}

\begin{frame}[t,fragile]{Alternativas}
\mode<presentation>{\vspace{-0.5em}}
\begin{itemize}
  \item Puntero a valor constante:
\begin{lstlisting}
const double * p = v; // p apunta a v[0]
p = v + 2; // OK. Modificación del puntero
*p = 10; // Error: elementos constantes
\end{lstlisting}

  \mode<presentation>{\vfill\pause}
  \item Puntero a valor constante (notación alternativa):
\begin{lstlisting}
double const * p = v; // p apunta a v[0]
p = v + 2; // OK. Modificación del puntero
*p = 10; // Error: elementos constantes
\end{lstlisting}

  \mode<presentation>{\vfill\pause}
  \item Puntero constante a valor variable:
\begin{lstlisting}
double * const p = v; // p apunta a v[0]
p = v + 2; // Error no se puede apuntar a otro sitio
*p = 10; // OK: Modificación de objeto apuntado
\end{lstlisting}

  \mode<presentation>{\vfill\pause}
  \item Puntero constante a valor constante:
\begin{lstlisting}
const double * const p = v; // p apunta a v[0]
p = v + 2; // Error no se puede apuntar a otro sitio
*p = 10; // OK: Modificación de objeto apuntado
\end{lstlisting}
\end{itemize}
\end{frame}

\begin{frame}[t,fragile]{Parámetros de función constantes}
\begin{itemize}
  \item \cppkey{const} forma parte del tipo $\rightarrow$ se puede usar para restringir las operaciones que se
        pueden realizar sobre un parámetro.
\begin{lstlisting}
void copia(char * dest, const char * orig) {
  while (*orig) {
    *dest++ = *orig++;
  }
}
\end{lstlisting}
\end{itemize}
\end{frame}

