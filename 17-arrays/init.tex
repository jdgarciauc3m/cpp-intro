\section{Iniciación de arrays}

\begin{frame}[t,fragile]{Iniciadores}
\begin{itemize}
  \item Un array puede \textgood{iniciarse} con una \textmark{lista de valores}.
\begin{lstlisting}
double v[4] = { 1.0, 2.0, 3.5, 4.7 };
char c[4] = { 'c', '+', '+', 0 };
\end{lstlisting}

  \mode<presentation>{\vfill\pause}
  \item Se puede \textgood{omitir el tamaño} si hay un valor inicial.
\begin{lstlisting}
double v[] = {1.2, 2.4, 3.6 }; // Tamaño = 3
char c[] = { 'H', 'o', 'l', 'a' }; // Tamaño = 4
\end{lstlisting}

  \mode<presentation>{\vfill\pause}
  \item Si \textbad{faltan iniciadores}, \textgood{se completa} con \cppid{0}.
\begin{lstlisting}
double v[4] = {1.5, 2.5}; // v[2]==0.0, v[3]==0.0
\end{lstlisting}

  \mode<presentation>{\vfill\pause}
  \item No se pueden especificar \textbad{más iniciadores de los necesarios}.
\begin{lstlisting}
char c[3] = { 'H', 'o', 'l', 'a' };
\end{lstlisting}

  \mode<presentation>{\vfill\pause}
  \item Iniciación a \textgood{0}.
\begin{lstlisting}
double v[10] = {};
\end{lstlisting}
\end{itemize}
\end{frame}

\begin{frame}[t,fragile]{Iniciación de cadenas}
\begin{itemize}
  \item Un \textmark{literal de cadena} equivale a 
        un \textgood{array de caracteres constantes} con 
        un \textemph{carácter adicional} para el terminador (\textgood{0}). 
\begin{lstlisting}
const char c1[5] = "Hola";
const char c2[5] = {'H', 'o', 'l', 'a', 0 };
\end{lstlisting}

  \mode<presentation>{\vfill\pause}
  \item Se puede asignar un literal de cadena a un \textmark{array no constante}.
\begin{lstlisting}
char c[] = "Hola";
c[0] = 'C';
c[2] = 's';
\end{lstlisting}

  \mode<presentation>{\vfill\pause}
  \item Se puede iniciar con un literal un \textmark{puntero a carácter constante}.
\begin{lstlisting}
const char * p = "Hola";
\end{lstlisting}

  \mode<presentation>{\vfill\pause}
  \item Pero no un puntero a carácter no constante.
\begin{lstlisting}
char * p = "Hola"; // Error
\end{lstlisting}
\end{itemize}
\end{frame}

\begin{frame}[t,fragile]{Otros tipos de cadenas}
\begin{itemize}
  \item Cadenas en modo \textemph{crudo} (\emph{raw}):
\begin{lstlisting}
cout << R"(\t")"; // Imprime '\', 't', '"'
\end{lstlisting}

  \mode<presentation>{\vfill\pause}
  \item Cadenas UTF:
    \begin{itemize}
      \item Cadenas UTF-8.
        \begin{itemize}
          \item \cppid{u8"Hola"};
        \end{itemize}
      \item Cadenas UTF-16.
        \begin{itemize}
          \item \cppid{u"Hola"};
        \end{itemize}
      \item Cadenas UTF-32.
        \begin{itemize}
          \item \cppid{U"Hola"};
        \end{itemize}
    \end{itemize}
\end{itemize}
\end{frame}

\begin{frame}[t,fragile]{Iniciación de array como dato miembro}
\begin{itemize}
  \item Un dato miembro que sea de tipo array \textgood{se puede iniciar} 
        con una \textmark{lista de valores}.
\begin{lstlisting}
class conversor {
private:
  double coef[4];
public:
  conversor(double x, double y, double z, double t) : coef{x,y,z,t} {}
  double convierte(double a) {
    return coef[0] * a + coef[1] * a + coef[2] * a + coef[3] * a;
  }
};

void f() {
  conversor c{1.0, 0.0, -1, 2};
  cout << c.convierte(3) << "\n";
}
\end{lstlisting}
\end{itemize}
\end{frame}
