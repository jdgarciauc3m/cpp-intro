\section{Arrays y aritmética de punteros}

\begin{frame}[t,fragile]{Punteros a posiciones de un array}
\begin{itemize}
  \item Un objeto de tipo \textmark{array} \textgood{se puede usar} 
        como un \textmark{puntero} a la primera posición del array.
\begin{lstlisting}
double v[4];
*v = 1.5;
double x = v[0]; // x == 1.5
\end{lstlisting}

  \mode<presentation>{\vfill\pause}
  \item Se puede \textgood{iniciar} un puntero a \textmark{cualquier posición} del array.
\begin{lstlisting}
double v[4];
double * p = &v[1];
p = &v[3];
\end{lstlisting}

  \mode<presentation>{\vfill\pause}
  \item También se puede apuntar a la \textmark{posición siguiente a la última}.
    \begin{itemize}
      \item Pero no a posiciones anteriores o posteriores.
    \end{itemize}
\begin{lstlisting}
double v[4];
double * p = &v[-1]; // Comportamiento no definido
double * p = &v[7]; // Comportamiento no definido
\end{lstlisting}
\end{itemize}
\end{frame}

\begin{frame}[t,fragile]{Aritmética de punteros}
\begin{itemize}
  \item Se puede \textgood{avanzar} con un puntero en un array, 
        \textmark{sumando un entero} al puntero.
\begin{lstlisting}
double v[4] = {}
double * p = v + 1; // Apunta a v[1]
p = p + 2; // Apunta a v[3];
++p; // Apunta a v[4]
\end{lstlisting}

  \mode<presentation>{\vfill\pause}
  \item Igualmente se puede \textgood{retroceder};
\begin{lstlisting}
--p; // Apunta a v[3]
p = p - 2; // Apunta a v[1]
p -=1; // Apunta a v[0]
\end{lstlisting}

  \mode<presentation>{\vfill\pause}
  \item Se puede \textgood{calcular la distancia} entre dos punteros 
        como número de elementos.
    \begin{itemize}
      \item Solamente si apuntan a objetos del mismo array.
    \end{itemize}
\begin{lstlisting}
double v[4] = {};
int n = &v[4] - &v[0]; // n == 4
int m = &v[4] - &w[2]; // No definido
\end{lstlisting}
\end{itemize}
\end{frame}

\begin{frame}[t,fragile]{Conversión}
\begin{itemize}
  \item Existe una \textgood{conversión implícita} 
        del \textmark{nombre de un array a puntero}.
\begin{lstlisting}
double v[4] = {};
f(v); // Pasa a f el puntero &v[0]
\end{lstlisting}

  \mode<presentation>{\vfill\pause}
  \item Esto \textbad{impide el paso por valor} de arrays a funciones.
    \begin{itemize}
      \item \textmark{Objetivo}: Evitar paso accidental por copia de arrays grandes.
    \end{itemize}
\begin{lstlisting}
double f(double vec[], int n) {
  double r{}; // r = 0.0
  for (int i = 0; i<n; ++i) {
    r += vec[i];
  }
  return r;
}
\end{lstlisting}
\end{itemize}
\end{frame}

\begin{frame}[t,fragile]{Conversión}
\begin{itemize}
  \item El resultado de la conversión es un valor, no una variable.
\begin{lstlisting}
double v[4] = {};
v = new double[4]; // Error: v no es una variable puntero

double * w = nullptr;
w = new double[4]; // OK
\end{lstlisting}
    \begin{itemize}
      \item Los punteros y arrays no son siempre intercambiables.
    \end{itemize}
\end{itemize}
\end{frame}
